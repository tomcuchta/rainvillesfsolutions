%%%%
%%
%%
%%%%
%%%% CHAPTER 4
%%%% CHAPTER 4
%%%%
%%
%%
%%%%
\section{Chapter 4 Solutions}
\begin{center}\hyperref[toc]{\^{}\^{}}\end{center}
\begin{center}\begin{tabular}{cccccccccccccc}
\hyperref[problem1chapter4]{P1} & \hyperref[problem2chapter4]{P2} & \hyperref[problem3chapter4]{P3} & \hyperref[problem4chapter4]{P4} & \hyperref[problem5chapter4]{P5} & \hyperref[problem6chapter4]{P6} & \hyperref[problem7chapter4]{P7} & \hyperref[problem8chapter4]{P8} & \hyperref[problem9chapter4]{P9} & \hyperref[problem10chapter4]{P10} & \hyperref[problem11chapter4]{P11} & \hyperref[problem12chapter4]{P12} & \hyperref[problem13chapter4]{P13} \\
\hyperref[problem14chapter4]{P14} & \hyperref[problem15chapter4]{P15} & \hyperref[problem16chapter4]{P16} & \hyperref[problem17chapter4]{P17} & \hyperref[problem18chapter4]{P18} & \hyperref[problem19chapter4]{P19} & \hyperref[problem20chapter4]{P20} & \hyperref[problem21chapter4]{P21} & \hyperref[problem22chapter4]{P22} & \hyperref[problem23chapter4]{P23} 
\end{tabular}\end{center}

\setcounter{problem}{0}
\setcounter{solution}{0}

\begin{problem}\label{problem1chapter4}
Show that 
$$\dfrac{\mathrm{d}}{\mathrm{d}x} F \left[ \begin{array}{rlr}
a, b; & & \\
& & x \\
c; & & \end{array} \right] = \dfrac{ab}{c} F \left[ \begin{array}{rlr}
a+1,b+1 ; & & \\
& & x \\
c+1; 
\end{array} \right].$$
\end{problem}
\begin{solution}
From
$$F(a,b;c;1) \equiv \displaystyle\sum_{n=0}^{\infty} \dfrac{(a)_n (b)_n}{(c)_n n!},$$
we get
$$\begin{array}{ll}
\dfrac{\mathrm{d}}{\mathrm{d}x} F(a,b;c;x) &= \displaystyle\sum_{n=1}^{\infty} \dfrac{(a)_n(b)_n x^{n-1}}{{(c)_n(c-1)!}} \\
&= \displaystyle\sum_{n=0}^{\infty} \dfrac{(a)_{n+1} (b)_{n+1} x^n}{(c)_{n+1} n!} \\
&= \dfrac{ab}{c} \displaystyle\sum_{n=0}^{\infty} \dfrac{(a+1)_n (b+1)_n x^n}{(c+1)_n n!} \\
&= \dfrac{ab}{c} F(a+1,b+1; c+1; x).
\end{array}$$
\end{solution}
%%%%
%%
%%
%%%%
\begin{problem}\label{problem2chapter4}
Show that
$$F \left[ \begin{array}{rlr} 
2a, 2b; & & \\
& & \dfrac{1}{2} \\
a + b + \dfrac{1}{2};
\end{array} \right] = \dfrac{\Gamma \left( a + b + \dfrac{1}{2} \right) \Gamma \left( \dfrac{1}{2} \right)}{\Gamma \left( \dfrac{1}{2}c + \dfrac{1}{2}a \right) \Gamma \left( \dfrac{1}{2} c - \dfrac{1}{2} a + \dfrac{1}{2} \right) }.$$
\end{problem}
\begin{solution}
Wish to evaluate $F \left[ \begin{array}{rlr}
2a,2b; & & \\
& & \dfrac{1}{2} \\
a+b+\dfrac{1}{2}; & & 
\end{array} \right]$. 
From Chapter~4, Theorem~25, 
$$F \left[ \begin{array}{rlr}
2a,2b; & & \\
& & x \\
a+b+\dfrac{1}{2} & & 
\end{array} \right] = F \left[ \begin{array}{rlr} 
a, b ; & & \\
& & 4x(1-x) \\
a+ b + \dfrac{1}{2} & & 
\end{array} \right]$$
for $|x| < 1$, $|4x(1-x)| < 1$. We need to use $x = \dfrac{1}{2}$, but note that $\mathrm{Re}(a+b+\dfrac{1}{2}-a-b) > 0$. Hence, by Chapter~4, Theorem~18,
$$F \left[ \begin{array}{rlr}
2a,2b; & & \\
& & \dfrac{1}{2} \\
a+b+\dfrac{1}{2} & &
\end{array} \right] = F \left[ \begin{array}{rlr}
a,b; & & \\
& & 1 \\
a+b+\dfrac{1}{2} & &
\end{array} \right] = \dfrac{\Gamma \left(a+b+\dfrac{1}{2} \right) \Gamma \left( \dfrac{1}{2} \right) }{\Gamma \left( b+ \dfrac{1}{2} \right) \Gamma \left( a + \dfrac{1}{2} \right)}.$$
\end{solution}
%%%%
%%
%%
%%%%
\begin{problem}\label{problem3chapter4}
Show that 
$$F \left[ \begin{array}{rlr}
a, 1-a; & & \\
& & \dfrac{1}{2} \\
c; & &
\end{array} \right] = \dfrac{2^{1-c}\Gamma(c) \Gamma \left( \dfrac{1}{2} \right)}{\Gamma \left( \dfrac{1}{2}c + \dfrac{1}{2} a \right) \Gamma \left( \dfrac{c-a+1}{2} \right)}$$
\end{problem}
\begin{solution}
Consider 
$$F \left[ \begin{array}{rlr}
a,1-a; & & \\
& & x \\
c; & & 
\end{array} \right] = (1-x)^{c-1} F \left[ \begin{array}{rlr}
\dfrac{c-1}{2}, \dfrac{c+a-1}{2}; & & \\
& & 4x(1-x) \\
c ; & & 
\end{array} \right].$$
By Chapter~4, Theorem~27, for $|x| < 1$, $|4x(1-x)| < 1$,
$$F \left[ \begin{array}{rlr}
a,1-a; & & \\
& & x \\
c; & &
\end{array} \right] = (1-x)^{c-1} F \left[ \begin{array}{rlr}
\dfrac{c-a}{2}, \dfrac{c+a-1}{2}; & & \\
& & 4x(1-x) \\
c; & &
\end{array} \right].$$
Since $\mathrm{Re} \left( c - \dfrac{c}{2} + \dfrac{a}{2} - \dfrac{c}{2} - \dfrac{a}{2} + \dfrac{1}{2} \right) > 0$, we may use Chapter~4, Theorem~18 to conclude that
$$\begin{array}{ll}
F \left[ \begin{array}{rlr}
a,1-a; & & \\
& & \dfrac{1}{2} \\
c; & & 
\end{array} \right] &= \left( \dfrac{1}{2} \right)^{c-1} F \left[ \begin{array}{rlr}
\dfrac{c-1}{2}, \dfrac{c+a-1}{2}; & & \\
& & 1 \\
c; & & 
\end{array} \right] \\
&= \dfrac{ \Gamma(c) \Gamma \left( \dfrac{1}{2} \right)}{2^{c-1} \Gamma \left( \dfrac{c+a}{2} \right) \Gamma \left( \dfrac{c-a+1}{2} \right)},
\end{array}$$
as desired.
\end{solution}
%%%%
%%
%%
%%%%
\begin{problem}\label{problem4chapter4}
Obtain the result
$$F \left[ \begin{array}{rlr}
-n,b; & & \\
& & 1 \\
c; & &
\end{array} \right] = \dfrac{(c-b)_n}{(c)_n}.$$
\end{problem}
\begin{solution}
Consider $F(-n,b;c;1).$
At once, if $\mathrm{Re}(c-b)>0$,
$$F(-n,b;c;1) = \dfrac{\Gamma(c) \Gamma(c-b+n)}{\Gamma(c+n)\Gamma(c-b)} = \dfrac{(c-b)_n}{(c)_n}.$$
Actually the condition $\mathrm{Re}(c-b)>0$ is not necessary because of the termination of the series involved.
\end{solution}
%%%%
%%
%%
%%%%
\begin{problem}\label{problem5chapter4}
Obtain the result 
$$F \left[ \begin{array}{rlr}
-n,a+n; & & \\
& & 1 \\
c; & 
\end{array} \right] = \dfrac{(-1)^n (1+a-c)_n}{(c)_n}.$$
\end{problem}
\begin{solution}
$$F \left[ \begin{array}{rlr}
-n, a+n; & & \\
& & 1 \\
c; & &
\end{array} \right] = \dfrac{\Gamma(c) \Gamma(c-a)}{\Gamma(c+n) \Gamma(c-a-n)}.$$
By Exercise 9, Chapter 2, if $(c-a)$ is nonintegral,
$$\dfrac{\Gamma(1-\alpha-n)}{\Gamma(1-\alpha)} = \dfrac{(-1)^n}{(\alpha)_n}.$$
Hence,
$$F \left[ \begin{array}{rlr}
-n, a+n; & & \\
& & 1 \\
a; & &
\end{array} \right] = \dfrac{(-1)^n ( 1-c+a)_n}{(c)_n},$$
as desired.
\end{solution}
%%%%
%%
%%
%%%%
\begin{problem}\label{problem6chapter4}
Show that
$$F \left[ \begin{array}{rlr}
-n 1-b-n; & & \\
& & 1 \\
a; & &
\end{array} \right] = \dfrac{(a+b-1)_{2n}}{(a)_n(a+b-1)_n}.$$
\end{problem}
\begin{solution}
$$F \left[ \begin{array}{rlr}
-n,1-b-n; & & \\
& & 1 \\
a; & & 
\end{array} \right] = \dfrac{\Gamma(a) \Gamma(a-1+b+2n)}{\Gamma(a+n)\Gamma(a-1+b+n)} = \dfrac{(a-1+b)_{2n}}{(a)_n (a-1+b)_n}.$$
Of course $a \neq$ nonpositive integer, as usual.
\end{solution}
%%%%
%%
%%
%%%%
\begin{problem} \label{problem7chapter4}
Prove that if $g_n = F(-n, \alpha; 1 +\alpha-n; 1)$ and $\alpha$ is not an integer, then $g_n=0$ for $n \geq 1, g_0=1$.
\end{problem}
\begin{solution}
Let $g_n = F(-n,\alpha;1+\alpha-n;1).$
Then
$$g_n = \displaystyle\sum_{k=0}^n \dfrac{(-n)_k (\alpha)_k}{k! (1+\alpha-n)_k} = \displaystyle\sum_{k=0}^n \dfrac{n! (\alpha)_k (-\alpha)_k}{n! (n-k)! (\alpha)_n}.$$
Hence, compute the series
$$\begin{array}{ll}
\displaystyle\sum_{n=0}^{\infty} \dfrac{(-\alpha)_n g_n t^n}{n!} &= \displaystyle\sum_{n=0}^{\infty} \displaystyle\sum_{k=0}^n \dfrac{(\alpha)_k (-\alpha)_{n-k} t^n}{k! (n-k)!} \\
&= \left( \displaystyle\sum_{n=0}^{\infty} \dfrac{(\alpha)_n t^n}{n!} \right) \left( \displaystyle\sum_{n=0}^{\infty} \dfrac{(-\alpha)_n t^n}{n!} \right) \\
&= (1-t)^{\alpha} (1-t)^{-\alpha} \\
&= 1.
\end{array}$$
Therefore, $g_0=1$ and $g_n=0$ for $n \geq 1$.
(Note: easiest to choose $\alpha$ to not be an integer; can actually do better than that probably.)
\end{solution}
%%%%
%%
%%
%%%%
\begin{problem}\label{problem8chapter4}
Show that
$$\dfrac{\mathrm{d}^n}{\mathrm{d}x^n} \left[ x^{a-1+n} F(a,b;c;x) \right] = (a)_x x^{a-1} F(a+n, b; c; x).$$
\end{problem}
\begin{solution}
Consider $\mathcal{D}^n[x^{a-1+n}F(a,b;c;x)]$ ($\mathcal{D} \equiv \dfrac{\mathrm{d}}{\mathrm{d}x}$).
We have
\begin{eqnarray*}
\lefteqn{\mathcal{D}^n[x^{a-1+n}F(a,b;c;x)]} \\
& &= \mathcal{D}^n \displaystyle\sum_{k=0}^{\infty} \dfrac{(a)_k (b)_k x^{n+k+a-1}}{(c)_k k!} \\
& &= \displaystyle\sum_{k=0}^{\infty} \dfrac{(a)_k (n+k+a-1)(n+k+a-2) \ldots (k+a) x^{k+1-a} (b)_k}{(c)_k k!} \\
& &= \displaystyle\sum_{k=0}^{\infty} \dfrac{(a)_k (a)_{n+k} x^{k+a-1}(b)_k}{(c)_k (a)_k k!} \\
& &= \displaystyle\sum_{k=0}^{\infty} \dfrac{(a+k)_n (a)_n x^{k+a-1} (b)_k}{k! (c)_k} \\
& &= (a)_n x^{a-1} F(a+n,b;c;x).
\end{eqnarray*}
\end{solution}
%%%%
%%
%%
%%%%
\begin{problem}\label{problem9chapter4}
Use pg.~66 (2) with $z = -x, b=-n,$ in which $n$ is a non-negative integer, to conclude that
$$F \left[ \begin{array}{rlr}
-n,a; & & \\
& & -x \\
1+a+n;
\end{array} \right] = (1-x)^{-a} F \left[ \begin{array}{rlr}
\dfrac{1}{2}a,\dfrac{1}{2}a+\dfrac{1}{2}; & & \\
& & \dfrac{-4x}{(1-x)^2} \\
1+a+n ;
\end{array} \right].$$
\end{problem}
\begin{solution}
From pg.~66 (2), we get
$$(1+z)^{-a} F \left[ \begin{array}{rlr}
\dfrac{a}{2}, \dfrac{a+1}{2}; & & \\
& & \dfrac{-4z}{(1+z)^2} \\
1+a-b; & & 
\end{array} \right] = F \left[ \begin{array}{rlr}
a,b;
& & z \\
1+a-b;
\end{array} \right].$$
Use $z = -x$, $b = -n$ to arrive at 
$$(1-x)^{-a} F \left[ \begin{array}{rlr}
\dfrac{a}{2}, \dfrac{a+1}{2}; & & \\
& & \dfrac{-4x}{(1-x)^2} \\
1+a+n; & & 
\end{array} \right] = F \left[ \begin{array}{rlr}
a, -n; & & \\
& & -x \\
1+a+n; & &
\end{array} \right],$$
as desired. 
\end{solution}
%%%%
%%
%%
%%%%
\begin{problem}\label{problem10chapter4}
In Theorem~23, page~65, put $b= \gamma$, $a = \gamma + \dfrac{1}{2}$, $4x(1+x)^{-2} = z$ and thus prove that

$$F \left[ \begin{array}{rlr}
\gamma, \gamma + \dfrac{1}{2}; & & \\
& & z \\
2 \gamma
\end{array} \right] = (1-z)^{\frac{1}{2}} \left[ \dfrac{2}{1+\sqrt{1-z}} \right]^{2 \gamma - 1}.$$
\end{problem}
\begin{solution}
Chapter~4, Theorem~23 gives us 
$$(1+x)^{-2a} F \left[ \begin{array}{rlr}
a,b; & & \\
& & \dfrac{4x}{(1+x)^2} \\
ab; & &
\end{array} \right] = F \left[ \begin{array}{rlr}
a, a-b+\dfrac{1}{2}; & & \\
& & x^2 \\
b + \dfrac{1}{2}; & & 
\end{array} \right].$$
Put $b = \gamma$, $a = \gamma + \dfrac{1}{2}$ and
$$\dfrac{4x}{(1+x)^2} = z.$$
Then
$$zx^2+2(z-2)x+z=0$$
$$zx = 2 - 3 \pm \sqrt{z^2-4z+4-3^2} = 2-z \pm 2 \sqrt{1-z}.$$
Now $x=0$ when $z=0$, so
$$zx = 2 - z - 2 \sqrt{1-z} = 1 - z + 1 - 2\sqrt{1-z}$$
or
$$x = \dfrac{(1-\sqrt{1-z})^2}{z} = \dfrac{(1-\sqrt{1-z} [1 - (1-z)]}{z ( 1 + \sqrt{1-z})}.$$
Thus
$$x = \dfrac{1 - \sqrt{1-z}}{1 + \sqrt{1-z}}$$
and
$$1 + x = \dfrac{2}{1+\sqrt{1-z}}.$$
Then we obtain
$$\dfrac{4x}{(1+x)^2} = \dfrac{4(1-\sqrt{1-z})}{1+\sqrt{1-z}} \cdot \dfrac{(1+\sqrt{1-z})^2}{4} = z,$$
a check. 
Now with $b= \gamma, a = \gamma + \dfrac{1}{2}$, Chapter~4, Theorem~23 yields
$$\begin{array}{ll}
\left( \dfrac{2}{1+\sqrt{1-z}} \right)^{-2\gamma-1} F \left[ \begin{array}{rlr}
\gamma + \dfrac{1}{2}, \gamma; & & \\
& & z \\
2 \gamma; & & 
\end{array} \right] &= F \left[ \begin{array}{rlr}
\gamma + \dfrac{1}{2}, 1; & & \\
& & x^2 \\
\gamma + \dfrac{1}{2}; & &
\end{array} \right] \\
&= _1 \!\!F_0 \left[ \begin{array}{rlr}
1; & &\\
& & x^2 \\
-; & &
\end{array} \right] \\
&= (1-x^2)^{-1}.
\end{array}$$
Since $1 - x = \dfrac{2 \sqrt{1-z}}{1 + \sqrt{1-z}}$ and $1 + x = \dfrac{2}{1 + \sqrt{1-z}}$,
$$(1-x^2) = \dfrac{4 \sqrt{1-z}}{(1+\sqrt{1-z})^2}.$$
Thus we have
$$\begin{array}{ll}
F \left[ \begin{array}{rlr}
\gamma, \gamma + \dfrac{1}{2}; & & \\
& & z \\
2 \gamma; & & 
\end{array} \right] &= \left( \dfrac{2}{1 + \sqrt{1-z}} \right)^{2\gamma+1} \left( \dfrac{2}{1+\sqrt{1-z}} \right)^{-2} (1-z)^{-\frac{1}{2}} \\
&= (1-z)^{-\frac{1}{2}} \left( \dfrac{2}{1+\sqrt{1-z}} \right)^{2 \gamma-1},
\end{array}$$
as defined. Now we use Chapter~4, Theorem~21 to see that
$$F \left[ \begin{array}{rlr}
\gamma, \gamma + \dfrac{1}{2}; & & \\
& & 2 \\
2 \gamma; & &
\end{array} \right] = (1-z)^{-\frac{1}{2}} F \left[ \begin{array}{rlr}
\gamma, \gamma - \dfrac{1}{2}; & & \\
& & z \\
2 \gamma; & &
\end{array} \right]$$
so that we also get
$$F \left[ \begin{array}{rlr}
\gamma, \gamma - \dfrac{1}{2}; & & \\
& & z \\
2 \gamma; & & 
\end{array} \right] = \left( \dfrac{2}{1 + \sqrt{1-z}} \right)^{2 \gamma -1},$$
as desired.
\end{solution}
%%%%
%%
%%
%%%%
\begin{problem}\label{problem11chapter4}
Use Chapter~4, Theorem~27 to show that
\begin{eqnarray*}
\lefteqn{(1-x)^{1-c} F \left[ \begin{array}{rlr}
a, 1-a; & & \\
& & x \\
c; & 
\end{array} \right]} \\
&&  = (1-2x)^{a-c} F \left[ \begin{array}{rlr}
\dfrac{1}{2}c - \dfrac{1}{2}a, \dfrac{1}{2}c - \dfrac{1}{2}a + \dfrac{1}{2}; & & \\
& & \dfrac{4x(x-1}{(1-2x)^2} \\
c;
\end{array} \right].
\end{eqnarray*}
\end{problem}
\begin{solution}
By Chapter~4, Theorem~27,
\begin{eqnarray*}
\lefteqn{(1-x)^{1-c} F \left[ \begin{array}{rlr}
a, 1-a; & & \\
& & x \\
a; & & 
\end{array} \right]} \\
& &= F \left[ \begin{array}{rlr}
\dfrac{c-a}{2}, \dfrac{c+a-1}{2}; & & \\
& & 4x(1-x) \\
c; & & 
\end{array} \right] \\
& &= F \left[ \begin{array}{rlr}
\dfrac{c-a}{2}, \dfrac{c+a-1}{2}; & & \\
& & 1 - (1-2x)^2 \\
c; & & 
\end{array} \right] \\
& &= (1-2x)^{2 \dfrac{a-c}{2}} F \left[ \begin{array}{rlr}
\dfrac{c-a}{2}, c-\dfrac{c-a+1}{2}; & & \\
& & \dfrac{-1 + (1-2x)^2}{(1-2x)^2} \\
c; & & 
\end{array} \right] \\
& &= (1-2x)^{a-c} F \left[ \begin{array}{rlr}
\dfrac{c-a}{2}, \dfrac{c-a+1}{2}; & & \\
& & \dfrac{4x(x-1)}{(1-2x)^2} \\
c; & &
\end{array} \right],
\end{eqnarray*}
which we wished to obtain.
\end{solution}
%%%%
%%
%%
%%%%
\begin{problem}\label{problem12chapter4}
In the differential equation (3), page~54, for 
$$w = F(a,b;c;z)$$
introduce a new dependent variable $u$ by $w = (1-z)^{-a}u$, thus obtaining
$$z(1-z)^2u'' + (1-z)[c+(a-b-1)z]u' + a(c-b)u = 0.$$
Next change the independent variable to $x$ by putting $x = \dfrac{-z}{1-z}$. Show that the equation for $u$ in terms of $x$ is
$$x(1-x)\dfrac{\mathrm{d}^2u}{\mathrm{d}x^2} + [ c - (a+c-b+1)x] \dfrac{\mathrm{d}u}{\mathrm{d}x} - a(c-b)u = 0,$$
and thus derive the solution
$$w = (1-z)^{-a} F \left[ \begin{array}{rlr}
a, c-b; & & \\
& & \dfrac{-z}{1-z} \\
c; & &
\end{array} \right].$$
\end{problem}
\begin{solution}
We know that $w = F(a,b;c;z)$ is a solution of the equation
$$(1) z(1-z)w'' + [c-(a+b+1)z]w' - abw = 0.$$
In $(1)$ put $w = (1-z)^{-a}u$. Then
$$w' = (1-z)^{-a}u' + a(1-z)^{-a-1}u,$$
$$w'' = (1-z)^{-a}u'' + 2a(1-z)^{-a-1}u' + a(a+1)(1-z)^{-a-2}u.$$
Hence the new equation is
\begin{eqnarray*}
\lefteqn{z(1-z)u'' + 2azu' + a(a+1)z(1-z)^{-1}u + cu' + ca(1-z)^{-1}u} \\
& & - (a+b+1)zu' - a(a+b+1)z(1-z)^{-1}u - abu = 0,
\end{eqnarray*}
or
$$z(1-z)u'' + [c - (b-a+1)z]u' + (1-z)^{-1} [(a^2+a)z + ca - (a^2+ab+a)z - ab(1-z)]u=0,$$
or
$$(2) z(1-z)^2u'' + (1-z)[c+(a-b-1)z] u' + a(c-b)u = 0.$$
Now put $x = \dfrac{-z}{1-z}$. Then $z = \dfrac{-x}{1-x}, 1-z = \dfrac{1}{1-x},$ and we use equation (12) on page 12 of IDE for the change of variable.
First, $\dfrac{\mathrm{d}x}{\mathrm{d}z} = \dfrac{-1}{(1-z)^2} = -(1-x)^2$: $\dfrac{\mathrm{d}^2x}{\mathrm{d}z^2} = \dfrac{-2}{(1-z)^3} = -2(1-x)^3.$
The old equation $(2)$ above may be written
$$\dfrac{\mathrm{d}^2u}{\mathrm{d}z^2} + \left[ \dfrac{c}{z(1-z)} + \dfrac{a-b-1}{1-z} \right] \dfrac{\mathrm{d}u}{\mathrm{d}t} + \dfrac{a(c-b)}{z(1-z)^2}u = 0,$$
which then leads to the new equation
\begin{eqnarray*}
\lefteqn{(1-x)^4 \dfrac{\mathrm{d}^2u}{\mathrm{d}x^2} + \left[ -2(1-x)^3 \phantom{\dfrac{1}{1}} \right.} \\
& & \left. - (1-x)^2 \left\{ \dfrac{c(1-x)^2}{-x} + (a-b-1)(1-x) \right\} \right] \dfrac{\mathrm{d}u}{\mathrm{d}x} - \dfrac{a(c-b)(1-x)^3}{x}u = 0,
\end{eqnarray*}
or
$$x(1-x) \dfrac{\mathrm{d}^2u}{\mathrm{d}x^2} + \left[ -2x - \left\{ -c (1-x) + (a-b-1)x \right\} \right] \dfrac{\mathrm{d}u}{\mathrm{d}x} - a(c-b)u = 0,$$
or
$$(3) x(1-x) \dfrac{\mathrm{d}^2u}{\mathrm{d}x^2} + [x - (a-b+c+1)x ] \dfrac{\mathrm{d}u}{\mathrm{d}x} - a(c-b)u =0.$$
Now $(3)$ is a hypergeometric equation with parameters $\gamma = c$, $\alpha + \beta + 1 = a - b + c + 1, \alpha \beta = a(c-b)$. 
Hence $\alpha=a, \beta = c-b, \gamma =c$. One solution of $(3)$ is
$$u = F(a,c-b;c;x),$$
so one solution of equation $(1)$ is 
$$W = (1-z)^{-a} F \left[ \begin{array}{rlr}
a, c-b; & & \\
& & \dfrac{-z}{1-z} \\
c; & &
\end{array} \right].$$
\end{solution}
%%%%
%%
%%
%%%%
\begin{problem}\label{problem13chapter4}
Use the result of Exercise 12 and the method of Section 40 to prove Chapter~4, Theorem~20. 
\end{problem}
\begin{solution}
We know that in the region in common to $|z|<1$ and $\left| \dfrac{z}{1-z} \right| < 1$, there is a relation
$$(1-z)^{-a} F \left[ \begin{array}{rlr}
a,c-b; & & \\
& & \dfrac{-z}{1-z} \\
c; & & 
\end{array} \right] = AF(a,b;c;z) + Bz^{1-c}F(a+1-c,b+1-c;z-c;z).$$
Since $c$ is neither zero nor a negative integer, the last term is not analytic at $z=0$. Hence $B=0$. Then use $z=0$ to obtain $1 \cdot 1 = A \cdot 1$, so $A = 1$. Hence
$$F(a;b;c;z) = (1-z)^{-a} F \left[ \begin{array}{rlr}
a,c-b; & & \\
& & \dfrac{-z}{1-z} \\
c; & & 
\end{array} \right].$$
\end{solution}
%%%%
%%
%%
%%%%
\begin{problem}\label{problem14chapter4}
Prove Chapter~4, Theorem~21 by the method suggested by Exercises 12 and 13. 
\end{problem}
\begin{solution}(Solution by Leon Hall)
Note that the first two parameters in $F(a,b;c;z)$ are interchangeable, so results involving one of them also apply to the other. By Exercise~12,
$$F(a,b;c;z) = (1-z)^{-a} F \left( a,c-b;c; \dfrac{-z}{1-z} \right).$$
Let $w = \dfrac{-z}{1-z}$ so this becomes
$$F(a,b;c;z) = (1-z)^{-a} F(a,c-b;c;w).$$
Again, by Exercise~12, applied to the second parameter,
$$F(a,b;c;z) = (1-z)^{-a}(1-w)^{-(c-b)}F \left(c-a,c-b;c; \dfrac{-w}{1-w} \right).$$
But $1-w=(1-z)^{-1}$, and $\dfrac{-w}{1-w} = z$, so
$$\begin{array}{ll}
F(a,b;c;z) &= (1-z)^{-a}((1-z)^{-1})^{-(c-b)}F(c-a,c-b;c,z) \\
&=(1-z)^{c-a-b}F(c-a,c-b;c;z)
\end{array}$$
as desired.
\end{solution}
%%%%
%%
%%
%%%%
\begin{problem}\label{problem15chapter4}
Use the method of Section 39 to prove that if both $|z|<1$ and $|1-z|<1$, and if $a,b,c$ are suitably restricted,
$$\begin{array}{ll}
F \left[ \begin{array}{rlr}
a,b; & & \\
& & z \\
c; & &
\end{array} \right] &= \dfrac{\Gamma(c)\Gamma(c-a-b)}{\Gamma(c-a)\Gamma(c-b)} F \left[ \begin{array}{rlr}
a,b; & & \\
& & 1-z \\
1+b+1-c; & & 
\end{array} \right] \\
&+ \dfrac{\Gamma(c)\Gamma(a+b-c)(1-z)^{c-a-b}}{\Gamma(a)\Gamma(b)} F \left[ \begin{array}{rlr}
c-a,c-b; & & \\
& & 1-z \\
c-a-b+1; & &
\end{array} \right].
\end{array}$$
\end{problem}
\begin{solution}(Solution by Leon Hall)
We denote the hypergeometric differential equation:
$$z(1-z)w''(z)+[c-(a+b+1)z]w'(z)-abw(z)=0$$
by HGDE. If we make the change of variable $z = 1-y,$ then HGDE becomes
$$y(1-y)w''(y)+[c*-(a+b+1)y]w'(y)-abw(y)=0$$
where $c*=a+b+1-c$. Thus, two linearly independent solutions are
$$F(a,b;c*;y)$$
and
$$y^{1-c*}F(a+1-c*,b+1-c*;2-c*;y).$$
These solutions as function of $z$ are valid in $|1-z|<1$ and are
$$F(a,b;a+b+1-c;1-z)$$
and
$$(1-z)^{c-a-b}F(c-a,c-b;c-a-b+1;1-z).$$
Thus, in the region $D$ where both $|z|<1$ and $|1-z|<1$, 
$$F(a,b;c;z)=AF(a,b;a+b+1-c;1-z) + B(1-z)^{c-a-b}F(c-a,c-b;c-a-b+1;1-z)$$
for some constants $A$ and $B$.

Assume $\mathrm{Re}(c-a-b)>0$ and $c \neq 0$ or a negative integer and let $z \rightarrow 1$ inside the region $D$ to get
$$F(a,b;c;1)=A \cdot 1 + B \cdot 0.$$
Thus, by Theorem~18, page 49, we get
$$A = \dfrac{\Gamma(c) \Gamma(c-a-b)}{\Gamma(c-a) \Gamma(c-b)}.$$
Now, let $z \rightarrow 0$ inside the region $D$ and assume $\mathrm{Re}(1-c)>0$ and neither $a+b+1-c$ nor $c-a-b+1$ is zero or a negative integer. Then
$$1 = AF(a,b;a+b+1-c;1) + BF(c-a,c-b;c-a-b+1;1)$$
and we get
$$B = \dfrac{1 - AF(a,b;a+b+1-c;1)}{F(c-a,c-b;c-a-b+1;1)}.$$
Again using Theorem~18, this becomes
$$B = \dfrac{1 - \frac{\Gamma(c) \Gamma(c-a-b) \Gamma(a+b+1-c) \Gamma(1-c)}{\Gamma(c-a) \Gamma(c-b) \Gamma(b+1-c) \Gamma(a+1-c)}}{\frac{\Gamma(c-a-b+1) \Gamma(1-c)}{\Gamma(1-b) \Gamma(1-a)}}.$$
By Exercise~15, page 32 the numerator is equal to
$$\dfrac{\Gamma(2-c) \Gamma(c-1) \Gamma(c-a-b) \Gamma(a+b+1-c)}{\Gamma(a) \Gamma(1-a) \Gamma(b) \Gamma(1-b)}.$$
Hence,
$$\begin{array}{ll}
B &= \dfrac{\Gamma(2-c) \Gamma(c-1) \Gamma(c-a-b) \Gamma(a+b+1-c)}{\Gamma(a)\Gamma(b) \Gamma(c-a-b+1) \Gamma(1-c)} \\
&= \dfrac{(1-c) \Gamma(1-c) \frac{\Gamma(c)}{c-1} (a+b-c) \Gamma(a+b-c)}{\Gamma(a) \Gamma(b) (c-a-b) \Gamma(c-a-b) \Gamma(1-c)} \\
&= \dfrac{\Gamma(c) \Gamma(a+b-c)}{\Gamma(a) \Gamma(b)}.
\end{array}$$
This yields the desired formula for $F(a,b;c;z)$ in terms of the given hypergeometric functions of $1-z$.
\end{solution}
%%%%
%%
%%
%%%%
\begin{problem}\label{problem16chapter4}
In a common notation for the Laplace transform
$$L \{F(t)\} = \displaystyle\int_0^{\infty} e^{-st} F(t) dt = f(s); L^{-1}\{f(s)\} = F(t).$$
Show that 
$$L^{-1} \left\{ \dfrac{1}{s} F \left[ \begin{array}{rlr}
a,b; & & \\
& & z \\
s+1; & & 
\end{array} \right] \right\} = F \left[ \begin{array}{rlr}
a, b; & & \\
& & z(1-e^{-t}) \\
1; & & 
\end{array} \right].$$
\end{problem}
\begin{solution}
Let $A = \mathscr{L}^{-1} \left\{ \dfrac{1}{s} F \left[ \begin{array}{rlr}
a,b; & & \\
& & z \\
1+s;
\end{array} \right] \right\}.$ We wish to evaluate $A$. Now
$$A = \displaystyle\sum_{n=0}^{\infty} \dfrac{(a)_n (b)_n z^n}{n!} \mathscr{L}^{-1} \left\{ \dfrac{1}{s(1+s)_n} \right\}.$$
But
$$\begin{array}{ll}
\dfrac{1}{s(s+1)_n} &= \dfrac{\Gamma(1+s)}{s \Gamma(1+s+n)} \\
&= \dfrac{1}{s n!} \dfrac{\Gamma(1+s) \Gamma(1+n)}{\Gamma(1+s+n) \Gamma(1)} \\
&= \dfrac{1}{n!s} F \left[ \begin{array}{rlr}
-n, s; & & \\
& & 1 \\
1+s; & & 
\end{array} \right] \\
&= \dfrac{1}{n!s} \displaystyle\sum_{k=0}^n \dfrac{(-n)_k (s)_k}{k! (1+s)_k}.
\end{array}$$
Hence
$$\dfrac{1}{s(s+1)_n} = \dfrac{1}{n!} \displaystyle\sum_{k=0}^n \dfrac{(-n)_k}{k!(s+k)}.$$
Then
$$\mathscr{L}^{-1} \left\{ \dfrac{1}{s(s+1)_n} \right\} = \dfrac{1}{n!} \displaystyle\sum_{k=0}^n \dfrac{(n)_k e^{-kt}}{k!} = \dfrac{1}{n!} (1 - e^{-t})^n.$$
Therefore
$$A = \displaystyle\sum_{n=0}^{\infty} \dfrac{(a)_n (b)_n z^n (1-e^{-t})^n}{n! n!} = F \left[ \begin{array}{rlr}
a,b; & & \\
& & z(1-e^{-t}) \\
1; & & 
\end{array} \right].$$
There are many other ways of doing Exercise 16. Probably the easiest, but most undesirable, is to work from right to left in the result to be proved. It is hard to see any chance for \underline{discovering} the relation that way.
\end{solution}
%%%%
%%
%%
%%%%
\begin{problem}\label{problem17chapter4}
With that notation of Exercise 16 show that
$$L \{t^n \sin at\} = \dfrac{a \Gamma(n+2)}{s^{n+2}} F \left[ \begin{array}{rlr}
1 + \dfrac{n}{2}, \dfrac{3+n}{2}; & & \\
& & -\dfrac{a^2}{s^2} \\
\dfrac{3}{2}; 
\end{array} \right].$$
\end{problem}
\begin{solution}
We wish to obtain the Laplace Transform of $t^n \sin at$. Now
$$t^n \sin at = \displaystyle\sum_{k=0}^{\infty} \dfrac{(-1)^k a^{2k+1} t^{n+2k+1}}{(2k+1)!}.$$
and
$$\mathscr{L} \left\{ t^m \right\} = \dfrac{\Gamma(m+1)}{s^{m+1}}.$$
Hence
$$\begin{array}{ll}
\mathscr{L} \left\{ t^n \sin at \right\} &= \displaystyle\sum_{k=0}^{\infty} \dfrac{(-1)^k a^{2k+1} \Gamma(n+2k+2)}{(2k+1)! s^{n+2k+2}} \\
&= \dfrac{a \Gamma(n+2)}{s^{n+2}} \displaystyle\sum_{k=0}^{\infty} \dfrac{(-1)^k (n+2)_{2k} a^{2k}}{(2)_{2k} s^{2k}} \\
&= \dfrac{a \Gamma(n+2)}{s^{n+2}} \displaystyle\sum_{k=0}^{\infty} \dfrac{(-1)^k \left( \dfrac{n+2}{2} \right)_k \left( \dfrac{n+3}{2} \right)_k a^{2k}}{k! \left( \dfrac{3}{2} \right)_k s^{2k}} \\
&= \dfrac{a \Gamma(n+2)}{s^{n+2}} F \left[ \begin{array}{rlr}
\dfrac{n+2}{2}, \dfrac{n+3}{2}; & & \\
& & \dfrac{-a^2}{s^2} \\
\dfrac{3}{2}; & & 
\end{array} \right].
\end{array}$$
\end{solution}
%%%%
%%
%%
%%%%
\begin{problem}\label{problem18chapter4}
Obtain the results
$$\log(1+x) = xF(1,1;2;-x),$$
$$\arcsin x = xF \left(\dfrac{1}{2}, \dfrac{1}{2}; \dfrac{3}{2}; x^2 \right),$$
$$\arctan x = x F \left( \dfrac{1}{2}, 1; \dfrac{3}{2}; -x^2 \right).$$
\end{problem}
\begin{solution}
Using $\dfrac{n!}{(n+1)!} = \dfrac{(1)_n}{(2)_n}$ we know that 
$$\begin{array}{ll}
\log(1+x) &= \displaystyle\sum_{n=0}^{\infty} \dfrac{(-1)^n x^{n+1}}{n+1} \\
&= x \displaystyle\sum_{n=0}^{\infty} \dfrac{(-1)^n (1)_n (1)_n x^n}{(2)_n n!} \\
&= x F(1,1;2;-x).
\end{array}$$
Next, start with
$$(1-y^2)^{-\frac{1}{2}} = \displaystyle\sum_{n=0}^{\infty} \dfrac{\left( \frac{1}{2} \right)_n y^{2n}}{n!}$$
using $\dfrac{1}{2n+1} = \dfrac{\frac{1}{2}}{n+\frac{1}{2}} = \dfrac{\left(\frac{1}{2} \right)_n}{\left( \frac{3}{2} \right)_n}$ to get
$$\displaystyle\int_0^x (1-y^2)^{-\frac{1}{2}} \mathrm{d}y = \displaystyle\sum_{n=0}^{\infty} \dfrac{\left( \frac{1}{2} \right)_n x^{2n+1}}{n! (2n+1)}.$$
Thus we arrive at
$$\begin{array}{ll}
\arcsin x &= \displaystyle\sum_{n=0}^{\infty} \dfrac{\left( \frac{1}{2} \right)_n \left( \frac{1}{2} \right)_n x^{2n+1}}{\left( \frac{3}{2} \right)_n n!} \\
&= x F \left( \dfrac{1}{2}, \dfrac{1}{2}; \dfrac{3}{2}; x^2 \right).
\end{array}$$
Finally form
$$(1+y^2)^{-1} = \displaystyle\sum_{n=0}^{\infty} (-1)^n y^{2n}$$
to obtain
$$\displaystyle\int_0^x (1+y^2)^{-1} dy = \displaystyle\sum_{n=0}^{\infty} \dfrac{(-1)^n x^{2n+1}}{2n+1} = \displaystyle\sum_{n=0}^{\infty} \dfrac{(-1)^n \left( \frac{1}{2} \right)_n x^{2n+1}}{\left( \frac{3}{2} \right)_n} \cdot \dfrac{(1)_n}{n!}$$
or
$$\arctan x = x F \left( \dfrac{1}{2}, 1 ; \dfrac{3}{2}; -x^2 \right).$$
\end{solution}
%%%%
%%
%%
%%%%
\begin{problem}\label{problem19chapter4}
The complete elliptic integral of the first kind is
$$K = \displaystyle\int_0^{\frac{\pi}{2}} \dfrac{\mathrm{d} \phi}{\sqrt{1-k^2 \sin^2\phi}}.$$
Show that $K = \dfrac{\pi}{2} F \left( \dfrac{1}{2}, \dfrac{1}{2}; 1 ; k^2 \right).$
\end{problem}
\begin{solution}
From $K = \displaystyle\int_0^{\frac{\pi}{2}} \dfrac{\mathrm{d} \phi}{\sqrt{1 - k^2 \sin^2 \phi}}$ we obtain
$$K = \displaystyle\int_0^{\frac{\pi}{2}} \displaystyle\sum_{n=0}^{\infty} \dfrac{ \left(\frac{1}{2} \right)_n k^{2n} \sin^{2n} \phi \mathrm{d} \phi}{n!}.$$
But
$$\displaystyle\int_0^{\frac{\pi}{2}} \sin^{2n}\phi \mathrm{d}\phi = \dfrac{1}{2} B \left(n+\dfrac{1}{2}, \dfrac{1}{2} \right) = \dfrac{\Gamma \left( n + \frac{1}{2} \right) \Gamma \left( \frac{1}{2} \right)}{2 \Gamma(n+1)} = \dfrac{\Gamma^2 \left(\frac{1}{2} \right) \left( \frac{1}{2} \right)_n}{2n!} = \dfrac{\pi}{2} \dfrac{\left( \frac{1}{2} \right)_n}{n!}.$$
Hence
$$K = \dfrac{\pi}{2} \displaystyle\sum_{n=0}^{\infty} \dfrac{ \left( \frac{1}{2} \right)_n \left( \frac{1}{2} \right)_n k^{2n}}{n! n!} = \dfrac{\pi}{2} F \left(\dfrac{1}{2},\dfrac{1}{2};1;k^2 \right).$$
\end{solution}
%%%%
%%
%%
%%%%
\begin{problem} \label{problem20chapter4}
The complete elliptic integral of the second kind is 
$$E = \displaystyle\int_0^{\frac{\pi}{2}} \sqrt{1 - k^2 \sin^2 \theta} \mathrm{d} \theta.$$
Show that $E = \dfrac{\pi}{2} F \left(\dfrac{1}{2}, -\dfrac{1}{2};1;k^2 \right).$
\end{problem}
\begin{solution}
From $E = \displaystyle\int_0^{\frac{\pi}{2}} \sqrt{1 - k^2 \sin^2 \theta} \mathrm{d} \theta$,
we get
$$\begin{array}{ll}
E &= \displaystyle\int_0^{\frac{\pi}{2}} \displaystyle\sum_{n=0}^{\infty} \dfrac{ \left( -\frac{1}{2} \right)_n k^{2n} \sin^{2n} \phi \mathrm{d} \phi}{n!} \\
&= \dfrac{\pi}{2} \displaystyle\sum_{n=0}^{\infty} \dfrac{ \left( -\frac{1}{2} \right)_n \left( \frac{1}{2} \right)_n k^{2n}}{n! n!}.
\end{array}$$
Hence
$$E = \dfrac{\pi}{2} F \left( - \dfrac{1}{2}, \dfrac{1}{2}; 1 ; k^2 \right).$$
\end{solution}
%%%%
%%
%%
%%%%
\begin{problem} \label{problem21chapter4}
From the contiguous function relations 1-5 obtain the relations 6-10.
\begin{enumerate}
\item $(a-b)F = aF(a+) - bF(b+),$
\item $(a-c+1)F = aF(a+) - (c-1)F(c-),$
\item $[a+(b-c)z]F = a(1-z)F(a+) - c^{-1}(c-a)(c-b)zF(c+),$
\item $(1-z)F = F(a-) - c^{-1}(c-b)zF(c+),$
\item $(1-z)F = F(b-) - c^{-1}(c-a)zF(c+),$
\item $[2a-c+(b-a)z]F = a(1-z)F(a+) - (c-a)F(a-),$
\item $(a+b-c)F = a(1-z)F(a+) - (c-b)F(b-),$
\item $(c-a-b)F = (c-a)F(a-) - b(1-z)F(b+),$
\item $(b-a)(1-z)F = (c-a)F(a-)-b(1-z)F(b+),$
\item $[1-z+(c-b-1)z]F = (c-a)F(a-) - (c-1)(1-z)F(c-),$
\item $[2b-c+(a-b)z]F = b(1-z)F(b+) - (c-b)F(b-),$
\item $[b+(a-c)z]F = b(1-z)F(b+) - c^{-1}(c-a)(c-b)zF(c+),$
\item $(b-c+1)F = bF(b+) - (c-1)F(c-),$
\item $[1-b+(c-a-1)z]F = (c-b)F(b-) - (c-1)(1-z)F(c-),$
\item $[c-1+(a+b+1-2c)z]F = (c-1)(1-z)F(c-) - c^{-1}(c-a)(c-b)zF(c+).$
\end{enumerate}
\end{problem}
\begin{solution}
From $(3)$ and $(4)$ we get
$$[a + (b-c)z - (c-a)(1-z)]F = a(1-z)F(a+)-(c-a)F(a-),$$
or
$$(6) \hspace{30pt} [2a-c + (b-a)z]F = a(1-z)F(a+) - (c-a) F(a-).$$
From $(3)$ and $(5)$ we get
$$[a + (b-c)z - (c-b)(1-z)]F = a(1-z)F(a+) - (c-b)F(b-),$$
or
$$(7) \hspace{30pt} [a+b-c]F = a(1-z)F(a+) - (c-b)F(b-).$$
From $(1)$ and $(6)$ we get
$$[(a-b)(1-z) - 2a + c - (b-a)z]F = (c-a)F(a-) - b(1-z)F(b+),$$
or
$$(8) \hspace{30pt} [c-a-b]F = (c-a)F(a-) - b(1-z)F(b+).$$
From $(6)$ and $(7)$ we get
$$(9) \hspace{30pt} (b-a)(1-z)F = (c-a)F(a-)-(c-b)F(b-).$$
Use $(2)$ and $(6)$ to obtain
$$[(a-c+1)(1-z)-2a+c - (b-a)z]F = (c-a) F(a-) - (c-1)(1-z)F(c-),$$
or
$$(10) \hspace{30pt} [1-a+(c-b-1)z] F = (c-a)F(a-) - (c-1)(1-z)F(c-).$$
From $(1)$ and $(7)$ we get
$$[a+b-c-(a-b)(1-z)]F = b(1-z)F(b+) - (c-b)F(b-),$$
or
$$(11) \hspace{30pt} [2b-c+(1-b)z]F = b(1-z)F(b+) - (c-b)F(b-),$$
which checks with $(6)$. Easier method: in $(6)$ interchange $a$ and $b$.
From $(1)$ and $(3)$ we get
$$[a + (b-c)z - (a-b)(1-z)]F = b(1-z)F(b+) - c^{-1}(c-a)(c-b)zF(c+),$$
or
$$(12) \hspace{30pt} [b+(a-c)z]F = b(1-z)F(b+) - c^{-1}(c-a)(c-b)zF(c+),$$
more easily found by changing $b$ to $a$ and $a$ to $b$ in $(3)$.
In $(2)$ interchange $a$ and $b$ to get
$$(13) \hspace{30pt} (b-c+1)F = bF(b+) - (c-1)F(c-).$$
In $(10)$ interchange $a$ and $b$ to get
$$(14) \hspace{30pt} [1-b+(c-a-1)z]F = (c-b)F(b-) - (c-1)(1-z)F(c-).$$
From $(2)$ and $(3)$ we get
$$[a+(b-c)z-(a-c+1)(1-z)]F = (c-1)(1-z)F(c-)-c^{-1}(c-a)(c-b)zF(c+),$$
or
$$(15) \hspace{30pt} [c-1+(a+b-2c+1)z]F = (c-1)(1-z)F(c-) - c^{-1}(c-a)(c-b)zF(c+).$$
\end{solution}
%%%%
%%
%%
%%%%
\begin{problem}\label{problem22chapter4}
The notation used in Exercise~\ref{problem21chapter4} and in Section~33 is often extended as in the examples
$$F(a-,b+) = F(a-1,b+1;c;z),$$
$$F(b+,c+) = F(a,b+1;c+1;z).$$
Use the relations (4) and (5) of Exercise~\ref{problem21chapter4} to obtain
$$F(a-) - F(b-) + c^{-1}(b-a)zF(c+) = 0$$
and from it, by changing $b$ to $(b+1)$ to obtain the relation
$$(c-1-b)F = (c-a)F(a-,b+) + (a-1-b)(1-z)F(b+),$$
or
$$(c-1-b)F(a,b;c;z) = (c-a)F(a-1,b+1;c;z) + (a-1-b)(1-z)F(a,b+1;c;z),$$
another relation we wish to use in Chapter~16.
\end{problem}
\begin{solution}
From Exercise~\ref{problem21chapter4} equation $(4)$ and $(5)$ we get
$$(1) (1-z)F=F(a-)-c^{-1}(c-b)zF(c+),$$
$$(2) (1-z)F = F(b-) - c^{-1}(c-a)zF(c+).$$
From the above we get
$$F(a-) - F(b-) + c^{-1}(b-a)zF(c+) = 0.$$
Now replace $b$ by $b+1$ to write
$$F(a-,b+) - F + c^{-1}(b+1-a)zF(b+,c+) = 0,$$
or
$$F(a,b;c;z) = F(a-1,b+1;c;z) + c^{-1}(b+1-a)zF(a,b+1;c+1;z).$$
\end{solution}
\begin{problem}\label{problem23chapter4}
In equation $(9)$ of Exercise~\ref{problem21chapter4} shift $b$ to $b+1$ to obtain the relation
$$(c-1-b)F = (c-a)F(c-,b+)+(a-1-b)(1-z)F(b+),$$
or
$$(c-1-b)F(a,b;c;z) = (c-a)F(a-1,b+1;c;z) + (a-1-b)(1-z)F(a,b+1;c;z),$$
another relation we wish to use in Chapter 16.
\end{problem}
\begin{solution}
Equation $(9)$ of Exercise~\ref{problem21chapter4} is
$$(b-a)(1-z)F = (c-a)F(a-) - (c-b)F(b-)$$
from which we may write
$$(b+1-a)(1-z)F(b+) = (c-a)F(a-,b+) - (c-b-1)F,$$
or
$$(c-b-1)F(a,b;c;z) = (c-a)F(a-,b+1;c;z) + (a-1-b)(1-z)F(a,b+1;c;z).$$
\end{solution}