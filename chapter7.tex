%%%%
%%
%%
%%%%
%%%% CHAPTER 7
%%%% CHAPTER 7
%%%%
%%
%%
%%%%
\section{Chapter 7 Solutions}
\begin{center}\hyperref[toc]{\^{}\^{}}\end{center}
\begin{center}\begin{tabular}{lllllllllllllllllllllllll}
\hyperref[problem1chapter7]{P1} & \hyperref[problem2chapter7]{P2} & \hyperref[problem3chapter7]{P3} & \hyperref[problem4chapter7]{P4} & \hyperref[problem5chapter7]{P5} & \hyperref[problem6chapter7]{P6} 
\end{tabular}\end{center}
\setcounter{problem}{0}
\setcounter{solution}{0}
\begin{problem}\label{problem1chapter7}
The function 

$$\mathrm{erf}(x) = \dfrac{2}{\sqrt{\pi}} \displaystyle\int_0^x \exp(-t^2) \mathrm{d}t$$
was defined on page 36. Show that
$$\mathrm{erf}(x) = \dfrac{2x}{\sqrt{\pi}} {}_1F_1 \left( \dfrac{1}{2}; \dfrac{3}{2}; -x^2 \right).$$
\end{problem}
\begin{solution}
Let 
$$\mathrm{erf}(x) = \dfrac{2}{\sqrt{\pi}} \displaystyle\int_0^x \exp(-t^2) \mathrm{d}t.$$
Then,
$$\begin{array}{ll}
\mathrm{erf}(x) &= \dfrac{2}{\sqrt{\pi}} \displaystyle\sum_{n=0}^{\infty} \dfrac{(-1)^n \displaystyle\int_0^x t^{2n} \mathrm{d}t}{n!} \\
&= \dfrac{2}{\sqrt{\pi}} \displaystyle\sum_{n=0}^{\infty} \dfrac{(-1)^n x^{2n+1}}{n! (2n+1)} \\
&= \dfrac{2x}{\sqrt{\pi}} \displaystyle\sum_{n=0}^{\infty} \dfrac{(-1)^n \left( \frac{1}{2} \right)_n x^{2n}}{n! \left( \frac{3}{2} \right)_n} \\
&= \dfrac{2x}{\sqrt{\pi}} {}_1F_1 \left( \dfrac{1}{2}; \dfrac{3}{2}; -x^2 \right).
\end{array}$$
\end{solution}
%%%%
%%
%%
%%%%
\begin{problem}\label{problem2chapter7}
The incomplete gamma function may be defined by the following equation for $\mathrm{Re}(\alpha)>0$:
$$\gamma(\alpha,x) = \displaystyle\int_0^x e^{-t} t^{\alpha-1} \mathrm{d}t.$$
Show that
$$\gamma(\alpha, x) = \alpha^{-1} x^{\alpha} {}_1F_1(\alpha; \alpha+1; -x).$$
\end{problem}
\begin{solution}
For $\mathrm{Re}(\alpha)>0$, let 
$$\gamma(\alpha, x) = \displaystyle\int_0^x e^{-t} t^{\alpha -1} \mathrm{d}t.$$
Then,
$$\begin{array}{ll}
\gamma(\alpha, x) &= \displaystyle\int_0^x \displaystyle\sum_{n=0}^{\infty} \dfrac{(-1)^n t^{n + \alpha - 1}}{n!} \mathrm{d}t \\
&= \displaystyle\sum_{n=0}^{\infty} \dfrac{(-1)^n x^{n + \alpha}}{n! (\alpha + n)}.
\end{array}$$
Now, $(\alpha + n) = \dfrac{\alpha (\alpha+1)_n}{(\alpha)_n}.$ Hence
$$\gamma(\alpha,x) = \alpha^{-1}x^{\alpha} \displaystyle\sum_{n=0}^{\infty} \dfrac{(-1)^n (\alpha)_n x^n}{n! (\alpha + 1)_n} = \alpha^{-1}x^{\alpha} {}_1F_1(\alpha; \alpha+1; -x).$$
\end{solution}
%%%%
%%
%%
%%%%
\begin{problem}\label{problem3chapter7}
Prove that
$$(b)_k \dfrac{\mathrm{d}^k}{\mathrm{d}z^k} \left[ e^{-z} {}_1F_1(a;b;z) \right] = (-1)^k (b-a)_ke^{-z} {}_1F_1(a;b+k;z).$$
You may find it helpful to use Kummer's formula, Theorem 42.
\end{problem}
\begin{solution}
Let $\mathscr{D} = \dfrac{\mathrm{d}}{\mathrm{d}x}$. Consider $\mathscr{D}[e^{-z} {}_1F_1(a;b;z)].$ We know that
$${}_1F_1(a;b;z) = e^z {}_1F_1(b-a;b;-z).$$
Hence
$$\begin{array}{ll}
\mathscr{D}^k[e^{-z}{}_1F_1(a;b;z)] &= \mathscr{D}^k {}_1F_1(b-a;b;-z) \\
&= \dfrac{(b-a)_k (-1)^k}{(b)_k} {}_1F_!(b-1+k;b+k;-z) \\
&= \dfrac{(-1)^k (b-a)_k}{(b)_k} e^{-z} {}_1F_!(a ; b+k;z).
\end{array}$$

We used $\mathscr{D}{}_1F_1(a;b;z) = \dfrac{a}{b} {}_1F_1(a+1;b+1;z)$ $k$ times.
\end{solution}
%%%%
%%
%%
%%%%
\begin{problem}\label{problem4chapter7}
Show that

$${}_1F_1(a;b;z) = \dfrac{1}{\Gamma(a)} \displaystyle\int_0^{\infty} e^{-t}t^{\alpha-1}{}_0F_1(-;b;zt) \mathrm{d}t.$$
\end{problem}
\begin{solution}
We know that for $\mathrm{Re}(x)>0$,
$$\Gamma(x) = \displaystyle\int_0^{\infty} e^{-t} t^{x-1} \mathrm{d}t.$$
Then for $\mathrm{Re}(a)>0$,
$$\begin{array}{ll}
{}_1F_1(a;b;z) &= \displaystyle\sum_{n=0}^{\infty} \dfrac{(a)_n z^n}{n! (b)_n} \\
&= \dfrac{1}{\Gamma(a)} \displaystyle\sum_{n=0}^{\infty} \dfrac{\Gamma(a+n)z^n}{n!(b)_n} \\
&= \dfrac{1}{\Gamma(a)} \displaystyle\int_0^{\infty} e^{-t} \displaystyle\sum_{n=0}^{\infty} \dfrac{t^{a+n-1}z^n}{n! (b)_n} \mathrm{d}t \\
&= \dfrac{1}{\Gamma(a)} \displaystyle\int_0^{\infty} e^{-t} t^{a-1} {}_0F_1(-;b;tz)\mathrm{d}t.
\end{array}$$
\end{solution}
%%%%
%%
%%
%%%%
\begin{problem}\label{problem5chapter7}
Show, with the aid of the result in Exercise~\ref{problem4chapter7}, that
$$\displaystyle\int_0^{\infty} \exp(-t^2) t^{2a-n-1}J_n(zt) \mathrm{d}t = \dfrac{\Gamma(a)z^n}{2^{n+1}\Gamma(n+1)} {}_1F_1 \left( a; n+1; -\dfrac{z^2}{4} \right).$$
\end{problem}
\begin{solution}
We obtain
$$\begin{array}{ll}
A &= \displaystyle\int_0^{\infty} \exp(-t^2) t^{2a-n-1} J_n(zt) \mathrm{d}t \\
&= \displaystyle\int_0^{\infty} \dfrac{e^{-t^2}t^{2a-n-1} z^n t^n}{2^n \Gamma(1+n)} {}_0F_1 \left( -;1+n;-\dfrac{z^2t^2}{4} \right) \mathrm{d}t.
\end{array}$$
Put $t^2=\beta$. Then
$$\begin{array}{ll}
A &= \dfrac{z^n}{2^n \Gamma(1+n)} \dfrac{1}{2} \displaystyle\int_0^{\infty} e^{-\beta} \beta^{a-1} {}_0F_1 \left( - ; 1+n; - \dfrac{z^2 \beta}{4} \right) \mathrm{d} \beta \\
&= \dfrac{z^n}{2^{n+1}\Gamma(1+n)} \dfrac{\Gamma(a)}{1} {}_1F_! \left( a; 1+n; -\dfrac{z^2}{4} \right),
\end{array}$$

as desired.
\end{solution}
%%%%
%%
%%
%%%%
\begin{problem}\label{problem6chapter7}
If $k$ and $n$ are non-negative integers, show that

$$\begin{array}{ll}
F \left[ \begin{array}{rlr}
-k, \alpha+n; & & \\
& & 1 \\
\alpha; & & 
\end{array} \right] = \left\{
\begin{array}{ll}
0 &; k > n \\
\dfrac{(-n)_k}{(\alpha)_k} &; 0 \leq k \leq n.
\end{array}
\right.
\end{array}$$
\end{problem}

\begin{solution}
Let $V(k,n) = F(-k,\alpha+n;\alpha;1).$ Then

$$V(k,n) = \displaystyle\sum_{s=0}^k \dfrac{(-k)_s (\alpha+n)_s}{s! (\alpha)_s} = \displaystyle\sum_{s=0}^k \dfrac{(-1)^s k! (\alpha)_{n+s}}{s! (k-s)! (\alpha)_s (\alpha)_n}.$$

Hence

$$\begin{array}{ll}
\psi &= \displaystyle\sum_{k=0}^{\infty} \dfrac{V(k,n)t^k}{k!} \\
&= \displaystyle\sum_{k=0}^{\infty} \displaystyle\sum_{s=0}^k \dfrac{(-1)^s (\alpha)_{n+s} t^k}{s! (k-s)! (\alpha)_s (\alpha)_n} \\
&= \displaystyle\sum_{k,j=0}^{\infty} \dfrac{(-1)^s (\alpha)_{n+s} t^{k+s}}{s! k! (\alpha)_s (\alpha)_n} \\
&= e^t \displaystyle\sum_{s=0}^{\infty} \dfrac{(-1)^s (\alpha+n)_s t^s}{s! (\alpha)_s} \\
&= e^t {}_1F_!(\alpha+n;\alpha;-t) \\
&= e^t e^{-t} {}_1F_!(-n;\alpha;t) \\
&= \displaystyle\sum_{k=0}^n \dfrac{(-n)_k t^k}{k! (\alpha)_k}.
\end{array}$$

Thus

$$V(k,n) = \left\{ \begin{array}{ll}
\dfrac{(-n)_k}{(\alpha)_k} &; 0 \leq k \leq n \\
0 &; k > n.
\end{array} \right.$$

note: this method requires material in chapter 7 (on line 5)

Can do by Chapter 4 if first show that $e^{-t} {}_1F_1(-n;\alpha;t) =  {}_1F_1(\alpha+n;\alpha;-t).$
\end{solution}