%%%%
%%
%%
%%%%
%%%% CHAPTER 11
%%%% CHAPTER 11
%%%%
%%
%%
%%%%
\section{Chapter 11 Solutions}
\begin{center}\hyperref[toc]{\^{}\^{}}\end{center}
\begin{center}\begin{tabular}{lllllllllllllllllllllllll}
\hyperref[problem1chapter11]{P1} & \hyperref[problem2chapter11]{P2} & \hyperref[problem3chapter11]{P3} & \hyperref[problem4chapter11]{P4} & \hyperref[problem5chapter11]{P5} & \hyperref[problem6chapter11]{P6} & \hyperref[problem7chapter11]{P7}
\end{tabular}\end{center}
\setcounter{problem}{0}
\setcounter{solution}{0}
\begin{problem}\label{problem1chapter11}
Use the fact that

$$\exp(2xt-t^2) = \exp(2xt-x^2t^2) \exp[t^2(x^2+1)]$$

to obtain the expansion

$$H_n(x) = \displaystyle\sum_{k=0}^{[\frac{n}{2}]} \dfrac{n! H_{n-2k}(1) x^{n-2k}(x^2+1)^k}{k!(n-2k)!}.$$
\end{problem}
\begin{solution}
From
$$\exp(2xt-t^2) = \exp(2xt-x^2t^2)\exp[t^2(x^2+1)]$$
we obtain
\begin{eqnarray*}
\lefteqn{\left( \displaystyle\sum_{n=0}^{\infty} \dfrac{H_n(1) (xt)^n}{n!} \right) \left( \displaystyle\sum_{n=0}^{\infty} \dfrac{(x^2+1)^n t^{2n}}{n!} \right)} \\
&&= \displaystyle\sum_{n=0}^{\infty} \displaystyle\sum_{k=0}^{[\frac{n}{2}]} \dfrac{H_{n-2k}(1) (x)^{n-2k}(x^2+1)^k t^n}{k! (n-2k)!}.
\end{eqnarray*}
It follows that
$$H_n(x) = \displaystyle\sum_{k=0}^{[\frac{n}{2}]} \dfrac{n! H_{n-2k}(1) x^{n-2k} (x^2+1)^k}{k! (n-2k)!}.$$
\end{solution}
%%%%
%%
%%
%%%%
\begin{problem}\label{problem2chapter11}
Use the expansion of $x^n$ in a series of Hermite polynomials to show that

$$\displaystyle\int_{-\infty}^{\infty} \exp(-x^2)x^n H_{n-2k}(x) \mathrm{d}x = 2^{-2k}n! \dfrac{\sqrt{\pi}}{k!}.$$

Note in particular the special case $k=0$.
\end{problem}
\begin{solution}
We know that 
$$x^n = \displaystyle\sum_{s=0}^{[\frac{n}{2}]} \dfrac{n! H_{n-2s}(x)}{2^n s! (n-2s)!}.$$
Then
\begin{eqnarray*}
\lefteqn{\displaystyle\int_{-\infty}^{\infty} \exp(-x^2) x^n H_{n-2k}(x) \mathrm{d}x} \\
&&= \displaystyle\sum_{s=0}^{[\frac{n}{2}]} \dfrac{n!}{^n s! (n-2k)!} \displaystyle\int_{-\infty}^{\infty} e^{-x^2} H_{n-2k}(x) H_{n-2s}(x) \mathrm{d}x.
\end{eqnarray*}

The integrals involved on the right are zero except for $s=k$. Hence

$$\begin{array}{ll}
\displaystyle\int_{-\infty}^{\infty} \exp(-x^2)x^n H_{n-2k}(x)\mathrm{d}x &= \dfrac{n!}{2^n k! (n-2k)!} \displaystyle\int_{\infty}^{\infty} e^{-x^2} H_{n-2k}^2(x) \mathrm{d}x \\
&= \dfrac{n!2^{n-2k}(n-2k)! \sqrt{\pi}}{2^n k! (n-2k)!} \\
&= \dfrac{n! \sqrt{\pi}}{2^{2k}k!}.
\end{array}$$

For $k=0$ the right member becomes $n! \sqrt{\pi}$.
\end{solution}
%%%%
%%
%%
%%%%
\begin{problem}\label{problem3chapter11}
Use the integral evaluation in equation (4), page 192, to obtain the result
$$\displaystyle\int_0^{\infty} \exp(-x^2) H_{2k}(x) H_{2n+1}(x) \mathrm{d}x = \dfrac{(-1)^{k+s}2^{2k+2s}(\frac{1}{2})_k (\frac{3}{2})_s}{2s+1-2k}.$$
\end{problem}
\begin{solution}
We know that for $m \neq n$,
$$\displaystyle\int_{-\infty}^{\infty} e^{-x^2} H_n(x) H_m(x) \mathrm{d}x =0.$$
If $m$ and $n$ are both odd or both even, the integrand above is an even function of $x$ and we obtain
$$2 \displaystyle\int_0^{\infty} e^{-x^2} H_n(x) H_m(x)\mathrm{d}x =0$$
for $m \equiv n \hspace{4pt}\mathrm{mod} \hspace{3pt} 2$.
Now consider
$$\displaystyle\int_0^{\infty} e^{-x^2} H_{2n}(x) H_{2s+1}(x)\mathrm{d}x.$$
By equation (4), page 331, we get
\begin{eqnarray*}
\lefteqn{2(2n-2s-1) \displaystyle\int_0^{\infty} e^{-x^2} H_{2n}(x) H_{2s+1}(x) \mathrm{d}x} \\
& &= \left[ e^{-x^2} \left\{ H_{2n}(x) H_{2s+1}'(x) - 2_{2n}'(x) H_{2s+1}(x) \right\} \right]_0^{\infty} \\
&&= -H_{2n}(0)H_{2s+1}'(0) + H_{2n}'(0)H_{2s+1}(0) \\
&&= -H_{2n}(0)H_{2s+1}'(0).
\end{eqnarray*}
Using the values of $H_{2n}(0)$ and $H_{2s+1}'(0)$ from page 323, we get
$$\begin{array}{ll}
\displaystyle\int_0^{\infty} e^{-x^2}H_{2n}(x)H_{2s+1}(x) \mathrm{d}x &= \dfrac{-(-1)^n2^{2n}(\frac{1}{2})_n (-1)^s 2^{2s+1} (\frac{3}{2})_s}{2(2n-2s-1)} \\
&= \dfrac{(-1)^{n+s} 2^{2n+2s} (\frac{1}{2})_n (\frac{3}{2})_s}{2s+1-2n}.
\end{array}$$
\end{solution}
%%%%
%%
%%
%%%%
\begin{problem}\label{problem4chapter11}
By evaluating the integral on the right, using equation (2), page 187, ans term-by-term integration, show that

$$(A) \hspace{30pt} P_n(x) = \dfrac{2}{n! \sqrt{\pi}} \displaystyle\int_0^{\infty} \exp(-t^2) t^n H_n(xt) \mathrm{d}t,$$
which is Curzon's integral for $P_n(x)$, equation (4), page 191.
\end{problem}
\begin{solution}
Consider the integral

$$\begin{array}{ll}
\dfrac{2}{n!\sqrt{\pi}} \displaystyle\int_0^{\infty} e^{-t^2} t^n H_n(xt) \mathrm{d}t &= \dfrac{2}{\sqrt{\pi}} \displaystyle\int_0^{\infty} e^{-t^2} t^n \displaystyle\sum_{k=0}^{[\frac{n}{2}]} \dfrac{(-1)^k (2xt)^{n-2k}}{k! (n-2k)!} \mathrm{d}t \\
&= \displaystyle\sum_{k=0}^{[\frac{n}{2}]} \dfrac{(-1)^k (2k)^{n-2k}}{k! (n-2k)!} \dfrac{1}{\sqrt{\pi}} \displaystyle\int_0^{\infty} e^{-\beta} \beta^{n-k-\frac{1}{2}} \mathrm{d} \beta \\
&= \displaystyle\sum_{k=0}^{[\frac{n}{2}]} \dfrac{(-1)^k (2x)^{n-2k}}{k! (n-2k)!} \dfrac{\Gamma(n-k+\frac{1}{2})}{\Gamma(\frac{1}{2})} \\
&= \displaystyle\sum_{k=0}^{[\frac{n}{2}]} \dfrac{(-1)^k (\frac{1}{2})_{n-k} (2x)^{n-2k}}{k! (n-2k)!}.
\end{array}$$

Hence

$$\dfrac{2}{n! \sqrt{\pi}} \displaystyle\int_0^{\infty} e^{-t^2} t^n H_n(xt) \mathrm{d}t = P_n(x).$$
\end{solution}
%%%%
%%
%%
%%%%
\begin{problem}\label{problem5chapter11}
Let $v_n(x)$ denote the right member of equation $(A)$ of Exercise~\ref{problem4chapter11}. Prove $(A)$ by showing that
$$\displaystyle\sum_{n=0}^{\infty} v_n(x) y^n = (1-2xy+y^2)^{-\frac{1}{2}}.$$
\end{problem}
\begin{solution}
Put $v_n(x) = \dfrac{2}{n! \sqrt{\pi}} \displaystyle\int_0^{\infty} e^{-t^2} t^n H_n(xt) \mathrm{d}t.$
Then
$$\begin{array}{ll}
\displaystyle\sum_{n=0}^{\infty} v_n(x)y^n &= \dfrac{2}{\sqrt{\pi}} \displaystyle\int_0^{\infty} e^{-t^2} \displaystyle\sum_{n=0}^{\infty} \dfrac{t^n H_n(xt) t^n}{n!} \mathrm{d}t \\
&= \dfrac{2}{\sqrt{\pi}} \displaystyle\int_0^{\infty} e^{-t^2} e^{2(xt)yt-y^2t^2}\mathrm{d}t \\
&= \dfrac{2}{\sqrt{\pi}} \displaystyle\int_0^{\infty} e^{-t^2[1-2xy+y^2]} \mathrm{d}t.
\end{array}$$
Use $t \sqrt{1-2xy+y^2}=\beta$ to get
$$\begin{array}{ll}
\displaystyle\sum_{n=0}^{\infty} v_n(x) y^n &= \dfrac{2}{\sqrt{\pi}\sqrt{1-2xy+y^2}} \displaystyle\int_0^{\infty} e^{-\beta^2} \mathrm{d} \beta \\
&= (1-2xy+y^2)^{-\frac{1}{2}} \\
&= \displaystyle\sum_{n=0}^{\infty} P_n(x) y^n.
\end{array}$$
Hence $v_n(x) = P_n(x).$
\end{solution}
%%%%
%%
%%
%%%%
\begin{problem}\label{problem6chapter11}
Evaluate the integral on the right in 

$$(B) \hspace{30pt} H_n(x) = 2^{n+1} \exp(x^2) \displaystyle\int_x^{\infty} \exp(-t^2) t^{n+1} P_n \left( \dfrac{x}{t} \right) \mathrm{d}t$$

by using

$$(2t)^n P_n \left( \dfrac{x}{t} \right) = \displaystyle\sum_{k=0}^{[\frac{n}{2}]} \dfrac{n! (x^2-t^2)^k (2x)^{n-2k}}{(k!)^2 (n-2k)!}$$

derived from equation (1), page 164, and term-by-term integration to prove the validity of $(B)$, which is equation (5), page 191.
\end{problem}
\begin{solution}
We know

$$P_n(x) = \displaystyle\sum_{k=0}^{[\frac{n}{2}]} \dfrac{n! (x^2-1)^k x^{n-2k}}{2^{2k}(k!)^2 (n-2k)!}$$

from (1), page 280. Then

$$(2t)^n P_n \left( \dfrac{x}{t} \right) = \displaystyle\sum_{k=0}^{[\frac{n}{2}]} \dfrac{n! (x^2-t^2)^k (2x)^{n-2k}}{(k!)^2 (n-2k)!}$$

which leads to the result

\begin{eqnarray*}
\lefteqn{2^{n+1} e^{x^2} \displaystyle\int_x^{\infty} e^{-t^2} t^{n+1} P_n \left( \dfrac{x}{t} \right) \mathrm{d}t} \\
&& = \displaystyle\sum_{k=0}^{[\frac{n}{2}]} \dfrac{n! e^{x^2}}{(k!)^2 (n-2k)!} \displaystyle\int_x^{\infty} e^{-t^2} (x^2-t^2)^k (2x)^{n-2k}(2t \mathrm{d}t).
\end{eqnarray*}
Put $t-x^2=\beta$. Then
\begin{eqnarray*}
\lefteqn{2^{n+1} e^{x^2} \displaystyle\int_x^{\infty} e^{-t^2} t^{n+1}P_n \left( \dfrac{x}{t} \right) \mathrm{d}t} \\
& &= \displaystyle\sum_{k=0}^{[\frac{n}{2}]} \dfrac{n! (2x)^{n-2k}}{(k!)^2 (n-2k)!} \displaystyle\int_0^{\infty} e^{-\beta} (-1)^k \beta^k \mathrm{d} \beta \\
& &= \displaystyle\sum_{k=0}^{[\frac{n}{2}]} \dfrac{n! (-1)^k (2x)^{n-2k} k!}{(k!)^2 (n-2k)!} \\
& &= \displaystyle\sum_{k=0}^{[\frac{n}{2}]} \dfrac{n! (-1)^k (2x)^{n-2k}}{k! (n-2k)!}
\end{eqnarray*}
Hence
$$2^{n+1} e^{x^2} \displaystyle\int_x^{\infty} e^{-t^2} t^{n+1} P_n \left( \dfrac{x}{t} \right) \mathrm{d}t = H_n(x).$$
\end{solution}
%%%%
%%
%%
%%%%
\begin{problem}\label{problem7chapter11}
Use the Rodrigues formula
$$\exp(-x^2)H_n(x) = (-1)^n D^n \exp(-x^2); \mathscr{D} := \dfrac{\mathrm{d}}{\mathrm{d}x}$$
an iteration integration by parts to show that
$$\displaystyle\int_{-\infty}^{\infty} \exp(-x^2)H_n(x) H_m(x) \mathrm{d}x = \left\{ \begin{array}{ll}
0 &; m \neq n \\
2^n n! \sqrt{\pi} &; m=n
\end{array} \right.$$
\end{problem}
\begin{solution}
We know $H_n(x) = (-1)^n e^{x^2} \mathscr{D}^n e^{-x^2}$.
Then
\begin{eqnarray*}
\lefteqn{\displaystyle\int_{-\infty}^{\infty} e^{-x^2} H_n(x) H_m(x) \mathrm{d}x} \\
& &= (-1)^n \displaystyle\int_{\infty}^{\infty} [\mathscr{D}^n e^{-x^2}] H_m(x) \mathrm{d}x \\
& &= (-1)^n \left[ \left\{ \mathscr{D}^{n-1} e^{-x^2} \right\} H_m(x) \right]_{\infty}^{\infty} + (-1)^{n+1} \displaystyle\int_{-\infty}^{\infty} [\mathscr{D}^{n-1}e^{-x^2}][\mathscr{D}H_m(x)]\mathrm{d}x \\
& &= \ldots \\
& &= (-1)^{2n} \displaystyle\int_{-\infty}^{\infty} e^{-x^2} [\mathscr{D}^n H_m(x)] \mathrm{d}x.
\end{eqnarray*}
Hence
$$\displaystyle\int_{-\infty}^{\infty} e^{-x^2} H_n(x) H_m(x) \mathrm{d}x = 0, m \neq n$$
and, since $H_n(x) = 2^n x^n + \pi_{n-2}(x)$ (where $\pi_{n-2}(x)$ denotes a polynomial of degree $(n-2)$),
$$\begin{array}{ll}
\displaystyle\int_{-\infty}^{\infty} e^{-x^2}H_n^2(x) \mathrm{d}x &= \displaystyle\int_{-\infty}^{\infty} e^{-x^2} [\mathscr{D}^n H_n(x)] \mathrm{d}x \\
&= 2^n n! \displaystyle\int_{-\infty}^{\infty} e^{-x^2} \mathrm{d}x \\
&= 2^n n! \sqrt{\pi}.
\end{array}$$
\end{solution}