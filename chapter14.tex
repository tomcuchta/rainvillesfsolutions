%%%%
%%
%%
%%%%
%%%% CHAPTER 14
%%%% CHAPTER 14
%%%%
%%
%%
%%%%
\section{Chapter 14 Solutions}
\begin{center}\hyperref[toc]{\^{}\^{}}\end{center}
\begin{center}\begin{tabular}{lllllllllllllllllllllllll}
\hyperref[problem1chapter14]{P1} & \hyperref[problem2chapter14]{P2} & \hyperref[problem3chapter14]{P3} & \hyperref[problem4chapter14]{P4} & \hyperref[problem5chapter14]{P5} & \hyperref[problem6chapter14]{P6} & \hyperref[problem7chapter14]{P7} & \hyperref[problem8chapter14]{P8} & \hyperref[problem9chapter14]{P9} & \hyperref[problem10chapter14]{P10} & \hyperref[problem11chapter14]{P11} & \hyperref[problem12chapter14]{P12} & \hyperref[problem13chapter14]{P13} \\
\hyperref[problem14chapter14]{P14} & \hyperref[problem15chapter14]{P15} & \hyperref[problem16chapter14]{P16} & \hyperref[problem17chapter14]{P17} 
\end{tabular}\end{center}
\setcounter{problem}{0}
\setcounter{solution}{0}
\begin{problem}\label{problem1chapter14}
Show that the Bateman's polynomial (see Section 146, page 285)

$$Z_n(t) = {}_2F_2(-n,n+1;1,1;t)$$

has the recurrence relation

\begin{eqnarray*}
\lefteqn{n^2(2n-3)Z_n(t) - (2n-1)[3n^2-6n+2-2(2n-3)t]Z_{n-1}(t)} \\
& &  +(2n-3)[3n^2-6n+2+2(2n-1)t]Z_{n-2}(t) - (2n-1)(n-2)^2Z_{n-3}(t)=0.
\end{eqnarray*}
  
\end{problem}
\begin{solution}
Let

$$Z_n(t) = {}_2F_2(-n,N=1;1,1;t) = \displaystyle\sum_{k=0}^n \dfrac{(-1)^k (n+k)! t^k}{(k!)^3(n-k)!}$$

or

$$Z_n(t) = \displaystyle\sum_{k=0}^{\infty} \gamma(k,n).$$

Then

$$Z_{n-1}(t) = \displaystyle\sum_{k=0}^{\infty} \dfrac{(-1)^k (n-1+k)!t^k}{(k!)^3(n-1-k)!} = \displaystyle\sum_{k=0}^{\infty} \dfrac{n-k}{n+k} \gamma(k,n),$$

$$Z_{n-2}(t) = \displaystyle\sum_{k=0}^{\infty} \dfrac{(n-k)(n-k-1)(n-k-2)}{(n+k)(n+k-1)(n+k-2)} \gamma(k,n).$$

Also

$$t Z_{n-1}(t) = \displaystyle\sum_{k=0}^{\infty} \dfrac{(-1)^k (n-1+k)! x^{k+1}}{(k!)^3(n-1-k)!} = \displaystyle\sum_{k=1}^{\infty} \dfrac{(-1)^{k-1}(n-2+k)! x^k}{[(k-1)!]^3(n-k)!}$$

or

$$t Z_{n-1}(t) = \displaystyle\sum_{k=0}^{\infty} \dfrac{-k^3}{(n+k)(n+k-1)} \gamma(k,n)$$

$$t Z_{n-2}(t) = \displaystyle\sum_{k=0}^{\infty} \dfrac{(-1)^k (n-2+k)! x^{k+1}}{(k!)^3(n-2-k)!} = \displaystyle\sum_{k=1}^{\infty} \dfrac{(-1)^{k-1} (n-3+k)!x^k}{[(k-1)!]^3(n-1-k)!}$$

$$t Z_{n-2}(t) = \displaystyle\sum_{k=0}^{\infty} \dfrac{-k^3(n-k)}{(n+k)(n+k-1)(n+k-2)} \gamma(k,n).$$

We may now proceed to a recurrence relation of the form

$$Z_n(t) + [A+Bt]Z_{n-1}(t) + [C+Dt]Z_{n-2}(t) + E Z_{n-3}(t)=0,$$

in which $A,B,C,D,E$ depend upon $k$. For the determination of those coefficients we need to satisfy the identity in $k$:
\begin{eqnarray*}
\lefteqn{(n+k)(n+k-1)(n+k-2)+A(n-k)(n+k-1)(n+k-2)-Bk^3(n+k-2)} \\
& & +C(n-k)(n-k-1)(n+k-2)-Dk^3(n-k)+E(n-k)(n-k-1)(n-k-2)=0.
\end{eqnarray*}
Use
$$\begin{array}{l|l|l}
k=n: & 2n(2n-1)(2n-2)-Bn^3(2n-2) = 0: & B= \dfrac{2(2n-1)}{n^2} \\
\mathrm{coeff} \hspace{3pt} k^4: &-B+D=0: & D = \dfrac{2(2n-1)}{n^2} \\
k=2-n: & -D (2-n)^3(2n-2) + E(2n-2)(2n-3)(2n-4)=0 \\
& D(n-2)^3 + 2E(n-2)(2n-3)=0 & E = -\dfrac{(n-2)^2}{2(2n-3)}D  \\
&& = -\dfrac{(2n-1)(n-2)^2}{n^2 (2n-3)} \\
k=1-n: & -B(1-n)^3(-1) + C(2n-1)(2n-2)(-1) \\
& -D(1-n)^3(2n-1)+E(2n-1)(2n-2)(2n-3)=0, \\
& or \\
& -B(n-1)^3 - 2C(n-1)(2n-1) + D(n-1)^3(2n-1) \\
& + E \cdot 2 (n-1)(2n-1)(2n-3) =0
\end{array}$$

Thus we have

$$\begin{array}{ll}
2C(2n-1) &= 2E(2n-1)(2n-3) + D(n-1)^2(2n-1) -B(n-1)^2 \\
&= -\dfrac{2(2n-1)^2(n-2)^2(2n-3)}{n^2(2n-3)} + \dfrac{2(n-1)^2(2n-1)^2}{n^2} - \dfrac{2(n-1)^2(2n-1)}{n^2}.
\end{array}$$

Then

$$\begin{array}{ll}
n^2C &= -(2n-1)(n-2)^2 + (n-1)^2(2n-1)-(n-1)^2 \\
&= (2n-1)[-n^2+4n-4+n^2-2n+1]-(n-1)^2 \\
&= (2n-1)(2n-3)-(n-1)^2 \\
&= 4n^2-8n+3-n^2+2n-1 \\
&= 3n^2-6n+2
\end{array}$$

Hence

$$C = \dfrac{3n^2-6n+2}{n^2}.$$

$$\begin{array}{ll}
k=n-1: & (2n-1)(2n-2)(2n-3)+A(1)(2n-2)(2n-3) \\
& -B(n-1)^3(2n-3)-D(n-1)^3(1)=0 \\
& or \\
& 2(2n-1)(2n-3) + 2A(2n-3) - B(n-1)^2(2n-3) \\
& -D(n-1)^2=0, \\
& or \\
& 2A(2n-3) = -2(2n-1)(2n-3) + \dfrac{2(2n-1)(n-1)^2(2n-3)}{n^2} \\
& + \dfrac{2(2n-1)(n-1)^2}{n^2}.
\end{array}$$

Hence we get

$$\begin{array}{ll}
n^2(2n-3)A &= -n^2(2n-1)(2n-3) + (2n-1)(n-1)^2(2n-3)+(2n-1)(n-1)^2 \\
&= (2n-1)[(2n-3)(-n^2+n^2-2n+1) + (n-1)^2 ] \\
&= (2n-1)[-4n^2+8n-3+n^2-2n+1] \\
&= -(2n-1)(3n^2-6n+2).
\end{array}$$

We now have

$$A = -\dfrac{(2n-1)(3n^2-6n+2)}{n^2(2n-3)}, B=\dfrac{2(2n-1)}{n^2},$$
$$C+\dfrac{3n^2-6n+2}{n^2}, D = \dfrac{2(2n-1)}{n^2},$$
and
$$E = -\dfrac{(2n-1)(n-2)^2}{n^2(2n-3)}.$$

The desired recurrence relation is therefore

$$n^2(2n-3)Z_n(t) - (2n-1) \left[ 3n^2-6n+2-2(2n-3)t \right] Z_{n-1}(t)$$

$$+(2n-3)[3n^2-6n+2+2(2n-1)t]Z_{n-2}(t) - (n-2)^2(2n-1)Z_{n-3}(t).$$


Note the one sign change from that in Sister Celine's paper. The sign was correct in her thesis.
\end{solution}
%%%%
%%
%%
%%%%
\begin{problem}\label{problem2chapter14}
Show that Sister Celine's polynomial $f_n(a;-;x),$ or 

$$F_n(x) = {}_3F_2 \left(-n,n+1,a;1,\dfrac{1}{2};x \right)$$

has the recurrence relation

$$nf_n(x) - [3n-2-4(n-1+a)x]f_{n-1}(x) + [3n-4-4(n-1-a)x]f_{n-2}(x)-(n-2)f_{n-3}(x)=0.$$
\end{problem}
\begin{solution}
Consider

$$f_n(x) = {}_3F_2 \left(-n,n+1,a;1, \dfrac{1}{2};x \right),$$

for which we seek a pure recurrence relation.

Now

$$f_n(x) = \displaystyle\sum_{k=0}^n \dfrac{(-1)^k(n+k)!(a)_k x^k}{k! (n-k)!k! (k)_k} = \displaystyle\sum_{k=0}^n \dfrac{(-1)^k(n+k)!(a)_k(4x)^k}{k!(2k)!(n-k)!}.$$

Put

$$\epsilon(k,n) = \dfrac{(-1)^k(n+k)!(a)_k(4x)^k}{k!(2k)!(n-k)!}.$$

Then

$$f_n(x) = \displaystyle\sum_{k=0}^{\infty} \epsilon(k,n),$$

$$f_{n-1}(x) = \displaystyle\sum_{k=0}^{\infty} \dfrac{(-1)^k (n-1+k)!(a)_k(4x)^k}{k!(2k)!(n-1-k)!} = \displaystyle\sum_{k=0}^{\infty} \dfrac{n-k}{n+k} \epsilon(k,n),$$

$$f_{n-2}(x) = \displaystyle\sum_{k=0}^{\infty} \dfrac{(n-k)(n-k-1)}{(n+k)(n+k-1)}\epsilon(k,n),$$

$$f_{n-3}(x) = \displaystyle\sum_{k=0}^{\infty} \dfrac{(n-k)(n-k-1)(n-k-2)}{(n+k)(n+k-1)(n+k-2)} \epsilon(k,n).$$

Furthermore,

$$\begin{array}{ll}
(4x) f_{n-1}(x) &= \displaystyle\sum_{k=0}^{\infty} \dfrac{(-1)^k (n-1+k)! (a)_k (4x)^{k+1}}{k! (2k)! (n-1-k)!} \\
&= \displaystyle\sum_{k=1}^{\infty} \dfrac{(-1)^{k-1}(n-2+k)!(a)_{k-1}(4x)^k}{(k-1)! (2k-2)! (n-k)!} \\
&= \displaystyle\sum_{k=0}^{\infty} \dfrac{-k \cdot 2k(2k-1)}{(n+k)(n+k-1)(a+k-1)} \epsilon(k,n),
\end{array}$$

and

$$(4x)f_{n-2}(x) = \displaystyle\sum_{k=0}^{\infty} \dfrac{(-1)^k (n-2+k)! (a)_k(4x)^{k+1}}{k! (2k)! (n-2-k)!} = \displaystyle\sum_{k=1}^{\infty} \dfrac{(-1)^{k-1}(n-3+k)! (a)_{n-1} (4x)^k}{(k-1)! (2k-2)! (n-1-k)!}$$

or

$$(4x)f_{n-2}(x) = \displaystyle\sum_{k=0}^{\infty} \dfrac{-k(2k)(2k-1)(n-k)}{(n+k)(n+k-1)(n+k-2)(a+k-1)} \epsilon(k,n).$$

We therefore seek an identity of the form

$$f_n(x) + [A+B(4x)]f_{n-1}(x) + [C+D(4x)]f_{n-2}(x)+E f_{n-3}(x) =0,$$

in which $A,B,C,D,E$ may depend upon $n$. We are thus led to the identity (in $k$)

$$(1) \hspace{30pt} 1 + A \dfrac{n-k}{n+k} - B \dfrac{2k^2(2k-1)}{(n+k)(n+k-1)(a+k-1)} + C \dfrac{(n-k)(n-k-1)}{(n+k)(n+k-1)}$$

$$-D \dfrac{2k^2(2k-1)(n-k)}{(n+k)(n+k-1)(n+k-2)(a+k-1)} + E \dfrac{(n-k)(n-k-1)(n-k-2)}{(n+k)(n+k-1)(n+k-2)} \equiv 0,$$

or

$$(2) \hspace{30pt} (n+k)(n+k-1)(n+k-2)(a+k-1) + A( n-k)(n+k-1)(n+k-2)(a+k-1)$$

$$-2Bk^2(2k-1)(n+k-2) + C(n-k)(n-k-1)(n+k-2)(a+k-1)$$

$$-2Dk^2(2k-1)(n-k) + E(n-k)(n-k-1)(n-k-2)(a+k-1) \equiv 0.$$
We need five equations for the determination of $A,B,C,D,E$. Use
$$\begin{array}{lll}
k=n: & 2n(2n-1)(2n-2)(a+n-1) \\
& -2Bn^2(2n-1)(2n-2)=0, & \\
& or \\
& a+n-1 -Bn = 0 & B = \dfrac{a+n-1}{n}. \\
& \\
k=1-a: & -2B(1-a)^2(1-2a)(n-1-a) \\
& -2D(1-a)^2(1-2a)(n-1+a)=0, \\
& or \\
& B(n-1-a)+D(n-1+a)=0 & D= - \dfrac{(n-1-a)B}{a+n-1} \\
& & \phantom{D}=-\dfrac{n-1-a}{n} \\
& \\
k=2-n: & -2D(2-n)^2(3-2n)(2n-2) \\
& + E(2n-2)(2n-3)(2n-4)(a-n+1) =0, \\
& or \\
& D(n-2)+E(a-n+1)=0; & E = -\dfrac{n-2}{1+a-n} \\
& & \phantom{E}=-\dfrac{n-2}{n} \\
& \\
k=0: & n(n-1)(n-2)(a-1) \\
& +An(n-1)(n-2)(a-1) \\
& +Cn(n-1)(n-2)(a-1) \\
& +En(n-1)(n-2)(a-1)=0 \\
\mathrm{and} \\
\mathrm{coeff \hspace{2pt}} k^4: & 1-a-4B+C+4D-E=0 \\
k=0: & 1+A+C+E=0.
\end{array}$$

Hence 

$$2+2C-4B+4D=0.$$

Therefore we have

$$\begin{array}{ll}
C &= -1 + 2B - 2D \\
&= -1 + \dfrac{2E + 2n-2}{n} + \dfrac{2n-2-2a}{n}
\end{array}$$

from which we get

$$C = \dfrac{3n-4}{n}.$$

Finally,

$$A = -1-C-E = -1 - \dfrac{3n-4}{n} + \dfrac{n-2}{n}$$

so

$$A = -\dfrac{3n-2}{n}.$$

Thus we obtain

$$nf_n(x) - [(3n-2)-(4x)(a+n-1)]f_{n-1}(x) + [3n-4-4x(n-1-a)]f_{n-2}(x) - (n-2)f_{n-3}(x)=0.$$
\end{solution}
%%%%
%%
%%
%%%%
\begin{problem}\label{problem3chapter14}
Show that Rice's polynomial (see Section 147, page 287)

$$H_n = H_n(\zeta,p,v) = {}_3F_2(-n,n+1,\zeta;1,p;v)$$

satisfies the relation

\begin{eqnarray*}
\lefteqn{n(2n-3)(p+n-1)H_n - (2n-1)[(n-2)(p-n+1)+2(n-2)(2n-3)} \\
& & -2(2n-3)(\zeta+n-1)v]H_{n-1}+(2n-3)[2(n-1)^2-n(p-n+1) \\
& & +2(2n-1)(\zeta-n+1)v]H_{n-2}+(n-2)(2n-1)(p-n+1)H_{n-3}=0.
\end{eqnarray*}
\end{problem}
\begin{solution}
Now

$$\begin{array}{ll}
H_n(\zeta,p,v) &= {}_3F_2 (-n,n+1,\zeta;1,p;v) \\
&= \displaystyle\sum_{k=0}^n \dfrac{(-1)^k (n+k)! (\zeta)_k v^k}{(k!)^2(n-k)! (p)_k} \\
&= \displaystyle\sum_{k=0}^{\infty} \epsilon(k,n).
\end{array}$$

Then

$$H_{n-1} = \displaystyle\sum_{k=0}^{\infty} \dfrac{(-1)^k (\zeta)_k (n-1+k)! v^k}{(k!)^2 (p)_k (n-1-k)!} = \displaystyle\sum_{k=0}^{\infty} \dfrac{n-k}{n+k} \epsilon(k,n),$$

$$H_{n-2} = \displaystyle\sum_{k=0}^{\infty} \dfrac{(n-k)(n-k-1)}{(n+k)(n+k-1)} \epsilon(k,n),$$

$$H_{n-3} = \displaystyle\sum_{k=0}^{\infty} \dfrac{(n-k)(n-k-1)(n-k-2)}{((n+k) (n+k-1)(n+k-2)} \epsilon(k,n),$$

$$vH_{n-1} = \displaystyle\sum_{k=0}^{\infty} \dfrac{(-1)^k (\zeta)_k (n-1+k)! v^{k+1}}{(k!)^2(p)_k (n-1-k)!} = \displaystyle\sum_{k=1}^{\infty} \dfrac{(-1)^{k-1} (\zeta)_{k-1} (n-2+k)! v^k}{[(k-1)!]^2(p)_{k-1}(n-k)!}$$

$$vH_{n-1} = \displaystyle\sum_{k=0}^{\infty} \dfrac{-k^2(p+k-1)}{(n+k)(n+k-1)(\zeta+k-1)} \epsilon(k,n),$$

and

$$vH_{n-2} = \displaystyle\sum_{k=0}^{\infty} \dfrac{-k^2(p+k-1)(n-k)}{(n+k)(n+k-1)(n+k-2)(\zeta+k-1)} \epsilon(k,n).$$

Then there exists a relation

$$H_n + (A+Bv)H_{n-1} + (C+Dv)H_{n-2} + EH_{n-3} =0,$$

in which $A,B,C,D,E$ depend upon $n$ alone. We are led to the identity in $k$:
$$\begin{array}{ll}
(n+k)(n+k-1)(n+k-2)(\zeta+k-1)+A(n-k)(n+k-1)(n+k-2)(\zeta+k-1) \\
-Bk^2(p+k-1(n+k-1) + C(n-k)(n-k-1)(n_k-1)(n+k-2)(\zeta+k-1) \\
- Dk^2(p+k-1)(n-k)+E (n-k)(n-k-1)(n-k-2)(\zeta+k-1)=0
\end{array}$$
We now solve for $A,B,C,D,E:$
$$\begin{array}{lll}
k=1-\zeta: & -B(1-\zeta)^2(p-\zeta)(n-\zeta-1) \\
& -D(1-\zeta)^2(p-\zeta)(n-1+\zeta) & D = \dfrac{-B(n-\zeta-1)}{n+\zeta-1}. \\
k=n: & (2n)(2n-1)(2n-2)(\zeta+n-1) \\
& -Bn^2(p+n-1)(2n-2)=0 & B= \dfrac{2(2n-1)(n+\zeta-1)}{n(n+p-1)} \\
& & D = \dfrac{-2(2n-1)(n-\zeta-1)}{n(n+p-1)} \\
k=2-n: & -D(2-n)^2(p+1-n)(2n-2) \\
& +E(2n-2)(2n-3)(2n-4)(\zeta+1-n)=0 & E = \dfrac{(n-2)(p+1-n)}{2(2n-3)(\zeta+1-n)}D \\
&& \phantom{E}=\dfrac{(n-2)(2n-1)(p+1-n)}{n(2n-3)(n+p-1)} \\
k=1-n: & -B(1-n)^2(p-n)(-1) \\
& + C(2n-1)(2n-2)(-1)(\zeta-n) \\
& -D(1-n)^2(p-n)(2n-1) \\
& +E(2n-1)(2n-2)(2n-3)(p-n)=0;
\end{array}$$

we have

$$B(n-1)(p-n)-2C(2n-1)(\zeta-n)-D(n-1)(p-n)(2n-1)+2E(2n-1)(2n-3)(p-n)=0,$$

or

\begin{eqnarray*}
\lefteqn{\dfrac{-(2n-1)(n+p-1)(n-1)(p-n)}{n(n+p-1)}} \\
& & - 2C(2n-1)(\zeta-n) + \dfrac{2(2n-1)(n-\zeta-1)(n-1)(p-n)(2n-1)}{n(n+p-1)} \\
& & + \dfrac{2(n-2)(2n-1)(p+1-n)(2n-1)(2n-3)(\zeta-n)}{n(2n-3)(n-p-1)} = 0,
\end{eqnarray*}

or

\begin{eqnarray*}
\lefteqn{n(\zeta-n)(n+p-1)C = (n+\zeta-1)(n-1)(p-n) + (n-\zeta-1)(n-1)(p-n)(2n-1)} \\ 
& & + (n-2)(p+1-n)(2n-1)(p-n) \\
&& =\lefteqn{(n-1)(p-n)[n+\zeta-1+(n-\zeta-1)(2n-1)]} \\
&& +(n-2)(p+1-n)(2n-1)(\zeta-n) \\
&&=-2(n-1)^2(p-n)(\zeta-n) + (n-2)(p+1-n)(2n-1)(\zeta-n).
\end{eqnarray*}

Hence

$$\begin{array}{ll}
n(n+p-1)C &= -2(n-1)^2(p-n)+(n-2)(2n-1)(p+1-n) \\
&= -2(n-1)^2(p-n+1)+2(n-1)^2 + (n-2)(2n-1)(p+1-n) \\
&= 2(n-1)^2+(p-n+1)[-2n^2+4n-2+2n^2-5n+2] \\
&= 2(n-1)^2-n(p-n+1),
\end{array}$$

so

$$C = \dfrac{2(n-1)^2-n(p-n+1)}{n(p+n-1)}.$$

Finally,

$$\mathrm{coeff \hspace{3pt}} k^4: 1-A-B+C+D-E=0.$$

Then

$$A = 1-B+C+D-E$$

so

\begin{eqnarray*}
\lefteqn{n(2n-3)(p+n-1)A = n(2n-3)(p+n-1)-2(2n-3)(2n-1)(n+\zeta-1)} \\
&&+ (2n-3)[2(n-1)^2-n(p-n+1)]-2(2n-3)(2n-2)(n-\zeta-1) \\
&&- (n-2)(2n-1)(p+1-n),
\end{eqnarray*}

or

$$\begin{array}{ll}
n(2n-3)(p+n-1)A &= (2n-3)[pn+n^2-n+2n^2-4n+2-pn+n^2-n] \\
&\phantom{=}- (2n-1)[2(2n-3)(n+\zeta-1) + 2(2n-3)(n-\zeta-1)\\
&\phantom{=}+(n-2)(p+1-n)] \\
&= 2(2n-3)(n-1)(2n-1) - 2(2n-1)(2n-3)(n+\zeta-1)\\
&\phantom{=}-(2n-1)[2(2n-3)(n-p-1)+(1-2)(p+1-n)] \\
&=(2n-1)[2(2n-3)(-n-\zeta+1-n+\zeta+1)+2(n-1)(2n-3) \\
&\phantom{=}-(n-2)(p+1-n)] \\
&= (2n-1)[-4(n-1)(2n-3) + 2(n-1)(2n-3) - (n-2)(p+1-n)] \\
&=-(2n-1)[(n-2)(p+1-n)+2(n-1)(2n-3)].
\end{array}$$

Hence we arrive at the recurrence relation

$$n(2n-3)(p+n-1)H_n$$

$$-(2n-1)[(n-2)(p+1-n)+2(n-1)(2n-3)-2(2n-3)(n+p-1)v]H_{n-1}$$
$$+(2n-3)[2(n-1)^2-n(p+1-n)-2(2n-1)(n-p-1)v]H_{n-2}$$
$$+(n-3)(2n-1)9p+1-n)H_{n-3}=0.$$
\end{solution}
%%%%
%%
%%
%%%%
\begin{problem}\label{problem4chapter14}
Show that the polynomial

$$f_n(x) = {}_1F_2(-n;1+\alpha,1+\beta;x),$$

which is intimately related to Bateman's $J_n^{u,v}$ of Section 147, page 287, satisfies the relation
\begin{eqnarray*}
\lefteqn{(\alpha+n)(\beta+n)f_n(x) - [3n^2-3n+1+(2n-1)(\alpha+\beta)+\alpha \beta-x]f_{n-1}(x)}\\
& & +(n-1)(3n-3+\alpha+\beta)f_{n-2}(x)-(n-1)(n-2)f_{n-3}(x)=0.
\end{eqnarray*}
\end{problem}
\begin{solution}
Let $f_n(x) = {}_1F_2(-m;1+\alpha,1+\beta;x) = \displaystyle\sum_{k=0}^n \dfrac{(-1)^k n x^k}{k! (1+\alpha)_k (1+\beta)_k(n-k)!}$.

Put $g_n(x) = \dfrac{f_n(x)}{n!}$. Then

$$g_n(x) = \displaystyle\sum_{k=0}^n \dfrac{(-1)^k x^k}{k! (1+\alpha)_k (1+\beta)_k(n-k)!} = \displaystyle\sum_{k=0}^{\infty} \epsilon(k,n).$$

Now 

$$g_{n-1}(x) = \displaystyle\sum_{k=0}^{\infty} \dfrac{(-1)^k x^k}{k!(1+\alpha)_k(1+\beta)_k(n-1-k)!} = \displaystyle\sum_{k=0}^{\infty} (n-k) \epsilon(k,n)$$

and

$$g_{n-2}(x) = \displaystyle\sum_{k=0}^{\infty} (n-k)(n-k-1) \epsilon(k,n)$$

$$g_{n-3}(x) = \displaystyle\sum_{k=0}^{\infty} (n-k)(n-k-1)(n-k-2) \epsilon(k,n).$$

Also

$$x g_{n-1}(x) = \displaystyle\sum_{k=0}^{\infty} \dfrac{(-1)^k x^{k+1}}{k! (1+\alpha)-k (1+\beta)_k (n-1-k)!} = \displaystyle\sum_{k=1}^{\infty} \dfrac{(-1)^{k-1} x^k}{(k-1)! (1+\alpha)_{k-1} (1+\beta)_{k-1} (n-k)!}$$

or

$$x g_{n-1}(x) = \displaystyle\sum_{k=0}^{\infty} -k(\alpha+k)(\beta+k)\epsilon(k,n).$$

Then there exists the relation

$$g_n(x) + (A+Bx)g_{n-1}(x) + Cg_{n-2}(x) + Dg_{n-3}(x) =0,$$

with $A,B,C,D$ dependent only on $n$.

We are led to the identity in $k$

$$1+A(n-k)-Bk(\alpha+k)(\beta+k) + C(n-k)(n-k-1)+D(n-k)(n-k-1)(n-k-2)=0.$$

Then use

$$\hspace{-20pt} \begin{array}{lll}
k=n: & 1-Bn(a+n)(\beta+n) = 0; & B = \dfrac{1}{n(\alpha+n)(\beta+n)}. \\
\mathrm{coeff \hspace{3pt}} k^3: & -B-D=0; & D = \dfrac{-1}{n(\alpha+n)(\beta+n)} \\
k=n-1:& 1+A-B(n-1)(\alpha+n-1)(\beta+n-1)=0; & A = \dfrac{(n-1)(\alpha+n-1)(\beta+n-1)}{n(\alpha+n)(\beta+n)} \\
&& \phantom{A=}-\dfrac{-n(\alpha+n)(\beta+n)}{n(\alpha+n)(\beta+n)} \\
& & \phantom{A}=-\dfrac{3n^2-3n+1+(2n-1)(\alpha+\beta)+\alpha \beta}{n(\alpha+n)(\beta+n)} \\
k=0: & 1+An+Cn(n-1)+Dn(n-1(n-2)=0
\end{array}$$

Then

\begin{eqnarray*}
\lefteqn{n(n-1)(\alpha+n)(\beta+n)C = -(\alpha+n)(\beta+n)+3n-3n+1+(\alpha+\beta)(2n-1)} \\
&&\phantom{=}+\alpha \beta + (n-1)(n-2) \\
&&= -n^2-(\alpha+\beta)_n - \alpha\beta + 3n^2 - 3n + 1 + (\alpha+\beta)(2n-1) + \alpha \beta + n^2 - 3n +2 \\
&&= 3n^2-6n+3+(\alpha+\beta)(-n+2n-1) \\
&&= 3(n-1)^2+(n-1)(\alpha+\beta) \\
&&= (n-1)[3n-3+\alpha+\beta]
\end{eqnarray*}

So

$$C = \dfrac{3n-3+\alpha+\beta}{n(\alpha+n)(\beta+n)}.$$

We thus find that

$$n(\alpha+n)(\beta+n)g_n(x) - [3n^2-3n+1+\beta n -1)(\alpha+\beta)+\alpha \beta - x] g_{n-1}(x)$$
$$+(3n-3+\alpha+\beta)g_{n-2}(x) - g_{n-3}(x) =0.$$

Now $g_n(x)=\dfrac{f_n(x)}{n!}.$ Hence we get

\begin{eqnarray*}
\lefteqn{\dfrac{(\alpha+n)(\beta+n)}{(n-1)!} f_n(x) - [3n^2-3n+1+\beta n -1)(\alpha+\beta)+\alpha \beta - x] \dfrac{f_{n-1}(x)}{(n-1)!}} \\
& & + \dfrac{3n-3+\alpha+\beta}{(n-2)!} f_{n-2}(x) - \dfrac{f_{n-3}(x)}{(n-3)!}=0,
\end{eqnarray*}

or

$$(\alpha+n)(\beta+n)f_n(x) - [3n^2-3n+1+(2n-1)(\alpha+\beta)+\alpha \beta-x]f_{n-1}(x)$$
$$+(n-1)[3n-3+\alpha+\beta]f_{n-2}(x)-(n-1)(n-2)f_{n-3}(x)=0.$$
\end{solution}
%%%%
%%
%%
%%%%
\begin{problem}\label{problem5chapter14}
Define the polynomial $w_n(x)$ by

$$w_n(x) = \displaystyle\sum_{k=0}^n \dfrac{(-1)^k n! L_k(x)}{(k!)^2 (n-k)!}$$

in terms of the Laguerre polynomial $L_k(x)$. Show that $w_n(x)$ possesses the pure recurrence relation

\begin{eqnarray*}
\lefteqn{n^2 w_n(x) - [(n-1)(4n-3)+x]w_{n-1}(x)+(6n^2-19n+16+x)w_{n-2}(x)} \\
& & -(n-2)(4n-9)w_{n-3}(x)+(n-2)(n-3)w_{n-4}(x)=0.
\end{eqnarray*}
\end{problem}
\begin{solution}
Let 

$$w_n(x) = \displaystyle\sum_{k=0}^n \dfrac{(-1)^k n! \mathscr{L}_k(x)}{(k!)^2(n-k)!}$$

in terms of the simple Laguerre polynomial. We know that

$$(1) \hspace{30pt} n\mathscr{L}_n(x) - (2n-1-x)\mathscr{L}_{n-1}(x) + (n-1)\mathscr{L}_{n-2}(x) = 0$$

and we seek a recurrence relation for $w_n(x).$

Put $\dfrac{w_n(x)}{n!} = \phi_n(x)$ and $\dfrac{(-1)^k \mathscr{L}_k(x)}{(k!)^2} = g_k(x).$

Then

$$(-1)^k (k!)^2kg_k(x) - (-1)^{k-1}[(k-1)!]^2 (2k-1-x)g_{k-1}(x) + (-1)^{k-2}[(k-2)!]^2(k-1)g_{k-2}(x)=0,$$

or

$$k^3(k-1)g_k(x) +(k-1)(2k-1-x)g_{k-1}(x)+g_{k-2}(x)=0.$$

Also

$$\phi_n(x) = \displaystyle\sum_{k=0}^n \dfrac{g_k(x)}{(n-k)!} = \displaystyle\sum_{k=0}^{\infty} \dfrac{g_k(x)}{(n-k)!}.$$

Then

$$\begin{array}{ll}
\phi_{n-1}(x) = \displaystyle\sum_{k=0}^{\infty} (n-k) \dfrac{g_k(x)}{(n-k)!}, \\
\phi_{n-2}(x) = \displaystyle\sum_{k=0}^{\infty} (n-k(N-k-1) \dfrac{g_k(x)}{(n-k)!},\\
\phi_{n-3}(x) = \displaystyle\sum_{k=0}^{\infty} (n-k)(n-k-1)(n-k-2) \dfrac{g_k(x)}{(n-k)!}, \\
\phi_{n-4}(x) = \displaystyle\sum_{k=0}^{\infty} (n-k)(n-k-1)(n-k-2)(n-k-3) \dfrac{g_k(x)}{(n-k)!}.
\end{array}$$

We first wish to find $A,B,C,D,E$ so that

\begin{eqnarray*}
\lefteqn{A \phi_n(x) + B \phi_{n-1}(x) + C \phi_{n-2}(x) + D \phi_{n-3}(x)} \\
& & + E \phi_{n-4}(x) = \displaystyle\sum_{k=0}^{\infty} \dfrac{k^3(k-1)g_k(x)}{(n-k)!}.
\end{eqnarray*}

The above requires that $A$ to $E$ satisfy the identity in $k$,

\begin{eqnarray*}
\lefteqn{A+B(n-k)+C(n-k)(n-k-1)+D(n-k)(n-k-1)(n-k-2)} \\
& & +E(n-k)(n-k-1)(n-k-2)(n-k-3)=k^3(k-1).
\end{eqnarray*}

We thus get

$$\begin{array}{ll}
k=n: & A=n^3(n-1) \\
\mathrm{coeff \hspace{3pt}} k^4: & E=1 \\
\mathrm{coeff \hspace{3pt}} k^3: & -D-E (n+n-1+n-2+n-3)=-1 \\
& or \\
& D+4n-6=1; D=-(4n-7) \\
k=n-1: & A+B(1)=(n-1)^3(n-2) \\
& B=(n-1)^3(n-2)-n^3(n-1) = (n-1)[(n-1)^2(n-3)-n^3] \\
& B=-(n-1)(4n^2-5n+2). \\
k=n-2: & A+2B+2 \cdot 1 \cdot C = (n-2)^3 (n-3) \\
& 2C = (n-2)^3 (n-3)-N^3(n-1)+2(n-1)(4n^2-5n+2) \\
&\hspace{16pt} = 12n^2-30n+20
\end{array}$$

So $C=2n^2-15n+10$. We now have

$$(1) \hspace{30pt} n^3(n-1)\phi_n(x) - (n-1)(4n^2-5n+2)\phi_{n-1}(x) + (5n^2-15n+10)\phi_{n-2}(x)$$
$$ -(4n-7)\phi_{n-3}(x) + \phi_{n-4}(x) = \displaystyle\sum_{k=0}^{\infty} \dfrac{k^3(k-1)g_k(x)}{(n-k)!}.$$

From

$$\phi_{n-1}(x) = \displaystyle\sum_{k=0}^{\infty} \dfrac{g_k(x)}{(n-1-k)!},$$

we get, with our usual convention that $g_s(x) \equiv 0$ for $s < 0$,

$$\phi_{n-1}(x) = \displaystyle\sum_{k=0}^{\infty} \dfrac{g_{k-1}(x)}{(n-k)!}$$

and in the same way

$$\phi_{n-2}(x) = \displaystyle\sum_{k=0}^{\infty} \dfrac{(n-k)g_{k-1}(x)}{(n-k)!},$$

$$\phi_{n-3}(x) = \displaystyle\sum_{k=0}^{\infty} (n-k)(n-k-1) \dfrac{g_{k-1}(x)}{(n-k)!},etc$$

We now wish to find $F,G,H$, so that

$$F \phi_{n-1}(x) + G \phi_{n-2}(x) + H \phi_{n-3}(x) = \displaystyle\sum_{k=0}^{\infty} (k-1)(2k-1) \dfrac{g_{k-1}(x)}{(n-k)!}.$$

We thus arrive at the identity

$$F + G(n-k)+H(n-k)(n-k-1) = (k-1)(2k-1).$$

Hence

$$\begin{array}{ll}
k=n: & F=(n-1)(2n-1) \\
\mathrm{coeff \hspace{3pt}} k^2: & H=2 \\
\mathrm{coeff \hspace{3pt}} k: & -G -H(n+n-1) = -3 \\
& G+2(2n-1) = 3 \\
& G = -(4n-5)
\end{array}$$

Therefore we get

$$(2) \hspace{30pt} (n-1)(2n-1)\phi_{n-1}(x) - (4n-5)\phi_{n-2}(x) + 2\phi_{n-3}(x) = \displaystyle\sum_{k=0}^{\infty} \dfrac{(k-1)(2k-1)g_{k-1}(x)}{(n-k)!}.$$

From $A_1 \phi_{n-1}(x) + B_1 \phi_{n-2}(x) = \displaystyle\sum_{k=0}^{\infty} \dfrac{-(k-1)g_{k-1}(x)}{(n-k)!},$

we determine $A_1$ and $B_1$ by

$$A_1 + B_1(n-k) \equiv -(k-1).$$

Then $A_1=-(n-1), B_1=1$. Hence

$$(3) \hspace{30pt} -(n-1)\phi_{n-1}(x) + \phi_{n-2}(x) = \displaystyle\sum_{k=0}^{\infty} \dfrac{-(k-1)g_{k-1}(x)}{(n-k)!}.$$

Also

$$(4) \hspace{30pt} \phi_{n-2}(x) = \displaystyle\sum_{k=0}^{\infty} \dfrac{g_k(x)}{(n-2-k)!} = \displaystyle\sum_{k=0}^{\infty} \dfrac{g_{k-2}(x)}{(n-k)!}.$$

Since

$$\displaystyle\sum_{k=0}^{\infty} \dfrac{k^3(k-1)g_k(x)+(k-1)(2k-1-x)g_{k-1}(x)+g_{k-2}(x)}{(n-k)!} =0,$$

we may combine $(1),(2),(3),(4)$ to get

$$n^3(n-1)\phi_n(x) - (n-3)(4n^2-5n+2)\phi_{n-1}(x)+(6n^2-15n+10)\phi_{n-2}(x)$$
$$-(4n-7)\phi_{n-3}(x) + \phi_{n-4}(x) + (n-1)(2n-1) \phi_{n-1}(x) - (4n-5)\phi_{n-3}(x)$$
$$+2\phi_{n-3}(x) - (n-1)x\phi_{n-1}(x) + x\phi_{n-2}(x)+\phi_{n-2}(x)=0,$$

or

$$n^3(n-1)\phi_n(x)-(n-1)[(4n^2-7n+3+x]\phi_{n-1}(x)+ (6n^2-19n+16+x)\phi_{n-2}(x)$$
$$-(4n-9)\phi_{n-3}(x) + \phi_{n-4}(x)=0$$

We recall that $\phi_n(x) = \dfrac{w_n(x)}{n!}$ and use it to obtain the final result:

\begin{eqnarray*}
\lefteqn{n^2w_n(x) - (4n^2-7n+3+x)w_{n-1}(x) + (6n^2-19n+16+x)w_{n-2}(x)} \\
&&-(n-2)(4n-9)w_{n-3}(x)+(n-2)(n-3)w_{n-4}(x)=0.
\end{eqnarray*}
\end{solution}
%%%%
%%
%%
%%%%
%%%%
%%
%%
%%%%
\begin{problem}\label{problem6chapter14}
Show that the polynomial $w_n(x)$ of Exercise~\ref{problem5chapter14} may be written

$$w_n(x)=\displaystyle\sum_{k=0}^n {}_1F_1 \left[ \begin{array}{rlr}
-n+k; & & \\
& & 1 \\
1+k; & & 
\end{array} \right] \dfrac{(-n)_k (-x)^k}{(k!)^3}.$$
\end{problem}
\begin{solution}
Consider

$$w_n(x) = \displaystyle\sum_{k=0}^n \dfrac{(-1)^kn! \mathscr{L}_k(x)}{(k!)^2(n-k)!}.$$

Let us form

$$\begin{array}{ll}
\psi &= \displaystyle\sum_{n=0}^{\infty} \dfrac{w_n(x) t^n}{n!} \\
&= \displaystyle\sum_{n=0}^{\infty} \displaystyle\sum_{k=0}^n \dfrac{(-1)^k \mathscr{L}_k(x)t^n}{(k!)^2(n-k)!} \\
&= \displaystyle\sum_{n,k=0}^{\infty} \dfrac{(-1)^k \mathscr{L}_k(x) t^{n+k}}{(k!)^2n!} \\
&= \displaystyle\sum_{n,k=0}^{\infty} \displaystyle\sum_{s=0}^{\infty} \dfrac{(-1)^{k+s} x^s t^{n+k}}{k! n! (s!)^2(k-s)!} \\
&= \displaystyle\sum_{k,s,n=0}^{\infty} \dfrac{(-1)^k x^s t^{n+k}}{(s!)^2k!k!(k+s)!}
\end{array}$$

Then

$$\begin{array}{ll}
\psi &= \displaystyle\sum_{n,s=0}^{\infty} \displaystyle\sum_{k=0}^n \dfrac{(-1)^k x^s t^{n+s}}{(s!)^2k! (n-k)! (k+s)!} \\
&= \displaystyle\sum_{n,s=0}^{\infty} {}_1F_1 \left[ \begin{array}{rlr}
-n; & & \\
& & 1 \\
1+s; & & 
\end{array} \right] \dfrac{x^s t^{n+s}}{n! (s!)^3} \\
&= \displaystyle\sum_{n=0}^{\infty} \displaystyle\sum_{s=0}^n {}_1F_1 \left[ \begin{array}{rlr}
-n+s; & & \\
& & 1 \\
1+s; & & 
\end{array} \right] \dfrac{x^s t^n}{(s!)^3 (n-s)!}.
\end{array}$$

Hence

$$\begin{array}{ll}
w_n(x) &= \displaystyle\sum_{s=0}^n {}_1F_1\left[ \begin{array}{rlr}
-n+s; & & \\
& & 1 \\
1+s; & & 
\end{array} \right] \dfrac{n! x^s}{(s!)^3(n-s)!} \\
&= \displaystyle\sum_{s=0}^n {}_1F_1 \left[ \begin{array}{rlr}
-n+s; & & \\
& & 1 \\
1+s; & &
\end{array} \right] \dfrac{(-n)_s (-x)^s}{(s!)^3},
\end{array}$$

as desired.
\end{solution}
%%%%
%%
%%
%%%%
%%%%
%%
%%
%%%%
\begin{problem}\label{problem7chapter14}
Define the polynomial $v_n(x)$ by (see Section 131, page 251)

$$v_n(x) = \displaystyle\sum_{k=0}^n \dfrac{(-1)^k n! P_k(x)}{(k!)^2 (n-k)!}$$

in terms of the Legendre polynomial $P_k(x)$. Show that $v_n(x)$ satisfies the recurrence relation

\begin{eqnarray*}
\lefteqn{n^2v_n(x) - [4n^2-5n+2-(2n-1)x]v_{n-1}(x) + [6n^2-15n+11} \\
&&-(4n-5)x]v_{n-2}(x) - (n-2)(4n-7-2x)v_{n-3}(x)+(n-2)(n-3)v_{n-4}(x)=0.
\end{eqnarray*}
\end{problem}
\begin{solution}
Consider

$$v_n(x) = \displaystyle\sum_{k=0}^n \dfrac{(-1)^k n! P_k(x)}{(k!)^2(n-k)!}$$

Put $\dfrac{v_n(x)}{n!} = \sigma_n(x)$ and $\dfrac{(-1)^k P_k(x)}{(k!)^2} = g_k(x).$

We know that $P_k(x)$ satisfies the relation

$$k P_k(X) - (2k-1)x P_{k-1}(x) + (k-1)P_{k-2}(x) = 0.$$

Hence $g_k(x)$ satisfies

$$(-1)^k k (k!)^2 g_k(x) - (-1)^{k-1}[(k-1)!]^2 (2k-1)xg_{k-1}(x) + (-1)^{k-2}(k-1)[(k-2)!]^2g_{k-2}(x)=0,$$

or

$$(1) \hspace{30pt} k^3(k-1)g_k(x) + (k-1)(2k-1)xg_{k-1}(x) + g_{k-2}(x) =0.$$

Also

$$(2)\hspace{30pt} \sigma_n(x) = \displaystyle\sum_{k=0}^{\infty} \dfrac{g_k(x)}{(n-k)!}.$$

Because of the identity of form of equation $(2)$ with the corresponding relations of Exercise~\ref{problem5chapter14}, we may write immediately

$$(3) \hspace{3pt} n^3(n-1)\sigma_n(x) - (n-1)(4n^2-5n+2)\sigma_{n-1}(x) + (6n^2-15n+10)\sigma_{n-2}(x)$$
$$-(4n-7)\sigma_{n-3}(x) + \sigma_{n-4}(x) = \displaystyle\sum_{k=0}^{\infty} \dfrac{k^3(k-1)}{(n-k)!} g_k(x).$$

In the same way we use the work in Exercise~\ref{problem5chapter14} to conclude that

$$(4) \hspace{30pt} (n-1)(2n-1)\sigma_{n-1}(x)-(4n-5)\sigma_{n-2}(x)+2\sigma_{n-3}(x) = \displaystyle\sum_{k=0}^{\infty} \dfrac{(k-1)(2k-1)g_{k-1}(x)}{(n-k)!},$$

and

$$(5) \hspace{30pt} \sigma_{n-2}(x) = \displaystyle\sum_{k=0}^{\infty} \dfrac{g_{k-2}(x)}{(n-k)!}.$$

Since

$$\displaystyle\sum_{k=0}^{\infty} \dfrac{k^3(k-1)g_k(x)+(k-1)(2k-1)xg_{k-1}(x) + g_{k-2}9x)}{(n-k)!}=0,$$

we get from $(3),(4),(5)$ that

$$n^3(n-1)\sigma_n(x)-(n-1)[4n^2-5n+2-(2n-1)x]\sigma_{n-1}(x)$$
$$+[6n^2-15n+11-(4n-5)x]\sigma_{n-2}(x) - [4n-7-2x]\sigma_{n-3}(x) + \sigma_{n-4}(x)=0.$$

But $\sigma_n(x) = \dfrac{v_n(x)}{n!}$. Hence we arrive at the result

$$n^2v_n(x) - [4n^2-5n+2-(2n-1)x]v_{n-1}(x)+[6n^2-15n+11-(4n-5)x]v_{n-2}(x)$$
$$-(n-2)(4n-7-2x)v_{n-3}(x)+(n-2)(n-3)v_{n-4}(x)=0.$$
\end{solution}
%%%%
%%
%%
%%%%
%%%%
%%
%%
%%%%
\begin{problem}\label{problem8chapter14}
Show that the $v_n(x)$ of Exercise~\ref{problem7chapter14} satisfies the relations

$$(1-x^2)v_n''(x)-2xv_n'(x)+n(n=1)v_n(x)=2n^2v_{n-1}9x)-n(n-1)v_{n-2}(x)$$

and

$$(1-x^2)v_n'(x)+nxv_n(x) = [(2n-1)x-1]v_{n-1}(x)-(n-1)xv_{n-2}(x)+(1-x^2)v_{n-1}'(x).$$
\end{problem}
\begin{solution}
(solution by Donald Meyers) Let 
\begin{align*}
v_{n}(x) & =\sum_{k=0}^{n}\frac{(-1)^{k}n!P_{k}(x)}{(k!)^{2}(n-k)!}\\
 & =\sum_{k=0}^{n}\frac{(-1)^{k}}{k!}\dbinom{n}{k}P_{k}(x).
\end{align*}
a) To save space, let 
\[
Q_{n}(x)=(1-x^{2})v_{n}^{\prime\prime}(x)-2xv_{n}^{\prime}(x)+n(n+1)v_{n}(x).
\]
The Legendre polynomials $P_{n}$ solve Legendre's ODE, so 
\begin{align*}
Q_{n}(x) & =\sum_{k=0}^{n}\frac{(-1)^{k}}{k!}\dbinom{n}{k}[(1-x^{2})P_{k}^{\prime\prime}(x)-2xP_{k}^{\prime}(x)+n(n+1)P_{k}(x)]\\
 & =\sum_{k=0}^{n}\frac{(-1)^{k}}{k!}\dbinom{n}{k}[n(n+1)-k(k+1)]P_{k}(x)\\
 & =\sum_{k=0}^{n-1}\frac{(-1)^{k}}{k!}\dbinom{n}{k}(n-k)(n+k+1)P_{k}(x)\\
 & =n\sum_{k=0}^{n-1}\frac{(-1)^{k}}{k!}\dbinom{n-1}{k}(n+k+1)P_{k}(x)\\
 & =n^{2}v_{n-1}(x)+n\sum_{k=0}^{n-1}\frac{(-1)^{k}}{k!}\dbinom{n-1}{k}(n-n+k+1)P_{k}(x)\\
 & =2n^{2}v_{n-1}(x)-n\sum_{k=0}^{n-1}\frac{(-1)^{k}}{k!}\dbinom{n-1}{k}(n-k-1)P_{k}(x)\\
 & =2n^{2}v_{n-1}(x)-n\sum_{k=0}^{n-2}\frac{(-1)^{k}}{k!}\dbinom{n-1}{k}(n-k-1)P_{k}(x)\\
 & =2n^{2}v_{n-1}(x)-n(n-1)\sum_{k=0}^{n-2}\frac{(-1)^{k}}{k!}\dbinom{n-2}{k}P_{k}(x)\\
 & =2n^{2}v_{n-1}(x)-n(n-1)v_{n-2}(x).
\end{align*}
b) We need to show that 
\begin{align*}
(1-x^{2})v_{n}^{\prime}(x)+nxv_{n}(x) & =[(2n-1)x-1]v_{n-1}(x)-(n-1)xv_{n-2}(x)+(1-x^{2})v_{n-1}^{\prime}(x).
\end{align*}
To save space let $u_{n}(x)=(1-x^{2})(v_{n}^{\prime}(x)-v_{n-1}^{\prime}(x)).$
Then 
\begin{align*}
u_{n}(x) & =(1-x^{2})\left[\sum_{k=1}^{n}\frac{(-1)^{k}}{k!}\dbinom{n}{k}P_{k}^{\prime}(x)-\sum_{k=1}^{n-1}\frac{(-1)^{k}}{k!}\dbinom{n-1}{k}P_{k}^{\prime}(x)\right]\\
 & =\sum_{k=1}^{n}\frac{(-1)^{k}}{k!}\dbinom{n}{k}k[P_{k-1}(x)-xP_{k}(x)]-\sum_{k=1}^{n-1}\frac{(-1)^{k}}{k!}\dbinom{n-1}{k}k[P_{k-1}(x)-xP_{k}(x)]\\
 & =\frac{(-1)^{n}}{n!}n[P_{n-1}(x)-xP_{n}(x)]+\sum_{k=1}^{n-1}\frac{(-1)^{k}}{(k-1)!}\left[\dbinom{n}{k}-\dbinom{n-1}{k}\right][P_{k-1}(x)-xP_{k}(x)]\\
 & =-nx\frac{(-1)^{n}}{n!}P_{n}(x)-\frac{(-1)^{n-1}}{(n-1)!}P_{n-1}(x)+\sum_{k=1}^{n-1}\frac{(-1)^{k}}{(k-1)!}\dbinom{n-1}{k-1}[P_{k-1}(x)-xP_{k}(x)]\\
 & =-nx\frac{(-1)^{n}}{n!}P_{n}(x)-\frac{(-1)^{n-1}}{(n-1)!}P_{n-1}(x)-\sum_{k=0}^{n-2}\frac{(-1)^{k}}{k!}\dbinom{n-1}{k}[P_{k}(x)-xP_{k+1}(x)]\\
 & =-nx\frac{(-1)^{n}}{n!}P_{n}(x)-\frac{(-1)^{n-1}}{(n-1)!}P_{n-1}(x)-x\frac{(-1)^{n-1}}{(n-1)!}(n-1)^{2}P_{n-1}(x)\\
 & -\sum_{k=0}^{n-2}\frac{(-1)^{k}}{k!}\dbinom{n-1}{k}P_{k}(x)+x\sum_{k=0}^{n-3}\frac{(-1)^{k}}{k!}\dbinom{n-1}{k}P_{k+1}(x)\\
 & =-nx\frac{(-1)^{n}}{n!}P_{n}(x)+\frac{(-1)^{n-1}}{(n-1)!}[n^{2}x+(2n-1)x-1]P_{n-1}(x)\\
 & -\sum_{k=0}^{n-2}\frac{(-1)^{k}}{k!}\dbinom{n-1}{k}P_{k}(x)+x\sum_{k=0}^{n-3}\frac{(-1)^{k}}{k!}\dbinom{n-1}{k}P_{k+1}(x)\\
 & =-nx\frac{(-1)^{n}}{n!}P_{n}(x)+\frac{(-1)^{n-1}}{(n-1)!}[n^{2}x+(2n-1)x-1]P_{n-1}(x)\\
 & -x\sum_{k=1}^{n-2}\frac{(-1)^{k}}{(k-1)!}\left(\frac{(n-1)!}{(k-1)!(n-k)!}\right)P_{k}(x)-\sum_{k=0}^{n-2}\frac{(-1)^{k}}{k!}\dbinom{n-1}{k}P_{k}(x)\\
 & =-nx\frac{(-1)^{n}}{n!}P_{n}(x)+\frac{(-1)^{n-1}}{(n-1)!}[n^{2}x+(2n-1)x-1]P_{n-1}(x)\\
 & +\sum_{k=0}^{n-2}\frac{(-1)^{k}}{k!}\dbinom{n-1}{k}\left(\frac{1}{n-k}\right)\left[-k^{2}x-(n-k)\right]P_{k}(x)\\
 & =-nx\frac{(-1)^{n}}{n!}P_{n}(x)+\frac{(-1)^{n-1}}{(n-1)!}[n^{2}x+(2n-1)x-1]P_{n-1}(x)\\
 & +\sum_{k=0}^{n-2}\frac{(-1)^{k}}{k!}\dbinom{n-1}{k}\left[-nx\left(\frac{k}{n-k}\right)+kx-1\right]P_{k}(x)\\
 & =-nx\frac{(-1)^{n}}{n!}P_{n}(x)+\frac{(-1)^{n-1}}{(n-1)!}[n^{2}x+(2n-1)x-1]P_{n-1}(x)\\
 & +\sum_{k=0}^{n-2}\frac{(-1)^{k}}{k!}\dbinom{n-1}{k}\left[-nx\left(\frac{n-k+k}{n-k}\right)+[(2n-1)x-1]-(n-k-1)x\right]P_{k}(x)\\
 & =-nx\frac{(-1)^{n}}{n!}P_{n}(x)+\frac{(-1)^{n-1}}{(n-1)!}[n^{2}x+(2n-1)x-1]P_{n-1}(x)\\
 & +\sum_{k=0}^{n-2}\frac{(-1)^{k}}{k!}\dbinom{n-1}{k}\left[-nx\left(\frac{n}{n-k}\right)+[(2n-1)x-1]-(n-k-1)x\right]P_{k}(x)
\end{align*}
Continuing 
\begin{align*}
 & =-nx\left[\frac{(-1)^{n}}{n!}P_{n}(x)-\frac{(-1)^{n-1}}{(n-1)!}nP_{n-1}(x)\right]+[(2n-1)x-1]\frac{(-1)^{n-1}}{(n-1)!}P_{n-1}(x)\\
 & +\sum_{k=0}^{n-2}\frac{(-1)^{k}}{k!}\left[-nx\dbinom{n}{k}+[(2n-1)x-1]\dbinom{n-1}{k}-(n-1)x\dbinom{n-2}{k}\right]P_{k}(x)\\
 & =-nx\sum_{k=0}^{n}\frac{(-1)^{k}}{k!}\dbinom{n}{k}P_{k}(x)+[(2n-1)x-1]\sum_{k=0}^{n-1}\frac{(-1)^{k}}{k!}\dbinom{n-1}{k}P_{k}(x)\\
 & -(n-1)x\sum_{k=0}^{n-2}\frac{(-1)^{k}}{k!}\dbinom{n-2}{k}P_{k}(x)\\
 & =-nxv_{n}(x)+[(2n-1)x-1]v_{n-1}(x)-(n-1)xv_{n-2}(x).
\end{align*}
Rearranging yields 
\[
(1-x^{2})v_{n}^{\prime}(x)+nxv_{n}(x)=[(2n-1)x-1]v_{n-1}(x)-(n-1)xv_{n-2}(x)+(1-x^{2})v_{n-1}^{\prime}(x).
\]
\end{solution}
%%%%
%%
%%
%%%%
%%%%
%%
%%
%%%%
\begin{problem}\label{problem9chapter14}
Let

$$\gamma(k,n)= \dfrac{(-1)^k (\frac{1}{2})_{n-k} (2x)^{n-2k}}{k! (n-2k)!},$$

so that the Legendre polynomial of Chapter 10 may be written

$$P_n(x) = \displaystyle\sum_{k=0}^{\infty} \gamma(k,n).$$

Show that

$$\begin{array}{ll}
xP_{n-1}(x)=\displaystyle\sum_{k=0}^{\infty} \dfrac{(n-2k)\gamma(k,n)}{2n-2k-1}, & P_{n-2}(x) = \displaystyle\sum_{k=0}^{\infty} \dfrac{-2k \gamma(k,n)}{2n-2k-1}, \\
xP_n'(x) = \displaystyle\sum_{k=0}^{\infty} (n-2k)\gamma(k,n), & P_{n+1}'(x) = \displaystyle\sum_{k=0}^{\infty} (1+2n-2k) \gamma(k,n), \\
P_{n-1}'(x) = \displaystyle\sum_{k=0}^{\infty} -2k\gamma(k,n), & etc.
\end{array}$$

Use Sister Celine's method to discover the various differential recurrence relations and the pure recurrence relation for $P_n(x)$.
\end{problem}
\begin{solution}
Let $\gamma(x,n) = \dfrac{(-1)^k (\frac{1}{2})_{n-k} (2x)^{n-2k}}{k! (n-k)!}.$

Then

$$P_n(x) = \displaystyle\sum_{k=0}^{\infty} \gamma(k,n)$$

with our usual conventions. Also

$$x P_{n-1}(x) = \displaystyle\sum_{k=0}^{\infty} \dfrac{(-1)^k (\frac{1}{2})_{n-1-k}(2x)^{n-2k}}{2 \cdot k! (n-1-2k)!} = \displaystyle\sum_{k=0}^{\infty} \dfrac{n-2k}{2(n-k-\frac{1}{2}} \gamma(k,n)$$

or

$$x P_{n-1}(x) = \displaystyle\sum_{k=0}^{\infty} \dfrac{n-2k}{2n-2k-1} \gamma(k,n).$$

Next

$$\begin{array}{ll}
P_{n-2}(x) &= \displaystyle\sum_{k=0}^{\infty} \dfrac{(-1)^k (\frac{1}{2})_{n-2-k} (2x)^{n-2-2k}}{k! (n-2-2k)!} \\
&= \displaystyle\sum_{k=1}^{\infty} \dfrac{(-1)^{k-1} (\frac{1}{2})_{n-1-k} (2x)^{n-2k}}{(k-1)! (n-2k)!},
\end{array}$$

so that 

$$P_{n-2}(x) = \displaystyle\sum_{k=0}^{\infty} \dfrac{-k}{n-k-\frac{1}{2}} \gamma(k,n) = \displaystyle\sum_{k=0}^{\infty} \dfrac{-2k}{2n-2k-1} \gamma(k,n).$$

Now

$$x P_n'(x) = \displaystyle\sum_{k=0}^{\infty} \dfrac{(-1)^k (\frac{1}{2})_{n-k} (2x)^{n-2k}}{k! (n-1-2k)!} = \displaystyle\sum_{k=0}^{\infty} (n-2k)\gamma(k,n),$$

and

$$P_{n+1}'(x) = \displaystyle\sum_{k=0}^{\infty} \dfrac{(-1)^k (\frac{1}{2})_{n+1-k} \cdot 2 (2x)^{n-2k}}{k! (n-2k)!} = \displaystyle\sum_{k=0}^{\infty} (2n-2k+1) \gamma(k,n).$$

Finally,

$$\begin{array}{ll}
P_{n-1}'(x) &= \displaystyle\sum_{k=0}^{\infty} \dfrac{9-1)^k (\frac{1}{2})_{n-1-k} 2 (2x)^{n-2-2k}}{k! (n-2-2k)!} \\
&= \displaystyle\sum_{k=1}^{\infty} \dfrac{(-1)^{k-1} (\frac{1}{2})_{N-k} 2 (2x)^{n-2k}}{(k-1)! (n-2k)!} \\
&= \displaystyle\sum_{k=0}^{\infty} -2k \gamma(k,n).
\end{array}$$
\end{solution}
%%%%
%%
%%
%%%%
%%%%
%%
%%
%%%%
\begin{problem}\label{problem10chapter14}
Apply sister Celine's method to discover relations satisfied by the Hermite polynomials of Chapter 11.
\end{problem}
\begin{solution}

\end{solution}
%%%%
%%
%%
%%%%
%%%%
%%
%%
%%%%
\begin{problem}\label{problem11chapter14}
Find the various relations of Section 114 on Laguerre polynomials by using Sister Celine's technique.
\end{problem}
\begin{solution}

\end{solution}
%%%%
%%
%%
%%%%
%%%%
%%
%%
%%%%
\begin{problem}\label{problem12chapter14}
Consider the pseudo-Laguerre polynomials (Boas and Buck [2;16]) $f_n(x)$ defined for nonintegral $\lambda$ by

$$f_n(x) = \dfrac{(-\lambda)_n}{n!} {}_1F_1(-n;1+\lambda-n;x) = \displaystyle\sum_{k=0}^n \dfrac{(-\lambda)_{n-k} x^k}{k! (n-k)!}.$$

Show that the polynomials $f_n(x)$ are not orthogonal with respect to any weight function over any interval because no relation of the form

$$f_n(x) = (A_n + B_n x) f_{n-1}(x) + C_n f_{n-2}(x)$$

is possible. Obtain the pure recurrence relation

$$nf_n(x) = (x+n-1-\lambda)f_{n-1}(x) - xf_{n-2}(x).$$
\end{problem}
\begin{solution}
Consider $f_n(x)$ defined by

$$f_n(x) = \dfrac{(-\lambda)_n}{n!} {}_1F_1(-n; 1+\lambda-n;x) = \displaystyle\sum_{k=0}^n \dfrac{(-\lambda)_{n-k}x^k}{k!(n-k)!}.$$

Put

$$\epsilon(k,n) = \dfrac{(-\lambda)_{n-k} x^k}{k! (n-k)!}.$$

Then

$$f_n(x) = \displaystyle\sum_{k=0}^{\infty} \epsilon(k,n)$$

$$f_{n-1}(x) = \displaystyle\sum_{k=0}^{\infty} \dfrac{(-\lambda)_{n-1-k} x^k}{k! (n-1-k)!} = \displaystyle\sum_{k=0}^{\infty} \dfrac{n-k}{\lambda+n-1-k} \epsilon(k,n)$$

$$f_{n-2}(x) = \displaystyle\sum_{k=0}^{\infty} \dfrac{(n-k)(n-k-1)}{(-\lambda + n-1-k)(-\lambda+n-2-k)}\epsilon(k,n)$$

and

$$x f_{n-1}(x) = \displaystyle\sum_{k=0}^{\infty} \dfrac{(-\lambda)_{n-1-k}x^{k+1}}{k!(n-1-k)!} = \displaystyle\sum_{k=1}^{\infty} \dfrac{(-\lambda)_{n-k}x^k}{(k-1)!(n-k)!} = \displaystyle\sum_{k=0}^{\infty} k \epsilon(k,n)$$

$$x f_{n-2}(x) = \displaystyle\sum_{k=0}^{\infty} \dfrac{(-\lambda)_{n-2-k} x^{k+1}}{k!(n-2-k)!} = \displaystyle\sum_{k=0}^{\infty} \dfrac{(-\lambda)_{n-1-k} x^k}{(k-1)! (n-1-k)!} = \displaystyle\sum_{k=0}^{\infty} \dfrac{k(n-k)}{-\lambda + n-1-k} \epsilon(k,n).$$

Now consider

$$T_n \equiv f_n(x) - A_n f_{n-1}9x) - B_nx f_{N-1}(x) - C_n f_{n-2}(x).$$

At once

$$T_n \equiv \displaystyle\sum_{k=0}^{\infty} \left[ 1 - A_n \dfrac{n-k}{-\lambda+n-1-k} - B_nk - C_n \dfrac{(n-k)(n-k-1)}{(-\lambda+n-1-k)(-\lambda+n-2-k)} \right] \epsilon(k,n).$$

If $T_n \equiv 0$, then the following must be an identity in $k$ with $\lambda$ independent of $n$:

\begin{eqnarray*}
\lefteqn{(1) \hspace{35pt}(-\lambda+n-1-k)(-\lambda+n-2-k)+A_n(n-k)(-\lambda+n-2-k) } \\
&& - B_nk(-\lambda+n-1-k)(\lambda+n-2-k)-C_n(n-k)(n-k-1) \equiv 0.
\end{eqnarray*}

From the coefficient of $k^3$ we see that $B_n=0$. Then $T_n \equiv 0$ is impossible because $f_{n-1}(x)$ and $f_{n-2}(x)$ are of lower degree than $f_n(x).$

Now consider the identity

$$f_n(x) - D_n x f_{n-1}(x) - E_n f_{n-1}(x) - F_nxf_{n-2}(x) =0.$$

We need to determine $D_n, E_n, F_n$ to satisfy

$$1 - D_nk - E_n \dfrac{n-k}{-\lambda +n-1-k} - F_n \dfrac{k(n-k)}{-\lambda+n-1-k} =0,$$

or

$$(2) \hspace{30pt} -\lambda+n-1-k - D_n k(-\lambda+n-1-k) -E_n(n-k)-F_nk(n-k)\equiv 0.$$

$$\begin{array}{ll}
k=n: & -\lambda-1 - D_n n(-\lambda-1) = 0; D_n = \dfrac{1}{n} \mathrm{\hspace{3pt} or \hspace{3pt}} \lambda=-1 \\
\mathrm{coeff \hspace{3pt}} k^2: & D_n+F_n=0; F_n = -\dfrac{1}{n} \\
k=-\lambda+n-1: & (-\lambda-1)E_n + (\lambda-1)F_n(-\lambda+n-1)=0; E_n = \dfrac{-\lambda+n-1}{n}.
\end{array}$$

Hence, either $\lambda=-1$ or

$$nf_n(x) - xf_{n-1}(x) -(-\lambda+n-1)f_{n-1}(x) + xf_{n-2}(x) = 0,$$

or

$$nf_n(x) = [x+n-1-\lambda]f_{n-1}(x) - xf_{n-2}(x) =0,$$

as desired.
\end{solution}
%%%%
%%
%%
%%%%
%%%%
%%
%%
%%%%
For the polynomials in each of the following examples, use Sister Celine's technique to discover the pure recurrence relation and whatever mixed relations exist.

\begin{problem}\label{problem13chapter14}
The Bessel polynomials of Section 150.
\end{problem}
\begin{solution}

\end{solution}
%%%%
%%
%%
%%%%
%%%%
%%
%%
%%%%
\begin{problem}\label{problem14chapter14}
Bendient's polynomials $R_n$ of Section 151.
\end{problem}
\begin{solution}

\end{solution}
%%%%
%%
%%
%%%%
\begin{problem}\label{problem15chapter14}
Bendient's polynomials $G_n$ of Section 151.
\end{problem}
\begin{solution}

\end{solution}
%%%%
%%
%%
%%%%
\begin{problem}\label{problem16chapter14}
Shively's polynomials $R_n$ of Section 152.
\end{problem}
\begin{solution}

\end{solution}
%%%%
%%
%%
%%%%
\begin{problem}\label{problem17chapter14}
Shively's polynomials $\sigma_n$ of Section 152.
\end{problem}
\begin{solution}

\end{solution}