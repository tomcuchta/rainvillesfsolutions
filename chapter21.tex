%%%%
%%
%%
%%%%
%%%% CHAPTER 21
%%%% CHAPTER 21
%%%%
%%
%%
%%%%
\section{Chapter 21 Solutions}
\begin{center}\hyperref[toc]{\^{}\^{}}\end{center}
\begin{center}\begin{tabular}{lllllllllllllllllllllllll}
\hyperref[problem1chapter21]{P1} & \hyperref[problem2chapter21]{P2} & \hyperref[problem3chapter21]{P3} & \hyperref[problem4chapter21]{P4} & \hyperref[problem5chapter21]{P5} & \hyperref[problem6chapter21]{P6} 
\end{tabular}\end{center}
\setcounter{problem}{0}
\setcounter{solution}{0}
\begin{problem}\label{problem1chapter21}
Derive the preceding addition theorem $(7)$ by the method of Section 179. You may use results from the exercises at the end of Chapter 20.
\end{problem}
\begin{solution}
We seek an addition theorem for $\mathrm{cn}(u)$.
Now
$$\mathrm{cn}(u) = \dfrac{\theta_4}{\theta_2} \dfrac{\theta_2(u\theta_3^{-2})}{\theta_4(u\theta_3^{-2})}.$$
At the end of Exercise~5, Chapter~20 we obtained
$$\theta_4^2 \theta_4(x+4) \theta_4(x-y) = \theta_4^2(x) \theta_4^2(y) - \theta_1^2(x) \theta_1^2(y).$$
The next to last equation in Exercise~10, Chapter 20, is 
$$\theta_2 \theta_4 \theta_2(x+y) \theta_4(x-y) = \theta_2(x) \theta_4(x) \theta_2(y) \theta_4(y) - \theta_1(x) \theta_3(x) \theta_1(y) \theta_3(y).$$
We divide the last equation above by the one preceding it to get
$$\begin{array}{ll}
\dfrac{\theta_2(x+y)}{\theta_4(x+y)} &= \dfrac{\theta_2(x) \theta_4(x) \theta_2(y) \theta_4(y) - \theta_1(x) \theta_3(x) \theta_1(y) \theta_3(y)}{\theta_4^2(x) \theta_4^2(y) - \theta_1^2(x) \theta_1^2(y)} \\
&=\dfrac{\frac{\theta_2(x)}{\theta_4(x)} \frac{\theta_2(y)}{\theta_4(y)} - \frac{\theta_1(x)}{\theta_4(x)} \frac{\theta_3(x)}{\theta_4(x)} \frac{\theta_1(y)}{\theta_4(y)} \frac{\theta_3(y)}{\theta_4(y)}}{1 - \frac{\theta_1^2(x)}{\theta_4^2(x)} \frac{\theta_2^2(y)}{\theta_4^2(y)}}.
\end{array}$$
Now use $x = u\theta_3^{-2}$ and $y = v\theta_3^{-2}$ to obtain
$$\dfrac{\theta_2}{\theta_4} \dfrac{\theta_2}{\theta_4} \mathrm{cn}(u+v) = \dfrac{\frac{\theta_2}{\theta_4} \mathrm{cn}(u) \frac{\theta_2}{\theta_4} \mathrm{cn}(v) - \frac{\theta_2}{\theta_3} \mathrm{\mathrm{sn}}(u) \frac{\theta_3}{\theta_4} \mathrm{dn}(u) \frac{\theta_2}{\theta_3} \mathrm{\mathrm{sn}}(v)\frac{\theta_3}{\theta_4} \mathrm{dn}(v)}{1 -\ \frac{\theta_2^2}{\theta_3^2} \mathrm{\mathrm{sn}}^2(u) \frac{\theta_2^2}{\theta_3^2}\mathrm{\mathrm{sn}}^2(v)}$$
or
$$\mathrm{cn}(u+v) = \dfrac{\mathrm{cn}(u)\mathrm{cn}(v) - \mathrm{\mathrm{sn}}(u)\mathrm{dn}(u) \mathrm{\mathrm{sn}}(v) \mathrm{dn}(v)}{1 - k^2 \mathrm{\mathrm{sn}}^2(u) \mathrm{\mathrm{sn}}^2(v)},$$
as desired.
\end{solution}
%%%%
%%
%%
%%%%
\begin{problem}\label{problem2chapter21}
Derive preceding $(8)$ by the method of Section 179, with the aid of results from the exercises at the end of Chapter 20.
\end{problem}
\begin{solution}
We know that 
$$\mathrm{dn}(u) = \dfrac{\theta_4}{\theta_3} \dfrac{\theta_3(u\theta_3^{-2})}{\theta_4(u\theta_3^{-2})}, k = \dfrac{\theta_2}{\theta_3^2}.$$
From the last equation of Exercise~10, Chapter 20 we get
$$\theta_3 \theta_4 \theta_3(x+y) \theta_4(x-y) = \theta_3(x) \theta_4(x) \theta_3(y) \theta_4(y) - \theta_1(x) \theta_2(x) \theta_1(y) \theta_2(y)$$
and from the last equation of Exercise~8 Chapter 20 we get
$$\theta_4^2 \theta_4(x+y)\theta_4(x-y) = \theta_4^2(x) \theta_4^2(y) - \theta_1^2(x) \theta_1^2(y).$$
The two equations about at once yield
$$\begin{array}{ll}
\dfrac{\theta_3}{\theta_4} \dfrac{\theta_3}{\theta_4} \mathrm{dn}(u+v) &= \dfrac{\frac{\theta_3}{\theta_4} \mathrm{dn}(u) \frac{\theta_3}{\theta_4} \mathrm{dn}(v) - \frac{\theta_2}{\theta_3} \mathrm{\mathrm{sn}}(u)) \frac{\theta_2}{\theta_4} \mathrm{cn}(u) \frac{\theta_2}{\theta_3} \mathrm{\mathrm{sn}}(v) \frac{\theta_2}{\theta_4} \mathrm{\mathrm{sn}}(v)}{1 - \frac{\theta_2^2}{\theta_3^2} \mathrm{\mathrm{sn}}^2(u) \frac{\theta_2^2}{\theta_3^2} \mathrm{\mathrm{sn}}^2(v)}.
\end{array}$$
Hence we arrive at the desired result,
$$\mathrm{dn}(u+v) = \dfrac{\mathrm{dn}(u) \mathrm{dn}(v) - k^2 \mathrm{\mathrm{sn}}(u) \mathrm{cn}(u) \mathrm{\mathrm{sn}}(v) \mathrm{cn}(v)}{1 - k^2 \mathrm{\mathrm{sn}}^2(u) \mathrm{\mathrm{sn}}^2(v)}.$$
\end{solution}
%%%%
%%
%%
%%%%
\begin{problem}\label{problem3chapter21}
Show that $\displaystyle\int \mathrm{cn}^3(x) \mathrm{dn}(x) dx = \mathrm{\mathrm{sn}}(x) - \dfrac{1}{3} \mathrm{\mathrm{sn}}^3(x) + c.$
\end{problem}
\begin{solution}
We wish to evaluate 
$$\displaystyle\int \mathrm{cn}^3(x) \mathrm{dn}(x) dx.$$
We know that
$$\dfrac{\mathrm{d}}{\mathrm{dx}} \mathrm{dn}(x) = \mathrm{cn}(x) \mathrm{dn}(x)$$
and that
$$\mathrm{cn}^2(x) =1-\mathrm{\mathrm{sn}}^2(x).$$
Hence
$$\begin{array}{ll}
\displaystyle\int \mathrm{cn}^3(x) \mathrm{dn}(x) dx &= \displaystyle\int [1-\mathrm{\mathrm{sn}}^2(x)] \mathrm{cn}(x) \mathrm{dn}(x) dx \\
&= \mathrm{\mathrm{sn}}(x) - \dfrac{1}{3} \mathrm{\mathrm{sn}}^2(x) +c.
\end{array}$$
\end{solution}
%%%%
%%
%%
%%%%
\begin{problem}\label{problem4chapter21}
Show that if $g(x)$ and $h(x)$ are any two different ones of the three functions $\mathrm{\mathrm{sn}}(x), \mathrm{cn}(x), \mathrm{dn}(x),$ and if $m$ is a non-negative integer, you can perform the integration
$$\displaystyle\int g^{2m+1}(x) h(x) dx.$$
\end{problem}
\begin{solution}
Consider
$$A = \displaystyle\int g^{2m+1}(x) h(x) dx.$$
We know that
$$\dfrac{\mathrm{d}}{\mathrm{dx}} \mathrm{\mathrm{sn}}(x) = \mathrm{cn}(x)\mathrm{dn}(x);$$
$$\dfrac{\mathrm{d}}{\mathrm{dx}} \mathrm{cn}(x) = -\mathrm{\mathrm{sn}}(x) \mathrm{dn}(x);$$
$$\dfrac{\mathrm{d}}{\mathrm{dx}} \mathrm{dn}(x) = -k^2 \mathrm{\mathrm{sn}}(x) \mathrm{cn}(x).$$
Now let $g(x)$ and $h(x)$ be any two different ones of $\mathrm{\mathrm{sn}}(x), \mathrm{cn}(x), \mathrm{dn}(x)$. Let $\psi(x)$ be the other of the three functions. Then
$$g(x) h(x) dx = c_1 d[\psi(x)],$$
in which $c_1=1, -1$ or $-K^{-2}$ according to whether $\psi(x)$ is $\mathrm{\mathrm{sn}}(x), \mathrm{cn}(x),$ or $\mathrm{dn}(x).$
Now we also know that
$$\mathrm{\mathrm{sn}}^2(x) + \mathrm{cn}^2(x) = 1,$$
$$k^2 \mathrm{\mathrm{sn}}^2(x) + \mathrm{dn}^2(x)=1,$$
$$\mathrm{dn}^2(x) - k^2 \mathrm{cn}^2(x) = x^2.$$
Therefore the square of any one of $\mathrm{\mathrm{sn}}(x), \mathrm{cn}(x), \mathrm{dn}(x)$ is a linear function of the square of any other one of them. Hence we may write
$$g^2(x) = a_1+b_1 \psi^2(x).$$
We know have
$$\begin{array}{ll}
\displaystyle\int g^{2m+1}(x) h(x) dx &= \displaystyle\int [g^2(x)]^m g(x) h(x) dx \\
&= c_1 \displaystyle\int [a_1 + b_1 \psi^2(x)]^m d \psi(x),
\end{array}$$
which is integrable as a sum of powers.
\end{solution}
%%%%
%%
%%
%%%%
\begin{problem}\label{problem5chapter21}
Obtain the result 
$$\displaystyle\int \mathrm{\mathrm{sn}}(x) dx = \dfrac{1}{k} \log[\mathrm{dn}(x) - k \mathrm{cn}(x)] + c.$$
\end{problem}
\begin{solution}
Consider $\displaystyle\int \mathrm{\mathrm{sn}}(x) dx$. We may write
$$\begin{array}{ll}
\displaystyle\int \mathrm{\mathrm{sn}}(x) dx &= \displaystyle\int \dfrac{\mathrm{\mathrm{sn}}(x)[\mathrm{dn}(x)-k\mathrm{cn}(x)]}{\mathrm{dn}(x)-k\mathrm{cn}(x)} dx \\ 
&= \displaystyle\int \dfrac{\mathrm{\mathrm{sn}}(x) \mathrm{dn}(x) - k\mathrm{\mathrm{sn}}(x) \mathrm{cn}(x)}{\mathrm{dn}(x) - k\mathrm{cn}(x)} dx \\
&= \displaystyle\int \dfrac{-d[\mathrm{dn}(x)] + \frac{1}{k} d[\mathrm{dn}(x)]}{\mathrm{dn}(x) - k\mathrm{cn}(x)} \\
&= \dfrac{1}{k} \displaystyle\int \dfrac{d[\mathrm{dn}(x)]-k[\mathrm{cn}(x)]}{\mathrm{dn}(x) -k \mathrm{cn}(x)} \\
&= \dfrac{1}{k} \log[\mathrm{dn}(x) -k\mathrm{cn}(x)]+c.
\end{array}$$
\end{solution}
%%%%
%%
%%
%%%%
%%%%
%%
%%
%%%%
\begin{problem}\label{problem6chapter21}
Show that
$$\mathrm{dn}(2x) - k \mathrm{cn}(2x) = \dfrac{(1-k)[1+k \mathrm{\mathrm{sn}}^2(x)]}{1-k \mathrm{\mathrm{sn}}^2(x)}.$$
\end{problem}
\begin{solution}
We wish to show that
$$\mathrm{dn}(2x) - k \mathrm{cn}(2x) = \dfrac{(1-k) [1+k \mathrm{\mathrm{sn}}^2(x)]}{1-k \mathrm{\mathrm{sn}}^2(x)}.$$
Now from the addition formulas of Exercises~\ref{problem1chapter21} and \ref{problem2chapter21} we obtain
$$\mathrm{dn}(2x) = \dfrac{\mathrm{dn}^2(x) - k^2 \mathrm{\mathrm{sn}}^2(x) \mathrm{dn}^2(x)}{1 - k^2 \mathrm{\mathrm{sn}}^4(x)}$$
and
$$\mathrm{cn}(2x) = \dfrac{\mathrm{cn}^2(x) - \mathrm{\mathrm{sn}}^2(x) \mathrm{dn}^2(x)}{1 - k^2 \mathrm{\mathrm{sn}}^4(x)}.$$
Let us put each of the above in terms of $\mathrm{\mathrm{sn}}(x)$. We get
$$\mathrm{dn}(2x) = \dfrac{1-k^2 \mathrm{\mathrm{sn}}^2(x) - k^2 \mathrm{\mathrm{sn}}^2(x) [1-\mathrm{\mathrm{sn}}^2(x)]}{1 - k^2 \mathrm{\mathrm{sn}}^4(x)} = \dfrac{1 - 2k^2 \mathrm{\mathrm{sn}}^2(x) + k^2 \mathrm{\mathrm{sn}}^4(x)}{1 - k^2 \mathrm{\mathrm{sn}}^4(x)}.$$
$$\mathrm{cn}(2x) = \dfrac{1 - \mathrm{\mathrm{sn}}^2(x) - \mathrm{\mathrm{sn}}^2(x) [1-x^2 \mathrm{\mathrm{sn}}^2(x)]}{1 -k^2\mathrm{\mathrm{sn}}^4(x)} = \dfrac{1-2k^2\mathrm{\mathrm{sn}}^2(x) + k^2\mathrm{\mathrm{sn}}^4(x)}{1-k^2 \mathrm{\mathrm{sn}}^4(x)}$$
Then
$$\begin{array}{ll}
\mathrm{dn}(2x) - k \mathrm{cn}(2x) &= \dfrac{1 - k^2 \mathrm{\mathrm{sn}}^2(x) - k^2 \mathrm{\mathrm{sn}}^2(x) + k^2\mathrm{\mathrm{sn}}^4(x) - k + k \mathrm{\mathrm{sn}}^2(x) + k \mathrm{\mathrm{sn}}^2(k) - k^3 \mathrm{\mathrm{sn}}^4(x)}{1 - k^2 \mathrm{\mathrm{sn}}^4(x)} \\
&= \dfrac{1 + k\mathrm{\mathrm{sn}}^2(x) + k \mathrm{\mathrm{sn}}^2(x)[1+k \mathrm{\mathrm{sn}}^2(x)] - k[1+k \mathrm{\mathrm{sn}}^2(x)] - k^2 \mathrm{\mathrm{sn}}^2(x) [1+k\mathrm{\mathrm{sn}}^2(x)]}{[1-k\mathrm{\mathrm{sn}}^2(x)][1+k\mathrm{\mathrm{sn}}^2(x)]} \\
&= \dfrac{1 + k \mathrm{\mathrm{sn}}^2(x) - k - k^2 \mathrm{\mathrm{sn}}^2(x)}{1 - k \mathrm{\mathrm{sn}}^2(x)} \\
&= \dfrac{1 - k + k (1-k)\mathrm{\mathrm{sn}}^2(x)}{1-k\mathrm{\mathrm{sn}}^2(x)}.
\end{array}$$
Hence
$$\mathrm{dn}(2x)-k\mathrm{cn}(2x) = \dfrac{(1-k)[1+k\mathrm{\mathrm{sn}}^2(x)]}{1-k\mathrm{\mathrm{sn}}^2(x)}.$$
\end{solution}
\end{document}
