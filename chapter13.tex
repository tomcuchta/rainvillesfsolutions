%%%%
%%
%%
%%%%
%%%% CHAPTER 13
%%%% CHAPTER 13
%%%%
%%
%%
%%%%
\section{Chapter 13 Solutions}
\begin{center}\hyperref[toc]{\^{}\^{}}\end{center}
\begin{center}\begin{tabular}{lllllllllllllllllllllllll}
\hyperref[problem1chapter13]{P1} & \hyperref[problem2chapter13]{P2} & \hyperref[problem3chapter13]{P3} & \hyperref[problem4chapter13]{P4} & \hyperref[problem5chapter13]{P5} & \hyperref[problem6chapter13]{P6} & \hyperref[problem7chapter13]{P7} & \hyperref[problem8chapter13]{P8} & \hyperref[problem9chapter13]{P9} & \hyperref[problem10chapter13]{P10} & \hyperref[problem11chapter13]{P11} & \hyperref[problem12chapter13]{P12}
\end{tabular}\end{center}
\setcounter{problem}{0}
\setcounter{solution}{0}
\begin{problem}\label{problem1chapter13}
Prove Theorem $A$ (Sheffer): If $\phi_n(x)$ is of Sheffer $A$-type zero, $g_n(m,x)=D^m \phi_{n+m}(x)$ is also of Sheffer $A$-type zero and belongs to the same operator as does $\phi_n(x)$.
\end{problem}
\begin{solution}
Let $\phi_n(x)$ be of Sheffer $A$-type zero. Put

$$g_n(m,x) = \mathscr{D}^m \phi_{n+m}(x); \mathscr{D} \equiv \dfrac{d}{dx}.$$

We know there exists an $A, \psi,$ and $H$ such that

$$\begin{array}{ll}
(1) \hspace{30pt} A(t) \psi(x H(t)) &= \displaystyle\sum_{n=0}^{\infty} \phi_n(x) t^n \\
&= \displaystyle\sum_{n=-m}^{\infty} \phi_{n+m}(x)t^{n+m}.
\end{array}$$

Now the $\phi_n(x)$ form a simple set. Hence $\mathscr{D}^m\phi_k(x) =0$ for $k<m$. Therefore, using the operator $\mathscr{D}^m$ on $(1)$ we obtain

$$A(t) [H(t)]^m \psi^{(m)}(xH(t)) = \displaystyle\sum_{n=0}^{\infty} \mathscr{D}^m \phi_{n+m}(x) t^{n+m}.$$

Since $H(0)=0,$ we may write

$$(2) \hspace{30pt} A(t) \left[ \dfrac{H(t)}{t} \right]^m \psi^{(m)} (xH(t)) = \displaystyle\sum_{n=0}^{\infty} \mathscr{D}^m \phi_{n+m}(x) t^n,$$

from which it follows at once (Theorem~72) that $\mathscr{D}^m\phi_{n+m}(x)$ are polynomials of Sheffer $A$-type zero an that they belong to the same operator as do the $\phi_n(x)$, because the $H(t)$ in $(1)$ and $(2)$ are the same function.
\end{solution}
%%%%
%%
%%
%%%%
\begin{problem}\label{problem2chapter13}
Prove Theorem $B$: If $\phi_n(x)$ is of Sheffer $A$-type zero,

$$\psi_n(x) = \phi_n(x) \left[ \displaystyle\prod_{i=1}^m (1+\rho_i)_n \right]^{-1}$$

is of Sheffer $A$-type $m$.
\end{problem}
\begin{solution}
We are given that $\phi_n(x)$ is of Sheffer $A$-type zero and wish to consider

$$\phi_n(x) = \dfrac{\phi_n(x)}{\displaystyle\prod_{i=1}^m (1 + \rho_i)_n}.$$

Let $\phi_n(x)$ belong to the operator $J(\mathscr{D})$. By Theorem 74, there exists $\mu_k, v_k$ such that

$$\displaystyle\sum_{k=0}^{n-1} (\mu_k+x v_k) \mathscr{D}^{k+1} \phi_n(x) = n \phi_n(x).$$

Hence there exists operator

$$B= B(x, \mathscr{D}) = \displaystyle\sum_{k=0}^{\infty} (\mu_k + x v_k) \mathscr{D}^{k+1} = B_1(\mathscr{D}) + x B_2(\mathscr{D})$$

such that

$$B\phi_n(x) = n \phi_n(x).$$

Then

$$(B = \rho_i) \phi_n(x) = (n + \rho_i) \phi_N(x),$$

so that

$$\left\{ J(\mathscr{D}) \displaystyle\prod_{i=1}^m [B(x, \mathscr{D})+\rho_i] \right\} \phi_n(x) = J(\mathscr{D}) \dfrac{\phi_n(x)}{\displaystyle\prod_{i=1}^m (1 + \rho_i)_{n-1}} = \dfrac{\phi_{n-1}(x)}{\displaystyle\prod_{i=1}^m (1 + \rho_i)_{n-1}} = \psi_{n-1}(x).$$

Since $\psi_n(x)$ belongs to one operator whose coefficients have maximum degree $m$, $\psi_n(x)$ is of Sheffer $A$-type $m$ and $\psi_n(x)$ belongs to the operator

$$J_1(x, \mathscr{D}) = J(\mathscr{D}) \displaystyle\prod_{i=1}^m [\rho_i + B(x,\mathscr{D})].$$

Two operators arise here. Above implies commutativity of the operators $[\rho_i + B(x,\mathscr{D})]$ and it implies that $J_1$ is a proper operator, that is transforms every polynomial into one of degree one lower than the original.

Let us now prove the desired commutativity property. Let 

$$\varepsilon_i = \rho_i + \displaystyle\sum_{k=0}^{\infty} \mu_k \mathscr{D}^{k+1} + x \displaystyle\sum_{k=0}^{\infty} v_k \mathscr{D}^{k+1}.$$

We shall show that $\varepsilon_1 \varepsilon_2 = \varepsilon_2 \varepsilon_1.$ Since 

$$\mathscr{D}^{m+1}[\rho + \mu_k \mathscr{D}^{k+1} + x v_k \mathscr{D}^{k+1}] = \rho \mathscr{D}^{m+1} + \mu_k \mathscr{D}^{m+k+2} + x v_k \mathscr{D}^{m+k+2} + (m+1) v_k B^{m+k+1},$$

we obtain

$$\begin{array}{ll}
\varepsilon_1 \varepsilon_2 &= \left[ \rho_1 + \displaystyle\sum_{m=0}^{\infty} \mu_m \mathscr{D}^{m+1} + x \displaystyle\sum_{m=0}^{\infty} v_m \mathscr{D}^{m+1} \right] \left[ \rho_2 + \displaystyle\sum_{k=0}^{\infty} \mu_k \mathscr{D}^{k+1} + x \displaystyle\sum_{k=0}^{\infty} v_k \mathscr{D}^{k+1} \right] \\
&= \rho_1 \rho_2 + \rho_1 \displaystyle\sum_{k=0}^{\infty} \mu_k \mathscr{D}^{k+1} + \rho_1 x \displaystyle\sum_{k=0}^{\infty} v_k \mathscr{D}^{k+1} + \rho_2 \displaystyle\sum_{m=0}^{\infty} \mu_m \mathscr{D}^{m+1} + x \rho_2 \displaystyle\sum_{m=0}^{\infty} v_m \mathscr{D}^{m+1} \\
&\phantom{=}+ \left[ \displaystyle\sum_{m=0}^{\infty} (\mu_m + x v_m) \mathscr{D}^{m+1} \right] \left[ \displaystyle\sum_{k=0}^{\infty} (\mu_k + x v_k) \mathscr{D}^{k+1} \right],
\end{array}$$

which is symmetric in $\rho_1$ and $\rho_2$. Hence $\varepsilon_1 \varepsilon_2 = \varepsilon_2 \varepsilon_1.$

We shall now prove a theorem of value in discussing what operators transform polynomials of degree $n$ into polynomials of degree $(n-1)$.

$\mathbf{Theorem:}$ If the operator $J(x, \mathscr{D})$ is such that a simple set of polynomials $\psi_n(x)$ belongs to it in the Sheffer sense,

$$(1) \hspace{30pt} J(x,\mathscr{D}) \psi_n(x) = \psi_{n-1}(x), J(x,\mathscr{D}) \psi_0(x) = 0,$$

then $J(x ,\mathscr{D})$ transforms every polynomial of degree precisely $n$ into a polynomial of degree precisely $n-1$.

$\mathbf{Proof:}$ Let $g_n(x)$ be of degree exactly $n$. Then we know there exists the expansion

$$g_n(x)=\displaystyle\sum_{k=0}^n A(k,n) \psi_{n-k}(x), A(0,n) \neq 0,$$

merely because the $\psi_n(x)$ form a simple set. Then, because of $(1)$,

$$(2) \hspace{30pt} J(x,\mathscr{D}) g_n(x) = \displaystyle\sum_{k=0}^{n-1} A(k,n) \psi_{n-1-k}(x),$$

and the right member of $(2)$ is of degree exactly $(n-1).$

\end{solution}
%%%%
%%
%%
%%%%
\begin{problem}\label{problem3chapter13}
Show that
$$\dfrac{H_n(x)}{n!}$$
is of Sheffer $A$-type zero, obtain the associated functions $J(t), H(t), A(t)$, and draw what conclusions you can from Theorems 73-76.
\end{problem}
\begin{solution}(Solution by Leon Hall)
From the defining relation for the Hermite polynomials (p.187) we have
$$\begin{array}{ll}
\displaystyle\sum_{n=0}^{\infty} \dfrac{H_n(x)t^n}{n!} &= \exp(2xt-t^2) \\
&= \exp(-t^2)\exp(2xt).
\end{array}$$
Thus by Theorem~72, $\phi_n(x)$ is a Sheffer $A$-type zero with $A(t)=\exp(-t^2)$ and $H(t)=2t$. Because $J(x)$ is the inverse of $H(t)$ we get $J(t)=\dfrac{t}{2}$. 

The operator $J$ may be found directly as follows. Denote $\dfrac{H_n(x)}{n!}$ by $\phi_n(x)$. Then we need (see p.189 for the first few Hermite polynomials):
$$T_0(x)D \phi_1(x) = \phi_0(x)$$
$$T_0(x)D(2x)=1.$$
So,
$$T_0(x)=\dfrac{1}{2}.$$
Continuing,
$$T_1(x) D^2 \phi_2(x) = \phi_1(x) - T_0(x) D \phi_2(x)$$
$$4 T_1(x) = 2x - \dfrac{1}{2}4x = 0.$$
So, $T_1(x)=0$. In general we need
$$T_n(x)D^{n+1} \phi_{n+1}(x) = \phi_n(x) - T_0(x)D \phi_{n+1}(x) - \displaystyle\sum_{k=1}^{n-1} T_k(x) D^{k+1} \phi_{n+1}(x).$$
Because $H_n'(x) = 2nH_{n-1}(x)$, we have
$$\phi'(x)=\dfrac{2nH_{n-1}(x)}{n!} = 2\phi_{n-1}(x)$$
(also see Theorem~76) and so 
$$T_0(x) D \phi_{n+1}(x) = \phi_n(x)$$
and
$$D^{k+1} \phi_{n+1}(x) = 2^{k+1} \phi_{n-k}(x).$$
This makes, for $n \geq 2$,
$$\begin{array}{ll}
T_n(x) 2^{n+1} &= \phi_n(x) - \phi_n(x) - \displaystyle\sum_{k=1}^{n-1} 2^{k+1} T_k(x) \phi_{n-k}(x) \\
&= -\displaystyle\sum_{k=1}^{n-1} 2^{k+1} T_k(x) \phi_{n-k}(x).
\end{array}$$
Because we have $T_1(x)=0$, this makes $T_2(x)=0$, which means $T_3(x)=0$, and so on. Therefore, from Theorem~70,
$$J(x,D)=\displaystyle\sum_{k=0}^{\infty} T_k(x) D^{k+1} = \dfrac{1}{2}D.$$
From the formulas for $A(t)$ and $H(t)$ it is easily seen that, in Theorem~73,
$$a_k = \left\{ \begin{array}{ll}
0 &; k \neq 1 \\
-2 &; k=1 
\end{array} \right.$$
and
$$\epsilon_k = \left\{ \begin{array}{ll}
0 &; k=1,2,\ldots \\
2 &; k=0.
\end{array} \right.$$
This makes equation $(11)$ from Theorem~73, for $n \geq 2$,
$$(xD-\dfrac{1}{2}D^2)\phi_n(x) = n\phi_n(x)$$
or
$$\phi_n''(x)-2x\phi_n'(x)+2n\phi_n(x)=0,$$
which is Hermite's differential equation. Hermite's differential equation is also a result of Theorem~74 using $\mu_k=0$ except $\mu_1=-\dfrac{1}{2}$ and $v_0=1, v_k=0; k \geq 1$. Equation $(17)$ in Theorem~75, using the $\alpha_k$ and $\epsilon_k$ values we found before, becomes
$$2x \phi_{n-1}(x) - 2\phi_{n-2}(x) = n\phi_n(x)$$
and this can be written as
$$2x \dfrac{H_{n-1}(x)}{(n-1)!} - \dfrac{2H_{n-2}(x)}{(n-2)!} = \dfrac{H_n(x)}{(n-1)!}$$
or
$$\dfrac{1}{(n-1)!} \left[ 2x H_{n-1}(x) - 2(n-1)H_{n-2}(x)=H_n(x) \right],$$
where the equation in brackets is the pure recurrence relation for Hermite polynomials.
\end{solution}
%%%%
%%
%%
%%%%
\begin{problem}\label{problem4chapter13}
Show that $L_n^{(\alpha)}(x)$ is of Sheffer $A$-type zero, and proceed as in Exercise~\ref{problem3chapter13}.
\end{problem}
\begin{solution}(Solution by Leon Hall)
A generating function for the Laguerre polynomials (see $(4)$, p.202) is 
$$\dfrac{1}{(1-t)^{1+\alpha}} \exp \left( \dfrac{-xt}{1-t} \right) = \displaystyle\sum_{n=0}^{\infty} L_n^{(\alpha)}(x)t^n,$$
so Theorem~72 says $L_n^{(\alpha)}(x)$ is of Sheffer $A$-type zero with
$$A(t)=(1-t)^{-1-\alpha}$$
and
$$H(t)=\dfrac{-t}{1-t}.$$
An easy inverse calculation for $H(t)$ yields 
$$J(t) = \dfrac{-t}{1-t} = -\displaystyle\sum_{n=0}^{\infty} t^{n+1}.$$
From 
$$\dfrac{A'(t)}{A(t)} = \dfrac{1+\alpha}{1-t} = \displaystyle\sum_{k=0}^{\infty} \alpha_k t^k$$
we have $\alpha_k=1+\alpha$ for all $k$, and from
$$H'(t)=-\displaystyle\sum_{k=0}^{\infty} (k+1)t^k$$
we have $\epsilon_k=-(k+1)$. Thus, $(11)$ in Theorem~73 is
$$(*) \hspace{35pt} \displaystyle\sum_{k=0}^{n-1} (1+\alpha-x(k+1))J^{k+1}(D)L_n^{(\alpha)}(x) = nL_n^{(\alpha)}(x).$$
By definition $J(D)L_n^{(\alpha)}(x) = L_{n-1}^{(\alpha)}(x)$ so
$$\displaystyle\sum_{k=0}^{n-1} J^{k+1}(D) L_n^{(\alpha)}(x) = \displaystyle\sum_{k=0}^{n-1} L_{n-1-k}^{(\alpha)}(x) = \displaystyle\sum_{k=0}^{n-1} L_k^{(\alpha)}(x),$$
and so by Section~114, equation $(3)$, 
$$\displaystyle\sum_{k=0}^{n-1} L_k^{(\alpha)}(x) = -D L_n^{(\alpha)}(x)$$
(see Theorem~76). The left side of $(*)$ then becomes
\begin{eqnarray*}
\lefteqn{\displaystyle\sum_{k=0}^{n-1} (1+\alpha-x-kx)J^{k+1}(D)L_n^{(\alpha)}(x)} \\
& &= (1+\alpha-x) \displaystyle\sum_{k=0}^{n-1} J^{k+1}(D)L_n^{(\alpha)}(x) - x\displaystyle\sum_{k=0}^{n-1} kJ^{k+1}(D)L_n^{(\alpha)}(x) \\
& &=-(1+\alpha-x)DL_n^{(\alpha)}(x)-x\displaystyle\sum_{k=0}^{n-1} kJ^{k+1}(D)L_n^{(\alpha)}(x) \\
& &= -(1+\alpha-x)DL_n^{(\alpha)}(x)-x\displaystyle\sum_{k=1}^{n-1} kL_{n-1-k}^{(\alpha)}(x).
\end{eqnarray*}
Rewrite the last sum as
$$\begin{array}{ll}
\displaystyle\sum_{k=1}^{n-1} kL_{n-1-k}^{(\alpha)}(x) &= \displaystyle\sum_{k=0}^{n-2} L_k^{(\alpha)}(x) + \displaystyle\sum_{k=0}^{n-3} L_k^{(\alpha)}(x) + \ldots + \displaystyle\sum_{k=0}^1 L_k^{(\alpha)}(x) + L_0^{(\alpha)}(x) \\
&= -D L_{n-1}^{(\alpha)}(x) - DL_{n-2}^{(\alpha)}(x) - \ldots - DL_2^{(\alpha)}(x) - DL_1^{(\alpha)}(x) \\
&=-D\displaystyle\sum_{k=0}^{n-1} L_k^{(\alpha)}(x) \\
&= D^2 L_n^{(\alpha)}(x).
\end{array}$$
Thus, $(*)$ becomes, finally
$$-(1+\alpha-x)DL_n^{(\alpha)}(x) - xD^2 L_n^{(\alpha)}(x) = nL_n^{(\alpha)}(x)$$
or
$$xD^2 L_n^{(\alpha)}(x) + (1+\alpha-x) D L_n^{(\alpha)}(x) + nL_n^{(\alpha)}(x)=0,$$
which is Laguerre's differential equation. Equation $(17)$ in Theorem~75 becomes, after an index shift,
$$\displaystyle\sum_{k=1}^n (1+\alpha-kx)L_{n-k}^{(\alpha)}(x) = nL_n^{(\alpha)}(x).$$
In Theorem~74, with $u=J(t)=\dfrac{-t}{1-t}$, we get
$$\dfrac{uA'(u)}{A(u)} = -(1+\alpha)t$$
and
$$uH'(u)=t-t^2,$$
making $\mu_0=-(1+\alpha)$, $v_0=1, v_1=-1,$ and the rest of the $\mu_k$ and $v_k$ zero. Then, using these values in equation $(14)$ of Theorem~74 we again get Laguerre's differential equation.
\end{solution}
%%%%
%%
%%
%%%%
\begin{problem}\label{problem5chapter13}
Show that the Newtonian polynomials
$$N_n(x) = \dfrac{(-1)^n (-x)_n}{n!}$$
are of Sheffer $A$-type zero, and proceed as in Exercise~\ref{problem3chapter13}.
\end{problem}
\begin{solution}(Solution by Leon Hall)
A standard Taylor series expansion about $t=0$ shows that
$$\displaystyle\sum_{n=0}^{\infty} N_n(x)t^n = \displaystyle\sum_{n=0}^{\infty} \dfrac{(-1)^n (-x)_n}{n!} t^n = (1+t)^x=e^{x\log(1+t)},$$
so in Theorem~72 we have $A(t)=1$ and
$$H(t)=\log(1+t) = \displaystyle\sum_{k=0}^{\infty} \dfrac{(-1)^k}{k+1} t^{k+1}.$$
Because $J(t)$ is the inverse of $H(t)$,
$$J(t) = e^t-1 = \displaystyle\sum_{k=0}^{\infty} \dfrac{t^{n+1}}{(n+1)!}.$$
$A(t)$ constant means all the $\alpha_k=0$ in Theorems~73~and~75 and that all the $\mu_k=0$ in Theorem~74. Because
$$H'(t) = \dfrac{1}{1+t} = \displaystyle\sum_{k=0}^{\infty} (-1)^k t^k,$$
we have $\epsilon_k = (-1)^k$ in Theorems~73~and~75 and, for $u=J(t)$,
$$uH'(u)=1-e^{-t}=\displaystyle\sum_{k=0}^{\infty} \dfrac{(-1)^k}{(k+1)!} t^{k+1}$$
we get $v_k=\dfrac{(-1)^k}{(k+1)!}$ in Theorem~74. Note that also if $\Delta N_n(x) \equiv N_n(x+1)-N_n(x)$, then
$$\begin{array}{ll}
\Delta N_n(x) &= \dfrac{(-1)^n}{n!} (-(x+1))_n - \dfrac{(-1)^n}{n!} (-x)_n \\
&= \dfrac{(-1)^{n-1}}{(n-1)!} (-x)_{n-1} \left[ \dfrac{x+1}{n} - \dfrac{x+1-n}{n} \right] \\
&= N_{n-1}(x).
\end{array}$$
Theorem~73 now gives
$$\displaystyle\sum_{k=0}^{n-1} (-1)^k x J^{k+1}(D) N_n(x) = nN_n(x)$$
or using Theorem~75
$$x \displaystyle\sum_{k=0}^{n-1} (-1)^k N_{n-1-k}(x) = nN_n(x).$$
And because $\Delta^{k+1}N_n(x) = N_{n-1-k}(x),$
$$x\displaystyle\sum_{k=0}^{n-1} (-1)^k \Delta^{k+1} N_n(x) = nN_n(x).$$
Theorem~74 yields
$$x\displaystyle\sum_{k=0}^{n-1} \dfrac{(-1)^k}{(k+1)!} D^{k+1} N_n(x) = nN_n(x).$$
Finally, from Theorem~76, recalling from $H(t)$ that $h_k=\dfrac{(-1)^k}{k+1}$, we have
$$\displaystyle\sum_{k=0}^{n-1} \dfrac{(-1)^k}{k+1} N_{n-1-k}(x) = DN_n(x)$$
or
$$N_n'(x)=N_{n-1}(x)-\dfrac{1}{2} N_{n-2}(x)+\dfrac{1}{3}N_{n-3}(x) - \ldots + \dfrac{(-1)^{n-1}}{n} N_0(x).$$
With some work, this last result could also be obtained by logarithmic differentiation.
\end{solution}
%%%%
%%
%%
%%%%
\begin{problem}\label{problem6chapter13}
Show that

$$\phi_n(x) = \dfrac{L_n^{(\alpha)}(x)}{(1+\alpha)_n}$$

is of Sheffer $A$-type unity but that with $\sigma$ chosen to be

$$\sigma = D(\theta + \alpha)$$

the polynomials $\phi_n(x)$ are of $\sigma$-type zero.
\end{problem}
\begin{solution}
Consider

$$\phi_n(x) = \dfrac{\mathscr{L}_n^{(\alpha)}(x)}{(1+\alpha)_n}.$$

Since $\mathscr{L}_n^{(\alpha)}(x)$ is of Sheffer $A$-type zero (Exercise~\ref{problem4chapter13}), it follows by Exercise~\ref{problem2chapter13} that $\phi_n(x)$ is of Sheffer $A$-type one.

Now

$$\displaystyle\sum_{n=0}^{\infty} \dfrac{\mathscr{L}_n^{(\alpha)}(x) t^n}{(1+\alpha)-n} = e^t {}_0F_1(-;1+\alpha;-xt),$$

so that, by Theorem 79, $\phi_n(x)$ is of $\sigma$-type zero with

$$\sigma = \mathscr{D}(\theta + 1 + 1 -a) = \mathscr{D}(\theta + \alpha).$$
\end{solution}
%%%%
%%
%%
%%%%
\begin{problem}\label{problem7chapter13}
Determine the Sheffer operator associated wit hthe set

$$\phi_n(x) = \dfrac{L_n(x)}{(n!)^2},$$

and thus show that $\phi_n(x)$ is of Sheffer $A$-type $2$.
\end{problem}
\begin{solution}
Consdier $\phi_n(x) = \dfrac{\mathscr{L}_n(x)}{(n!)^2}$. Since $\mathscr{L}_n(x)$ is of Sheffer $A$-type zero (by Exercise~\ref{problem4chapter13}), we may use Exercise~\ref{problem2chapter13} to conclude that $\phi_n(x)$ is of Sheffer $A$-type $2$. We now wish to find the operator to which $\phi_n(x)$ belongs.

Let

$$J(x, \mathscr{D}) = \displaystyle\sum_{k=0}^{\infty} T_k(x) \mathscr{D}^{k+1}.$$

We first proceed by brute strength methods. We know that

$$\mathscr{L}_0(x)=1,$$

$$\mathscr{L}_1(x) = 1-x,$$

$$\mathscr{L}_2(x) = 1-2x+\dfrac{1}{2}x^2,$$

$$\mathscr{L}_3(x) = 1-3x+\dfrac{3}{2}x^2 - \dfrac{1}{6}x^3,$$

$$\mathscr{L}_4(x) = 1-4x+3x^2-\dfrac{2}{3}x^3+\dfrac{1}{24}x^4,$$

then $J(x,\mathscr{D}) \phi_n(x) = \phi_{n-1}(x)$ requires that

$$T_0(x) \mathscr{D} \dfrac{1-x}{(1!)^2} = \dfrac{1}{(0!)^2}; T_0(x)(-1)=1, T_0(x)=-1.$$

Then also

$$[T_0(x) \mathscr{D} + T_1(x) \mathscr{D}^2 ] \dfrac{1}{(2!)^2} \mathscr{L}_2(x) = \dfrac{1}{(1!)^2} \mathscr{L}_1(x),$$

or

$$[\mathscr{D}+ T_1(x) \mathscr{D}^2] \left(1-2x+\dfrac{1}{2}x^2 \right) = 4(1-x)$$

$$-(-2+x)+T_1(x)(1) = 4-4x; T_1(x) = 2-3x.$$

Next we have

$$[-\mathscr{D} + (2-3x)\mathscr{D}^2 + T_2(x)\mathscr{D}^3]\left(1-3x+\dfrac{3}{2} x^2- \dfrac{1}{6}x^3 \right) = 9 \left(1-2x+\dfrac{1}{2}x^2 \right).$$

or

$$-\left(-3+3x-\dfrac{1}{2}x^2 \right) + (2-3x)(3-x) + T_2(x)(-1) = 9-18x+9x^2,$$

or

$$3-3x + \dfrac{1}{2}x^2 + 6-11x + 3x^2 - T_2(x) = 9-18x + \dfrac{9}{2}x^2,$$

$$-T_2(x) = -4x+x^2; T_2(x)=4x-x^2$$

Then
\begin{eqnarray*}
\lefteqn{[-\mathscr{D} + (2-3x)\mathscr{D}^2 + (4x-x^2) \mathscr{D}^3 + T_4(x) \mathscr{D}^4]} \\
& & \cdot\left(1-4x+3x^2-\dfrac{2}{3}x^3 + \dfrac{1}{24}x^4 \right)  = 16 \left( 1-3x + \dfrac{3}{2} x^2 - \dfrac{1}{6} x^3 \right),
\end{eqnarray*}

or

\begin{eqnarray*}
\lefteqn{-\left(-4+6x-2x^2+\dfrac{1}{6}x^3 \right)+(2-3x)\left(6-4x+\dfrac{1}{2}x^2 \right)} \\
& & + (4x-x^2)(-4+x) + T_4(x) = 16 -48x + 24x^2 - \dfrac{8}{3}x^3.
\end{eqnarray*}

Thus we get

$$T_4(x) = x^2 + \dfrac{x^3}{6}(1+9+6-16) = x^2; T_4(x) = x^2.$$

Let us consider the effect of the operator

$$J(x,\mathscr{D}) = -\mathscr{D} + (2-3x)\mathscr{D}^2 + (4x-x^2)\mathscr{D}^3+x^2\mathscr{D}^4$$

upon $\phi_n(x) = \dfrac{\mathscr{L}_n(x)}{(n!)^2}.$
\end{solution}
%%%%
%%
%%
%%%%
\begin{problem}\label{problem8chapter13}
Prove that if we know a generating function

$$y(x,t) = \displaystyle\sum_{n=0}^{\infty} \phi_n(x) t^n$$

or the simple set of polynomials $\phi_n(x)$ belonging to a Sheffer operator $J(x,D)$, no matter what the $A$-type of $\phi_n(x)$, we can obtain a generating fuction (sum the series)

$$\displaystyle\sum_{n=0}^{\infty} \phi_n(x) t^n$$

for any other polynomial $\phi_n(x)$ belonging to the same operator $J(x,D)$.
\end{problem}
\begin{solution}
We are given that $\phi_n(x)$ is a simple set of polynomials and

$$(1) \hspace{30pt} y(xt) = \displaystyle\sum_{n=0}^{\infty} \phi_n(x) t^n.$$

Let $\phi_n(x)$ belong to the operator $J(x,\mathscr{D})$ in the Sheffer sense. Let $\psi_n(x)$ belong to the same operator $J(x,\mathscr{D})$. Then, by Theorem~71, there exist constants $b_k$ such that

$$(2) \hspace{30pt} \psi_n(x) = \displaystyle\sum_{k=0}^n b_k \phi_{n-k}(x).$$

It follows from $(1)$ and $(2)$ that

$$\begin{array}{ll}
\displaystyle\sum_{n=0}^{\infty} \phi_n(x) t^n &= \displaystyle\sum_{n=0}^{\infty} \displaystyle\sum_{k=0}^{n} b_k \phi_{n-k}(x) t^n \\
&= \left( \displaystyle\sum_{n=0}^{\infty} b_n t^n \right) \left( \displaystyle\sum_{n=0}^{\infty} \phi_n(x) t^n \right) \\
&= B(t) y(x,t).
\end{array}$$

Note that the $b_k$'s can be computed identically.
\end{solution}
%%%%
%%
%%
%%%%
\begin{problem}\label{problem9chapter13}
Obtain a theorem analogous to that of Exercise~\ref{problem8chapter13} but with Sheffer $A$-type replaced by $\sigma$-type.
\end{problem}
\begin{solution}
Follow the proof in Exercise~\ref{problem8chapter13} except for two substitutions:

(1): Replace $J(x,\mathscr{D})$ by $J(x,\sigma)$;

(2): Replace Theorem 71 by Theorem 78.
\end{solution}
%%%%
%%
%%
%%%%
\begin{problem}\label{problem10chapter13}
Show that if $P_n(x)$ is the Legendre polynomial,

$$\phi_n(x) = \dfrac{(1+x^2)^{\frac{1}{2}n}}{n!} P_n \left( \dfrac{x}{\sqrt{1+x^2}} \right)$$

is a simple set of polynomials of Sheffer $A$-type zero.
\end{problem}
\begin{solution}
Consider

$$\phi_n(x) = \dfrac{(1+x^2)^{\frac{n}{2}}}{n!} P_n \left( \dfrac{x}{\sqrt{1+x^2}} \right).$$

We know that

$$\displaystyle\sum_{n=0}^{\infty} \dfrac{P_n(u) v^n}{n!} = e^{uv} {}_0F_1 \left( -;1;\dfrac{v^2(u^2-1)}{4} \right).$$

Therefore

$$\begin{array}{ll}
\displaystyle\sum_{n=0}^{\infty} \phi_n(x) t^n &= \displaystyle\sum_{n=0}^{\infty} \dfrac{P_n \left( \frac{x}{\sqrt{1+x}} \right)(1+x^2)^{\frac{n}{2}}t^n}{n!}\\
&= e^{t \sqrt{1+x^2} \frac{x}{\sqrt{1+x^2}}} {}_0F_1 \left[ \begin{array}{rlr}
-; & & \\
& & \dfrac{t^2(1+x^2) \left( \frac{x^2}{1+x^2}-1 \right)}{4} \\
1; & & 
\end{array} \right] \\
&= e^{xt} {}_0F_1 \left[ \begin{array}{rlr}
-; & & \\
& & \dfrac{-t^2}{4} \\
1; & & 
\end{array} \right].
\end{array}$$

Hence $\phi_n(x)$ is of Sheffer $A$-type zero with $H(t) = t, A(t) = {}_0F_1 \left(-;1; -\dfrac{t^2}{4} \right).$
\end{solution}
%%%%
%%
%%
%%%%
\begin{problem}\label{problem11chapter13}
Let $P_n(x)$ be the Legendre polynomial. Choose $\sigma=D\theta$ and show that the polynomials 

$$\phi_n(x) = \dfrac{(x-)^n}{(n!)^2} P_n \left( \dfrac{x+1}{x-1} \right)$$

are of $\sigma$-type zero for that $\sigma$.
\end{problem}
\begin{solution}
Consider

$$\phi_n(x) = \dfrac{(x-1)^n}{(n!)^2} P_n \left( \dfrac{x+1}{x-1} \right),$$

where $P_n(x)$ is the Legendre polynomial.

We know that

$$(1) \hspace{30pt} {}_0F_1 \left(-;1;\dfrac{v(u-1)}{2} \right) {}_0F_1 \left(-;1;\dfrac{v(u+1)}{2} \right) = \displaystyle\sum_{n=0}^{\infty} \dfrac{P_n(u) v^n}{(n!)^2}.$$

In $(1)$ put $v=t(x-1), u=\dfrac{x+1}{x_1}$. Then $\dfrac{u-1}{2}=\dfrac{1}{x-1}, \dfrac{u+1}{2}=\dfrac{x}{x-1},$ and $(1)$ yields 

$$(2) \hspace{30pt} {}_0F_1(-;1;t){}_0F_1(-;1;xt) = \displaystyle\sum_{n=0}^{\infty} \phi_n(x) t^n.$$

Using Theorem 79 we may conclude from $(2)$ that $\phi_n(x)$ is of $\sigma$-type zero with $\sigma=\mathscr{D}(\theta+1-1) =\mathscr{D}\theta$ and 

$$A(t) = {}_0F_1(-;1;t),$$

$$H(t) = t,$$

$$\phi(t) = {}_0F_1(-;1;t).$$
\end{solution}
%%%%
%%
%%
%%%%
\begin{problem}\label{problem12chapter13}
Prove that if the operator $J(x,D)$ is such that a simple set of polynomials $\phi_n(x)$ belongs to it in the Sheffer sense, then $J(x,D)$ transforms every polynomial of degree precisely $n$ into a polynomial of degree precisely $(n-1)$.
\end{problem}
\begin{solution}
This result was proved as part of our solution to Problem~2 of this chapter. See the theorem at the end of that solution.

$\mathbf{Theorem:}$ If the operator $J(x, \mathscr{D})$ is such that a simple set of polynomials $\psi_n(x)$ belongs to it in the Sheffer sense,

$$(1) \hspace{30pt} J(x,\mathscr{D}) \psi_n(x) = \psi_{n-1}(x), J(x,\mathscr{D}) \psi_0(x) = 0,$$

then $J(x ,\mathscr{D})$ transforms every polynomial of degree precisely $n$ into a polynomial of degree precisely $n-1$.

$\mathbf{Proof:}$ Let $g_n(x)$ be of degree exactly $n$. Then we know there exists the expansion

$$g_n(x)=\displaystyle\sum_{k=0}^n A(k,n) \psi_{n-k}(x), A(0,n) \neq 0,$$

merely because the $\psi_n(x)$ form a simple set. Then, because of $(1)$,

$$(2) \hspace{30pt} J(x,\mathscr{D}) g_n(x) = \displaystyle\sum_{k=0}^{n-1} A(k,n) \psi_{n-1-k}(x),$$

and the right member of $(2)$ is of degree exactly $(n-1).$
\end{solution}