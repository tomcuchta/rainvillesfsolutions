%%%%
%%
%%
%%%%
%%%% CHAPTER 8
%%%% CHAPTER 8
%%%%
%%
%%
%%%%
\section{Chapter 8 Solutions}
\begin{center}\hyperref[toc]{\^{}\^{}}\end{center}
\begin{center}\begin{tabular}{lllllllllllllllllllllllll}
\hyperref[problem1chapter8]{P1} & \hyperref[problem2chapter8]{P2} & \hyperref[problem3chapter8]{P3} & \hyperref[problem4chapter8]{P4} & \hyperref[problem5chapter8]{P5} & \hyperref[problem6chapter8]{P6} & \hyperref[problem7chapter8]{P7} & \hyperref[problem8chapter8]{P8} & \hyperref[problem9chapter8]{P9}
\end{tabular}\end{center}
\setcounter{problem}{0}
\setcounter{solution}{0}
\begin{problem}\label{problem1chapter8}
From $e^t \psi(xt) = \displaystyle\sum_{n=0}^{\infty} \sigma_n(x)t^n,$ show that

$$\sigma_n(xy) = \displaystyle\sum_{k=0}^n \dfrac{y^k (1-y)^{n-k} \sigma_k(x)}{(n-k)!},$$

and in particular that

$$2^n \sigma_n \left( \dfrac{1}{2} x \right) = \displaystyle\sum_{k=0}^n \dfrac{\sigma_k(x)}{(n-k)!}.$$
\end{problem}
\begin{solution}
Let 

$$e^t \psi(xt) = \displaystyle\sum_{n=0}^{\infty} \sigma_n(x)t^n.$$

Then

$$e^t \psi(xyt) = \displaystyle\sum_{n=0}^{\infty} \sigma_n(xy)t^n$$

and

$$e^{yt} \psi(xyt) = \displaystyle\sum_{n=0}^{\infty} \sigma_n(x) y^n t^n.$$

But

$$e^t \psi(xyt) = e^{(1-y)t} e^{yt} \psi(xyt)$$

so that

$$\begin{array}{ll}
\displaystyle\sum_{n=0}^{\infty} \sigma_n(xy)t^n &= \left( \displaystyle\sum_{n=0}^{\infty} \dfrac{(1-y)^n t^n}{n!} \right) \left( \displaystyle\sum_{n=0}^{\infty} \sigma_n(x) y^n t^n \right)\\ 
&= \displaystyle\sum_{n=0}^{\infty} \displaystyle\sum_{k=0}^n \dfrac{(1-y)^{n-k} y^k \sigma_k(x)}{(n-k)!} t^n. \\
\end{array}$$

Hence

$$\sigma_n(xy) = \displaystyle\sum_{k=0}^n \dfrac{y^k (1-y)^{n-k} \sigma_k(x)}{(n-k)!}.$$

For $y = \dfrac{1}{2}$, we obtain

$$2^n \sigma_n \left( \dfrac{1}{2} x \right) = \displaystyle\sum_{k=0}^n \dfrac{\sigma_k(x)}{(n-k)!}.$$
\end{solution}

%%%%
%%
%%
%%%%
\begin{problem}\label{problem2chapter8}
Consider the set (called Appell polynomials) $\alpha_n(x)$ generated by

$$e^{xt} A(t) = \displaystyle\sum_{n=0}^{\infty} \alpha_n(x) t^n.$$

Show that $\alpha'_0(x) = 0$, and that for $n \geq 1$, $\alpha_n'(x) = \alpha_{n-1}(x).$
\end{problem}
\begin{solution}
Consider $\alpha_n(x)$ defined by

$$(A) \hspace{30pt} e^{xt}A(t) = \displaystyle\sum_{n=0}^{\infty} \alpha_n(x) t^n.$$

From $(A)$ we obtain

$$t e^{xt} A(t) = \displaystyle\sum_{n=0}^{\infty} \alpha_n'(x) t^n.$$

Hence

$$\displaystyle\sum_{n=0}^{\infty} \alpha_n(x) t^{n+1} = \displaystyle\sum_{n=0}^{\infty} \alpha_n'(x) t^n,$$

or

$$\displaystyle\sum_{n=1}^{\infty} \alpha_{n-1}(x)t^n = \displaystyle\sum_{n=0}^{\infty} \alpha_n'(x) t^n.$$

Therefore, $\alpha_0'(x) =0$, and, for $n \geq 1,$ $\alpha_n'(x) = \alpha_{n-1}(x).$
\end{solution}
%%%%
%%
%%
%%%%
\begin{problem}\label{problem3chapter8}
Apply Theorem 50, page 141, to the polynomials $\sigma_n(x)$ of Section 73 and thus obtain Theorem 45.
\end{problem}
\begin{solution}
We have

$$e^t \psi(xt) = \displaystyle\sum_{n=0}^{\infty} \sigma_n(x) t^n$$

and wish to apply Theorem~50, page 239. In the notation of Theorem~50 we have

$$A(t)e^t, H(t)=t, \psi(t) = \displaystyle\sum_{n=0}^{\infty} \gamma_n t^n, \gamma_0 \neq 0.$$

Now $A(t) = e^t = \displaystyle\sum_{n=0}^{\infty} \dfrac{t^n}{n!},$, so $a_n = \dfrac{1}{n}; H(t) = t$, so $h_0=1, h_n=0$ for $n \geq 1.$

Then also

$$\dfrac{t A'(t)}{A(t)} = \dfrac{te^t}{e^t} = t = \displaystyle\sum_{n=0}^{\infty} \alpha_n t^{n+1},$$

so that $\alpha_0=1, \alpha_n=0$ for $n \geq 1$. 

Furthermore,

$$\dfrac{t H'(t)}{H(t)} = \dfrac{t \cdot 1}{t} = 1 = 1 + \displaystyle\sum_{n=0}^{\infty} \beta_n t^{n+1}$$

so that $\beta_n=0$ for $n \geq 0$. Hence by Theorem~50,

$$x \sigma_n'(x) - n \sigma_n(x) = -1 \cdot \sigma_{n-1-0}(x) = -\sigma_{n-1}(x),$$

which is Theorem~45, page 224.
\end{solution}
%%%%
%%
%%
%%%%
\begin{problem}\label{problem4chapter8}
The polynomials $\sigma_n(x)$ of Exercise~\ref{problem3chapter8} and Section 73 are defined by 

$$(A) \hspace{30pt} e^t \psi(xt) = \displaystyle\sum_{n=0}^{\infty} \sigma_n(x) t^n,$$

but by equation (9), page 134, they  also satisfy

$$(B) \hspace{30pt} (1-t)^{-c} F \left( \dfrac{xt}{1-t} \right) = \displaystyle\sum_{n=0}^{\infty} (c)_n \sigma_n(x) t^n,$$

for a certain function $F$. By applying Theorem 50, page 141, to $(B)$, conclude that the $\sigma_n(x)$ of $(A)$ satisfy the relation

$$(c)_n[x \sigma_n'(x) - n \sigma_n(x)] = - \displaystyle\sum_{k=0}^{n-1} (c)_k [c \sigma_k(x) + x \sigma_k'(x)],$$

for arbitrary $c$.
\end{problem}
\begin{solution}
For the $\sigma_n(x)$ of Exercise~\ref{problem3chapter8} above, we know that

$$(B) \hspace{30pt} (1-t)^{-c} F \left( \dfrac{xt}{1-t} \right) = \displaystyle\sum_{n=0}^{\infty} (c)_n \sigma_n(x) t^n,$$

where

$$F(u) = \displaystyle\sum_{n=0}^{\infty} (c)_n \gamma_n u^n$$

in terms of $\psi(u) = \displaystyle\sum_{n=0}^{\infty} \gamma_n u^n.$

We now use Theorem~50 on the polynomials $f_n(x) = (c)_n \sigma_n(x)$ of $(B)$. 

Here

$$A(t) = (1-t)^{-c}, H(t) = \dfrac{t}{1-t}.$$

Then

$$\log A(t) = -c \log(1-t),$$

$$H(t) = -1 + \dfrac{1}{1-t}; H'(t) = \dfrac{1}{(1-t)^3},$$

$$\dfrac{A'(t)}{A(t)} = \dfrac{c}{1-t},$$

$$\dfrac{t H'(t)}{H(t)} = \dfrac{t}{(1-t)^2} \dfrac{1-t}{t} = \dfrac{1}{1-t} = 1 + \displaystyle\sum_{n=1}^{\infty} t^n.$$

Hence

$\dfrac{t A'(t)}{A(t)} = \displaystyle\sum_{n=0}^{\infty} ct^{n+1},$ so $\alpha_n=c$ for $n \geq 0$; 

$\dfrac{t H'(t)}{H(t)} = 1 + \displaystyle\sum_{n=0}^{\infty} t^{n+1},$ so $\beta_n=1$ for $n \geq 0$.

Thus Theorem~50 yields

$$x (c)_{n} \sigma_{n}'(x) - n(c)_n \sigma_n(x) = - \displaystyle\sum_{k=0}^{n-1} c (c)_{n-1-k} \sigma_{n-1-k}(x) - x \displaystyle\sum_{k=0}^{n-1} 1 \cdot (c)_{n-1-k} \sigma'_{n-1-k}(x),$$

or, with reversal of order of summation,

$$(c)_n [x \sigma_n'(x) - n \sigma_n(x)] = - \displaystyle\sum_{k=0}^{n-1} (c)_k [c \sigma_k(x) + x \sigma_k'(x)].$$
\end{solution}
%%%%
%%
%%
%%%%
\begin{problem}\label{problem5chapter8}
Apply Theorem 50, page 141, to the polynomials $y_n(x)$ defined by $(1)$, page 135. You do not, of course, get Theorem 47, since that theorem depended upon the specific character of the exponential.
\end{problem}
\begin{solution}
On page 228 we find

$$A(t) \exp \left( \dfrac{-xt}{1-t} \right) = \displaystyle\sum_{n=0}^{\infty} y_n(x) t^n.$$

In the notation of Theorem~50 we have

$$A(t) = A(t), H(t) = - \dfrac{t}{1-t} = 1 - \dfrac{1}{1-t}$$

from which

$$H'(t) = -\dfrac{1}{(1-t)^2}, \dfrac{tH'(t)}{H(t)} = -\dfrac{-t}{(1-t)^2} \cdot \dfrac{1-t}{-t} = \dfrac{1}{1-t} = 1 + \displaystyle\sum_{n=1}^{\infty} t^n.$$

so

$$\dfrac{t A'(t)}{A(t)} = \displaystyle\sum_{n=0}^{\infty} \alpha_n t^{n+1}$$

and

$$\dfrac{t H'(t)}{H(t)} = 1 + \displaystyle\sum_{n=0}^{\infty} t^{n+1},$$

so that $\beta_n=1$ for $n \geq 0$.

Therefore, we may conclude that there exist constants $\alpha_n$ such that

$$x y_n'(x) - ny_n(x) = - \displaystyle\sum_{k=0}^{n-1} \alpha_k y_{n-1-k}(x) - x \displaystyle\sum_{k=0}^{n-1} 1 \cdot y_{n-1-k}'(x).$$
\end{solution}
%%%%
%%
%%
%%%%
\begin{problem}\label{problem6chapter8}
Apply Theorem 50 to the Laguerre polynomials through the genrating relation 914), page 135, to get

$$x DL_n^{(\alpha)}(x) - nL_n^{(\alpha)}(x) = - \displaystyle\sum_{k=0}^{n-1}[(1 + \alpha) L_k^{(\alpha)}(x)+x DL_k^{(\alpha)}(x)=0$$

for the Laguerre polynomials.
\end{problem}
\begin{solution}
From page 228 we have

$$(1-t)^{-1-\alpha} \exp \left( \dfrac{-xt}{1-t} \right) = \displaystyle\sum_{n=0}^{\infty} \mathscr{L}_n^{(\alpha)}(x)t^n.$$

We may therefore use Theorem~50 with

$$A(t) = (1-t)^{-1-\alpha},$$

$$H(t) = -\dfrac{t}{1-t} = 1 - \dfrac{1}{1-t},$$

$$\log A(t) = -(1+  \alpha) \log(1-t),$$

$$H'(t) = - \dfrac{1}{(1-t)^2},$$

$$\dfrac{tA'(t)}{A(t)} = \dfrac{(1+\alpha)t}{1-t} = (1 + \alpha) \displaystyle\sum_{n=0}^{\infty} t^{n+1},$$

$$\dfrac{tH'(t)}{H(t)} = -\dfrac{t}{(1-t)^2} \dfrac{1-t}{-t} = \dfrac{1}{1-t} = 1 + \displaystyle\sum_{n=1}^{\infty} t^n = 1+ \displaystyle\sum_{n=0}^{\infty} t^{n+1}.$$

Hence

$$\alpha_n=1+\alpha, n \geq 0;$$

$$\beta_n=1, n \geq 0.$$

Therefore we obtain

$$x \mathscr{D} \mathscr{L}_n^{(\alpha)}(x) - n \mathscr{L}_n^{(\alpha)}(x) = -(1+\alpha) \displaystyle\sum_{k=0}^{n-1} \mathscr{L}_{n-1-k}^{(\alpha)}(x) - x \displaystyle\sum_{k=0}^{n-1} \mathscr{D} \mathscr{L}_{n-1-k}^{(\alpha)}(x)$$

or

$$(A) \hspace{30pt} x \mathscr{D} \mathscr{L}_n^{(\alpha)}(x) - n \mathscr{L}_n^{(\alpha)}(x) = - \displaystyle\sum_{k=0}^{n-1} [ (1 + \alpha)\mathscr{L}_k^{(\alpha)}(x) + x \mathscr{D} \mathscr{L}_k^{(\alpha)}(x)].$$

On page 230 we had

$$(B) \hspace{30pt} \mathscr{D} \mathscr{L}_n^{(\alpha)}(x) = - \displaystyle\sum_{k=0}^{n-1} \mathscr{L}_k^{(\alpha)}(x).$$

Using $(B)$ with $(A)$ we obtain

$$x \mathscr{D} \mathscr{L}_n^{(\alpha)}(x) - n \mathscr{L}_n^{(\alpha)}(x) = (1 + \alpha) \mathscr{D} \mathscr{L}_n^{(\alpha)}(x) + x \mathscr{D}^2 \mathscr{L}_n^{(\alpha)}(x),$$

or

$$[x \mathscr{D}^2 + (1 + \alpha - x) \mathscr{D} + n] \mathscr{L}_n^{(\alpha)}(x) = 0,$$

as desired.
\end{solution}
%%%%
%%
%%
%%%%
\begin{problem}\label{problem7chapter8}
The Humbert polynomials $h_n(x)$ are defined by

$$(1-3xt+t^3)^{-p} = \displaystyle\sum_{n=0}^{\infty} h_n(x) t^n.$$

Use Theorem 52, page 144, to conclude that

$$xh_n'(x) - nh_n(x) = h'_{n-2}(x).$$
\end{problem}
\begin{solution}
For $h_n(x)$ we have

$$(1 - 3xt + t^3)^{- \nu} = \displaystyle\sum_{n=0}^{\infty} h_n(x)t^n = {}_1F_0 \left[ \begin{array}{rlr}
\nu; & & \\
& & 3xt-t^3 \\
-; & &
\end{array} \right].$$

In the notation of Theorem~52 we have

$$A(t)=1, H(t)=3t, g(t)=-t^3, \psi(t) = (1-t)^{-\nu}.$$

Then

$$\dfrac{t A'(t)}{A(t)} = 0, \dfrac{t H'(t)}{H(t)} = \dfrac{3t}{3t} = 1, \dfrac{t g'(t)}{H(t)} = \dfrac{t(-3t^2)}{3t} = -t^2.$$

Hence

$\alpha_n=0, n \geq =0; \beta_n=0, n \geq 0; \delta_1=-1, \delta_n=0$ for $n =0, n \geq 2.$

Therefore Theorem~52 yields

$$x h_n'(x) - nh_n(x) = h_{n-2}'(x).$$
\end{solution}
%%%%
%%
%%
%%%%
\begin{problem}\label{problem8chapter8}
For the $y_n(x)$ of Section~74 show that

$$F = A(t) \exp \left( \dfrac{-xt}{1-t} \right)$$

satisfies the equation

$$x \dfrac{\partial F}{\partial x} - t \dfrac{\partial F}{\partial t} = -t^2 \dfrac{\partial F}{\partial t} - \dfrac{(1-t)t A'(t)}{A(t)}F$$

and draw what conclusions you can about $y_n(x)$.
\end{problem}
\begin{solution}
Let $F = A(t) \exp \left( \dfrac{-xt}{1-t} \right) = \displaystyle\sum_{n=0}^{\infty} y_n(x) t^n.$

Then

$$\dfrac{\partial F}{\partial x} = -\dfrac{t}{1-t} A(t) \exp \left( \dfrac{-xt}{1-t} \right)$$

$$\dfrac{\partial F}{\partial t} = -\dfrac{x}{(1-t)^2} A(t) \exp \left( \dfrac{-xt}{1-t} \right) + A'(t) \exp \left( \dfrac{-xt}{1-t} \right).$$

Thus we obtain

$$x \dfrac{\partial F}{\partial x} - t(1-t) \dfrac{\partial F}{\partial t} = - t(1-t) \dfrac{A'(t)}{A(t)} F,$$

or

$$x \dfrac{\partial F}{\partial x} - t \dfrac{\partial F}{\partial t} = - t^2 \dfrac{\partial F}{\partial t} - (1-t) \dfrac{tA'(t)}{A(t)} F.$$

Now let

$$\dfrac{t A'(t)}{A(t)} = \displaystyle\sum_{n=0}^{\infty} \alpha_n t^{n+1}.$$

Then

\begin{eqnarray*}
\lefteqn{\displaystyle\sum_{n=0}^{\infty} [xy_n'(x) - ny_n(x)]t^n} \\
& &=\!-\!\displaystyle\sum_{n=1}^{\infty}\!y_n(x) c t^{n+1}\! -\! \left(\! \displaystyle\sum_{n=0}^{\infty} \alpha_n t^{n+1}\! \right)\! \left( \displaystyle\sum_{n=0}^{\infty} y_n(x) t^n \!\right)\! +\! \left(\! \displaystyle\sum_{n=0}^{\infty} \alpha_n t^{n+2}\! \right)\! \left(\! \displaystyle\sum_{n=0}^{\infty} y_n(x) t^n\! \right) \\
&&= -\displaystyle\sum_{n=0}^{\infty} n y_n(x) t^{n+1} - \displaystyle\sum_{n=0}^{\infty} \displaystyle\sum_{k=0}^n \alpha_{n-k} y_k(x) t^{n+1} + \displaystyle\sum_{n=0}^{\infty} \displaystyle\sum_{k=0}^n \alpha_{n-k} y_k(x)t^{n+2} \\
&&= - \displaystyle\sum_{n=2}^{\infty} (n-1) y_{n-1}(x)t^n - \displaystyle\sum_{n=1}^{\infty} \displaystyle\sum_{k=0}^{n-1} \alpha_{n-1-k}y_k(x)t^n + \displaystyle\sum_{n=2}^{\infty} \displaystyle\sum_{k=0}^{n-2} \alpha_{n-2-k}y_k(x) t^n.
\end{eqnarray*}

Hence we get $y_0'(x) = 0, xy_1'(x) - y_1(x) = -\alpha_0 y_0(x),$ and for $n \geq 2,$

\begin{eqnarray*}
\lefteqn{x y_n'(x) - ny_n(x)} \\
&& = -(n-1)y_{n-2}(x) - \displaystyle\sum_{k=0}^{n-2} (\alpha_{n-1-k}-\alpha_{n-2-k}) y_k(x) - \alpha_0 y_{n-1}(x).
\end{eqnarray*}

We may also write

$$x(1-t) \dfrac{\partial F}{\partial x} - t(1-t) \dfrac{\partial F}{\partial t} = -xt \dfrac{\partial F}{\partial x} - t(1-t) \dfrac{A'(t)}{A(t)} F,$$

or

$$x \dfrac{\partial F}{\partial x} - t \dfrac{\partial F}{\partial t} = - \dfrac{xt}{1-t} \dfrac{\partial F}{\partial x} - \dfrac{t A'(t)}{A(t)} F,$$

from which it follows that

\begin{eqnarray*}
\lefteqn{\displaystyle\sum_{n=0}^{\infty} [x y_n'(x) - ny_n(x)] y^n} \\
& &= - x \left( \displaystyle\sum_{n=0}^{\infty} t^{n+1} \right) \left( \displaystyle\sum_{n=0}^{\infty} y_n'(x) t^n \right) - \left( \displaystyle\sum_{n=0}^{\infty} \alpha_n t^{n+1} \right) \left( \displaystyle\sum_{n=0}^{\infty} y_n(x) t^n \right) \\
& &= -x \displaystyle\sum_{n=0}^{\infty} \displaystyle\sum_{k=0}^n y_k'(x) t^{n+1} - \displaystyle\sum_{n=0}^{\infty} \displaystyle\sum_{k=0}^n \alpha_{n-k} y_k(t) t^{n=1}.
\end{eqnarray*}

Hence, for $n \geq 1$,

$$x y_n'(x) - ny_n(x) = - \displaystyle\sum_{k=0}^{n-1} [x y_k'(x) + \alpha_{n-1-k} y_k(x)].$$
\end{solution}
%%%%
%%
%%
%%%%
\begin{problem}\label{problem9chapter8}
For polynomials $a_n(x)$ defined by

$$(1-t)^{-c} A \left( \dfrac{-xt}{1-t} \right) = \displaystyle\sum_{n=0}^{\infty} a_n(x) t^n$$

obtain what results you can parallel to those of Theorem~48, page 137.
\end{problem}
\begin{solution}
Let $a_n(x)$ be defined by

$$(1) \hspace{30pt} (1-t)^{-c} A \left( \dfrac{-xt}{1-t} \right) = \displaystyle\sum_{n=0}^{\infty} a_n(x)t^n.$$

Let

$$(2) \hspace{30pt} A(u) = \displaystyle\sum_{n=0}^{\infty} \alpha_n u^n.$$

Put

$$F = (1-t)^{-c} A \left( \dfrac{-xt}{1-t} \right).$$

Then

$$\dfrac{\partial F}{\partial x} = -t(1-t)^{-c-1} A'$$

$$\dfrac{\partial F}{\partial t} = c(1-t)^{-c-1} A - x(1-t)^{-c-2} A'.$$

Hence

$$x \dfrac{\partial F}{\partial x} - t(1-t) \dfrac{\partial F}{\partial t} = -ctF.$$

From which we may write

$$(3) \hspace{30pt} x \dfrac{\partial F}{\partial x} - t \dfrac{\partial F}{\partial t} = -ctF - t^2 \dfrac{\partial F}{\partial t}.$$

From $(3)$ we obtain 

$$\begin{array}{ll}
\displaystyle\sum_{n=0}^{\infty} [xa_n'(x) - na_n(x)]t^n &= - c \displaystyle\sum_{n=0}^{\infty} a_n(x) t^{n+1} - \displaystyle\sum_{n=0}^{\infty} n g_n(x) t^{n+1} \\
&= - \displaystyle\sum_{n=1}^{\infty} [c+n-1] g_{n-1}(x) t^n.
\end{array}$$

Therefore, $a_0'(x) =0$ and for $n \geq 2,$

$$x a_n'(x) - na_n(x) = -(c+n-1)a_{n-1}(x).$$

We may rewrite equation $(3)$ in the form

$$x(1-t) \dfrac{\partial F}{\partial x} - t(1-t) \dfrac{\partial F}{\partial t} = -ctF - xt \dfrac{\partial F}{\partial x}.$$

This leads to the identity

$$x \dfrac{\partial F}{\partial x} - t \dfrac{\partial F}{\partial t} = -\dfrac{ct}{1-t} F - \dfrac{xt}{1-t} \dfrac{\partial F}{\partial x}$$

from which we get

\begin{eqnarray*}
\lefteqn{\displaystyle\sum_{n=0}^{\infty} [x a_n'(x) - n a_n(x)]t^n} \\
& &= -c \left( \displaystyle\sum_{n=0}^{\infty} t^{n+1} \right) \left( \displaystyle\sum_{n=0}^{\infty} a_n(x) t^n \right) - x \left( \displaystyle\sum_{n=0}^{\infty} t^{n+1} \right) \left( \displaystyle\sum_{n=0}^{\infty} a_n'(x) t^n \right) \\
&&= -c \displaystyle\sum_{n=0}^{\infty} \displaystyle\sum_{k=0}^n a_k(x) t^{n+1} - x \displaystyle\sum_{k=0}^{\infty} \displaystyle\sum_{k=0}^n a_k'(x) t^{n+1} \\
&&= - \displaystyle\sum_{n=0}^{\infty} \displaystyle\sum_{k=0}^{n-1} [c a_k(x) + x a_k'(x) ] t^n.
\end{eqnarray*}

Hence we obtain the relation

$$xa_n'(x) - na_n(x) = - \displaystyle\sum_{k=0}^{n-1} [c a_k(x) + x a_k'(x)].$$

In $(1)$, let us put $v = -\dfrac{t}{1-t}$. Then $t = -\dfrac{v}{1-t}$ and $1-t = \dfrac{1}{1-v}$.

Hence $(1)$ becomes 

$$(1-v)^c A(xv) = \displaystyle\sum_{n=0}^{\infty} \dfrac{a_n(x) (-v)^n}{(1-v)^n},$$

or

$$\begin{array}{ll}
\displaystyle\sum_{n=0}^{\infty} \alpha_n x^n v^n &= \displaystyle\sum_{n=0}^{\infty} \dfrac{a_n(x) (-1)^n v^n}{(1-v)^{n+c}} \\
&= \displaystyle\sum_{n,k=0}^{\infty} \dfrac{(-1)^n a_n(x) (c)_{n+k} v^{n+k}}{k! (c)_n} \\
&= \displaystyle\sum_{n=0}^{\infty} \displaystyle\sum_{k=0}^n \dfrac{(c)_n (-1)^{n-k} a_{n-k}(x)}{k! (c)_{n-k}} v^n \\
&= \displaystyle\sum_{n=0}^{\infty} \displaystyle\sum_{k=0}^n \dfrac{(-1)^k (c)_n a_k(x)}{(c)_k (n-k)!} v^n,
\end{array}$$

which can be put in various other forms.

Next, from $(1)$ we get

$$\begin{array}{ll}
\displaystyle\sum_{n=0}^{\infty} a_n(x) t^n &= \displaystyle\sum_{n=0}^{\infty} \dfrac{\alpha_n (-x)^n t^n}{(1-t)^{n+c}} \\
&= \displaystyle\sum_{n,k=0}^{\infty} \dfrac{\alpha_n (-1)^n x^n(c)_{n+k} t^{n+k}}{(c)_n k!} \\
&= \displaystyle\sum_{n=0}^{\infty} \displaystyle\sum_{k=0}^n \dfrac{(-1)^{n-k} x^{n-k} \alpha_{n-k} (c)_n t^n}{k1 (c)_{n-k}}
\end{array}$$

or

$$\displaystyle\sum_{n=0}^{\infty} a_n(x) t^n = \displaystyle\sum_{n=0}^{\infty} \displaystyle\sum_{k=0}^n \dfrac{(-1)^k x^k \alpha_k (c)_n t^n}{(n-k)! (c)_k}.$$

Therefore,

$$a_n(x) = \displaystyle\sum_{k=0}^n \dfrac{(-1)^k x^k \alpha_k (c)_n}{(n-k)! (c)_k}.$$
\end{solution}