%%%%
%%
%%
%%%%
%%%% CHAPTER 6
%%%% CHAPTER 6
%%%%
%%
%%
%%%%
\section{Chapter 6 Solutions}
\begin{center}\hyperref[toc]{\^{}\^{}}\end{center}
\begin{center}\begin{tabular}{lllllllllllllllllllllllll}
\hyperref[problem1chapter6]{P1} & \hyperref[problem2chapter6]{P2} & \hyperref[problem3chapter6]{P3} & \hyperref[problem4chapter6]{P4} & \hyperref[problem5chapter6]{P5} & \hyperref[problem6chapter6]{P6} & \hyperref[problem7chapter6]{P7} & \hyperref[problem8chapter6]{P8} & \hyperref[problem9chapter6]{P9} & \hyperref[problem10chapter6]{P10} & \hyperref[problem11chapter6]{P11} & \hyperref[problem12chapter6]{P12} & \hyperref[problem13chapter6]{P13} \\
\hyperref[problem14chapter6]{P14} & \hyperref[problem15chapter6]{P15} & \hyperref[problem16chapter6]{P16} & \hyperref[problem17chapter6]{P17} & \hyperref[problem18chapter6]{P18} & \hyperref[problem19chapter6]{P19} & \hyperref[problem20chapter6]{P20} & \hyperref[problem21chapter6]{P21} & \hyperref[problem22chapter6]{P22} & \hyperref[problem23chapter6]{P23} 
\end{tabular}\end{center}
\setcounter{problem}{0}
\setcounter{solution}{0}
\begin{problem}\label{problem1chapter6}
By collecting powers of $x$ in the summation on the left, show that
$$\displaystyle\sum_{n=0}^{\infty} J_{2n+1}(x) = \dfrac{1}{2} \displaystyle\int_0^x J_0(y) dy.$$
\end{problem}
\begin{solution}
$$\begin{array}{ll}
\displaystyle\sum_{k=0}^{\infty} J_{2k+1}(x) &= \displaystyle\sum_{k,n=0}^{\infty} \dfrac{(-1)^n (\frac{x}{2})^{2n+2k+1}}{n! (n+2k+1)!} \\
&= \displaystyle\sum_{n=0}^{\infty} \displaystyle\sum_{k=0}^{n} \dfrac{(-1)^{n-k}(\frac{x}{2})^{2n+1}}{(n-k)! (n+k+1)!} \\
&= \displaystyle\sum_{n=0}^{\infty} \displaystyle\sum_{k=0}^{n} \dfrac{(-n)_k}{(n+1)_k} \dfrac{(-1)^n (\frac{x}{2})^{2n+1}}{n! (n+1)!} \\
&= \displaystyle\sum_{n=0}^{\infty} {}_2F_1 \left[ \begin{array}{rlr}
-n, 1; & & \\
& & 1 \\
n+2; & &
\end{array} \right] \dfrac{(-1)^n (\frac{x}{2})^{2n+1}}{n! (n+1)!} \\
&= \displaystyle\sum_{n=0}^{\infty} \dfrac{\Gamma(n+2) \Gamma(2n+1) (-1)^n (\frac{x}{2})^{2n+1}}{\Gamma(2n+2) \Gamma(n+1) n! (n+1)!} \\
&= \displaystyle\sum_{n=0}^{\infty} \dfrac{(-1)^n (\frac{x}{2})^{2n+1}}{(2n+1)n! n!} \\
&= \dfrac{1}{2} \displaystyle\int_0^x \displaystyle\sum_{n=0}^{\infty} \dfrac{(-1)^n (\frac{y}{2})^{2n}}{n! n!} dy \\
&= \dfrac{1}{2} \displaystyle\int_0^x J_0(y) dy.
\end{array}$$
\end{solution}
%%%%
%%
%%
%%%%
\begin{problem}\label{problem2chapter6}
Put the equation of Theorem 39, page 113, into the form
$$(A) \hspace{30pt} \exp \left[ \dfrac{1}{2}z(t-t^{-1}) \right] = J_0(z) + \displaystyle\sum_{n=1}^{\infty} J_n(z)[t^n+(-1)^nt^{-n}].$$
Use equation $(A)$ with $t=i$ to conclude that
$$\cos z = J_0(z) + 2 \displaystyle\sum_{k=1}^{\infty} (-1)^k J_{2k}(z),$$
$$\sin z = 2 \displaystyle\sum_{k=0}^{\infty} (-1)^k J_{2k+1}(z).$$
\end{problem}
\begin{solution}
We know that
$$\exp \left[ \dfrac{1}{2} z(t-t^{-1}) \right] = \displaystyle\sum_{n=-\infty}^{\infty} J_n(z)t^n = \displaystyle\sum_{n=-\infty}^{-1} J_n(z)t^n + J_0(z) + \displaystyle\sum_{n=1}^{\infty} J_n(z) t^n.$$
Now $J_{-n}(z) = (-1)^n J_n(z)$. Hence
$$\begin{array}{ll}
\exp \left[ \dfrac{z}{2}(t- \dfrac{1}{t} \right] &= \displaystyle\sum_{n=1}^{\infty} J_{-n}(z)E^n + J_0(z) + \displaystyle\sum_{n=1}^{\infty} J_n(z)Tt^n \\
&= J_0(z) + \displaystyle\sum_{n=1}^{\infty} J_n(z) [t^n + (-1)^n t^{-n}].
\end{array}$$
Now use $t=i$. Then $(-1)^n i^{-n} = i^{2n}i^{-n} = i^n$ and
$$\exp \left[ \dfrac{\pi}{2} \left( i-\dfrac{1}{i} \right) \right] = \exp(iz) = \cos z + i \sin z.$$
Therefore
$$\cos z + i \sin z = J_0(z) + 2 \displaystyle\sum_{n=1}^{\infty} i^n J_n(z).$$
But $J_n(-z) = (-1)^n J_n(z)$, so we have
$$\cos z - i \sin z = J_0(z) + 2 \displaystyle\sum_{n=1}^{\infty} (-1)^n i^n J_n(z).$$
(note: above argument can be done more easily by equating even and odd function of $z$)
Thus we get
$$\cos z = J_0(z) + 2 \displaystyle\sum_{n=1}^{\infty} i^{2n} d_{2n}(z) = J_0(z) + 2 \displaystyle\sum_{n=1}^{\infty} (-1)^n J_{2n}(z)$$
and
$$i \sin z = 2 \displaystyle\sum_{n=0}^{\infty} i^{2n+1} J_{2n+1}(z),$$
or
$$\sin z = 2 \displaystyle\sum_{n=0}^{\infty} (-1)^n J_{2n+1}(z).$$
\end{solution}
%%%%
%%
%%
%%%%
\begin{problem}\label{problem3chapter6}
Use $t=e^{i \theta}$ in equation $(A)$ of Exercise~\ref{problem2chapter6} to obtain the results
$$\cos(z \sin (\theta)) = J_0(z) + 2 \displaystyle\sum_{k=1}^{\infty} J_{2k}(z) \cos(2k \theta),$$
$$\sin (z \sin (\theta)) =  2 \displaystyle\sum_{k=0}^{\infty} J_{2k+1}(z) \sin(2k+1)\theta.$$
\end{problem}
\begin{solution}
Put $t=e^{i \theta}$. Then $t^n + (-1)^n t^{-n} = e^{ni \theta} + (-1)^N e^{-ni \theta}$ and 
$$\exp \left[ \dfrac{z}{2} \left(t - \dfrac{1}{t} \right) \right] = \exp(iz \sin(\theta)) = \cos(z \sin(\theta)) + i \sin(z \sin (\theta)).$$
Also $e^{ni \theta}+(-1)^n e^{-ni \theta} = [ 1 + (-1)^n] \cos (n \theta) + [1 - (-1)^n]  \sin (n \theta).$
From 
$$\exp \left[ \dfrac{z}{2} \left(t - \dfrac{1}{t} \right) \right] = J_0(z) + \displaystyle\sum_{n=1}^{\infty} J_n(z) [t^n + 9-1)^n t^{-n}]$$
we thus obtain
$$\cos(z \sin \theta) + i \sin (z \sin \theta) = J_0(z) + \displaystyle\sum_{n=1}^{\infty} J_n(z) [(1 + (-1)^n) \cos(n \theta) + (1 - (-1)^n) \sin(n \theta)].$$
Now equate even function of $z$ on the two sides, then odd functions of $z$ on the two sides to get
$$\cos(z \sin \theta) = J_0(z) + 2 \displaystyle\sum_{k=1}^{\infty} J_{2k}(z) \cos(2k \theta)$$
and
$$\sin(z \sin \theta) = 2 \displaystyle\sum_{k=0}^{\infty} J_{2k+1}(z) \sin(2k+1)\theta.$$
\end{solution}
%%%%
%%
%%
%%%%
\begin{problem}\label{problem4chapter6}
Use Bessel's integral, page 114, to obtain for integral $n$ in the relations
$$(B) \hspace{30pt} [1+(-1)^n]J_n(z) = \dfrac{2}{\pi} \displaystyle\int_0^{\pi} \cos(n \theta) \cos(z \sin (\theta)) \mathrm{d} \theta,$$
$$(C) \hspace{30pt} [1-(-1)^n]J_n(z) = \dfrac{2}{\pi} \displaystyle\int_0^{\pi} \sin(n \theta) \sin(z \sin (\theta)) \mathrm{d} \theta.$$
With the aid of $(B)$ and $(C)$ show that for integral $k$,
$$J_{2k}(z) = \dfrac{1}{\pi} \displaystyle\int_0^{\pi} \cos(2k \theta) \cos(z \sin (\theta)) \mathrm{d} \theta,$$
$$J_{2k+1}(z) = \dfrac{1}{\pi} \displaystyle\int_0^{\pi} \sin(2k+1) \theta \sin(z \sin (\theta)) \mathrm{d} \theta,$$
$$\displaystyle\int_0^{\pi} \cos(2k+1)\theta \cos(z \sin(\theta) \mathrm{d}\theta = 0,$$
$$\displaystyle\int_0^{\pi} \sin(2k \theta) \sin(z \sin(\theta)) \mathrm{d} \theta = 0.$$
\end{problem}
\begin{solution}
We know that
$$J_n(z) = \dfrac{1}{\pi} \displaystyle\int_0^{\pi} \cos(n \theta - z \sin \theta) \mathrm{d} \theta.$$
Then
$$J_n(z) = \dfrac{1}{\pi} \displaystyle\int_0^{\pi} \cos (n \theta) \cos (z \sin \theta) \mathrm{d} \theta + \dfrac{1}{\pi} \displaystyle\int_0^{\pi} \sin(n \theta) \sin(z \sin \theta) \mathrm{d} \theta.$$
Now change $z$ to $(-z)$ to get 
$$(-1)^n J_n(z) = \dfrac{1}{\pi} \displaystyle\int_0^{\pi} \cos(n \theta) \cos(z \sin \theta) \mathrm{d} \theta - \dfrac{1}{\pi} \displaystyle\int_0^{\pi} \sin(n \theta) \sin(z \sin \theta) \mathrm{d} \theta.$$
We then obtain
$$(B) \hspace{30pt} [1+(-1)^n]J_n(z) = \dfrac{2}{\pi} \displaystyle\int_0^{\pi} \cos(n \theta) \cos(z \sin \theta) \mathrm{d} \theta$$
$$(C) \hspace{30pt} [1 - (-1)^n] J_n(z) = \dfrac{2}{\pi} \displaystyle\int_0^{\pi} \sin(n \theta) \sin(z \sin \theta) \mathrm{d} \theta.$$
Use $(B)$ with $n=2k$ and $(C)$ with $n=2k+1$ to obtain
$$J_{2k}(z) = \dfrac{1}{\pi} \displaystyle\int_0^{\pi} \cos(2k \theta) \cos(z \sin \theta) \mathrm{d} \theta,$$
$$J_{2k+1}(z) = \dfrac{1}{\pi} \displaystyle\int_0^{\pi} \sin(2k+1)\theta \sin(z \sin \theta) \mathrm{d} \theta.$$
Use $(B)$ with $n= 2k+1$ and $(C)$ with $n=2k$ to obtain
$$\displaystyle\int_0^{\pi} \cos((2k+1)\theta) \cos(z \sin \theta) \mathrm{d} \theta = 0,$$
$$\displaystyle\int_0^{\pi} \sin(2k \theta) \sin(z \sin \theta) \mathrm{d} \theta = 0.$$
\end{solution}
%%%%
%%
%%
%%%%
\begin{problem}\label{problem5chapter6}
Expand $\cos(z \sin (\theta))$ and $\sin(z \sin(\theta))$ in Fourier series over the interval $-\pi < \theta < \pi$. Thus use Exercise~\ref{problem4chapter6} to obtain in another way the expansions in Exercise~\ref{problem3chapter6}.
\end{problem}
\begin{solution}
In the interval $-\pi < \theta < \pi$, the Fourier approximation of $f(\theta)$ is
$$f(\theta) = \dfrac{1}{2} a_0 + \displaystyle\sum_{n=1}^{\infty} (a_n \cos (n \theta) + b_n \sin(n \theta)),$$
in which
$$a_n = \dfrac{1}{\pi} \displaystyle\int_{-\pi}^{\pi} f(\theta) \cos(n \theta) \mathrm{d} \theta,$$
$$b_n = \dfrac{1}{\pi} \displaystyle\int_{-\pi}^{\pi} f(\theta) \sin(n \theta) \mathrm{d} \theta.$$
Consider first $f(\theta) = \cos(z \sin \theta)$, an \underline{even} function of $\theta$. For this function
$$a_n = \dfrac{2}{\pi} \displaystyle\int_0^{\pi} \cos(n \theta) \cos(z \sin \theta) \mathrm{d} \theta, b_n=0.$$
By the results in Exercise~\ref{problem4chapter6} we obtain $a_{2k+1}=0, a_{2k}=2 J_{2k}(z).$ Hence
$$\cos(z \sin \theta) = J_0(z) + 2 \displaystyle\sum_{k=1}^{\infty} J_{2k}(z) \cos(2k \theta), - \pi < \theta < \pi.$$
Next, let $f(\theta) = \sin(z \sin \theta),$ an \underline{odd} function of $\theta$. For this function,
$$a_n=0, b_n = \dfrac{2}{\pi} \displaystyle\int_0^{\pi} \sin(n \theta) \sin(z \sin \theta) \mathrm{d} \theta.$$
From Exercise~\ref{problem4chapter6} we get $b_{2k}=0, b_{2k+1}=2 J_{2k+1}(z)$. Hence
$$\sin(z \sin \theta) = 2 \displaystyle\sum_{k=0}^{\infty} J_{2k+1}(z) \sin((2k+1)\theta)), - \pi < \theta < \pi.$$
\end{solution}
%%%%
%%
%%
%%%%
\begin{problem}\label{problem6chapter6}
In the product of $\exp \left[ \dfrac{1}{2}x(t-t^{-1}) \right]$ by $\exp \left[ -\dfrac{1}{2} x(t-t^{-1}) \right],$ obtain the coefficient of $t^0$ and thus show that 
$$J_0^2(x) + 2 \displaystyle\sum_{n=1}^{\infty} J_n^2(x) = 1.$$
For real $x$ conclude that $|J_0(x)| \leq 1$ and $|J_n(x)| \leq 2^{-\frac{1}{2}}$ for $n \geq 1$.
\end{problem}
\begin{solution}
We know that 
$$\exp \left[ \dfrac{\gamma}{2} \left( t - \dfrac{1}{t} \right) \right] = \displaystyle\int_{n=-\infty}^{\infty} J_n(x) t^n$$
and thus that
$$\exp \left[ - \dfrac{\gamma}{2} \left( t - \dfrac{1}{t} \right) \right] = \displaystyle\int_{n=-\infty}^{\infty} (-1)^k J_k(x)J_n(x) t^{n+k}.$$
The coefficient of $t^0$ on the right is $(k=-n)$
$$\displaystyle\sum_{n=-\infty}^{\infty} (-1)^{-n} J_{-n}(x) J_n(x) = 0,$$
from which we obtain 
$$J_0^2(x) + 2 \displaystyle\sum_{n=1}^{\infty} (-1)^n J_{-n}(x) J_n(x) =1.$$
But $J_{2n}(x) = (-1)^n J_n(x).$ Hence
$$J_0^2(x) + 2 \displaystyle\sum_{n=1}^{\infty} J_n^2(x) = 1.$$
It follows at once, for real $x$, that
$$|J_0(x)| \leq 1,$$
$$|F_n(x)| \leq \dfrac{1}{\sqrt{2}}, n \geq 1.$$
\end{solution}
%%%%
%%
%%
%%%%
\begin{problem}\label{problem7chapter6}
Use Bessel's integral to show that $|J_n(x)| \leq 1$ for real $x$ and integral $n$.
\end{problem}
\begin{solution}
Bessel's integral is
$$J_n(x) = \dfrac{1}{\pi} \displaystyle\int_0^{\pi} \cos(n \theta - x \sin \theta) \mathrm{d} \theta.$$
For real $x$ (and $n$), $|\cos(n \theta - x \sin \theta)| \leq 1$. Hence
$$|J_n(x)| \leq \dfrac{1}{\pi} \displaystyle\int_0^{\pi} \mathrm{d} \theta = 1.$$
\end{solution}
%%%%
%%
%%
%%%%
\begin{problem} \label{problem8chapter6}
By iteration of equation $(8)$, page 111, show that
$$2^m \dfrac{\mathrm{d}^m}{\mathrm{d}z^m} J_n(z) = \displaystyle\sum_{n=0}^m (-1)^{m-k} C_{m,k} J_{n+m-2k}(z),$$
where $C_{m,k}$ is the binomial coefficient.
\end{problem}
\begin{solution}
We have, with $\mathscr{D} = \dfrac{\mathrm{d}}{\mathrm{d}z}$, from (8), page 100,
$$2 \mathscr{D}J_n(z) = J_{n-1}(z) - J_{n+1}(z).$$
Then
$$\begin{array}{ll}
2^2 \mathscr{D}^2J_n(z) &= 2 \mathscr{D}J_{n-1}(z) - 2 \mathscr{D}J_{n+1}(z) \\
&= J_{n-2}(z) - J_n(z) - J_n(z) + J_{n+2}(z) \\
&= J_{n-2}(z) - 2J_n(z) + J_{n+2}(z).
\end{array}$$
Let us use induction. Assume
$$2^m \mathscr{D}^m J_n(z) = \displaystyle\sum_{k=0}^m (-1)^{m-k} C_{m,k} J_{n+m-2k}(z),$$
as we k now is true for $m=1,2.$ Then
\begin{eqnarray*}
\lefteqn{2^{m+1} \mathscr{D}^{m+1}J_n(z)} \\
& &= \displaystyle\sum_{k=0}^m (-1)^{m-k} C_{m,k} [J_{n+m-2k-1}(z) - J_{n+m-2k+1}(z) ] \\
& &= \displaystyle\sum_{k=1}^{m+1} (-1)^{m-k+1} C_{m, k-1} J_{n+m+1-2k}(z) + \displaystyle\sum_{k=0}^m (-1)^{m+1-k} C_{m,k} J_{n+m+1-2k}(z) \\
& &= \displaystyle\sum_{k=1}^m\!(-1)^{m+1-k}\![C_{m,k}\!+\!C_{m,k-1}]\!J_{n\!+\!m\!+\!1\!-\!2\!k}(z)\!+ \!(-1)^{m+1} J_{n+m+1}(z)\!+\!J_{\!n\!-\!m\!-\!1\!}(\!z\!).
\end{eqnarray*}
Now $C_{m,k}+C_{m,k-1} = C_{m+1,k}$ (Pascal's triangle), so that (using the last two terms also),
$$2^{m+1}\mathscr{D}^{m+1}J_n(z) = \displaystyle\sum_{k=0}^{m+1} (-1)^{m+1-k} C_{m+1,k} J_{n+m+1-2k}(z),$$
which completes the induction.
\end{solution}
%%%%
%%
%%
%%%%
\begin{problem} \label{problem9chapter6}
Use the result in Exercise 1, page 105, to obtain the probduct of two Bessel functions of equal argument.
\end{problem}
\begin{solution}
We know already that
$${}_0F_1(-;a;x) {}_0F_1(-;b;x) = {}_2F_3 \left[ \begin{array}{rlr}
\dfrac{a+b}{2}, \dfrac{a+b-1}{2}; & & \\
& & 4x \\
a,b,a+b-1; & & 
\end{array} \right].$$
Then
$$\begin{array}{ll}
J_n(z)J_m(z) &= \dfrac{\left( \dfrac{z}{2} \right)^n \left( \dfrac{z}{2} \right)^m}{\Gamma(n+1) \Gamma(m+1)} {}_0F_1(-;n+1; - \dfrac{z^2}{4}) {}_0F_1(-;m+1;-\dfrac{z^2}{4}) \\
&= \dfrac{\left( \dfrac{z}{2} \right)^{n+m}}{\Gamma(n+1) \Gamma(m+1)} {}_2F_3 \left[ \begin{array}{rlr}
\dfrac{n+m+2}{2}, \dfrac{n+m+1}{2}; & & \\
& & -z^2 \\
n+1,m+1, n+m+1; & & 
\end{array} \right].
\end{array}$$
\end{solution}
%%%%
%%
%%
%%%%
\begin{problem} \label{problem10chapter6}
Start with the power series for $J_n(z)$ and use the form (2), page 18, of the Beta function to arrive at the equation
$$J_n(z) = \dfrac{2 (\frac{1}{2}z)^n}{\Gamma(\frac{1}{2})\Gamma(n+\frac{1}{2})} \displaystyle\int_0^{\frac{\pi}{2}} \sin^{2n} \phi \cos(z \cos \phi) d\phi,$$
for $\Re(n) > -\dfrac{1}{2}.$
\end{problem}
\begin{solution}
We know that
$$J_n(z) = \displaystyle\sum_{k=0}^{\infty} \dfrac{(-1)^k z^{2k+n}}{2^{2k+1} k! \Gamma(k+n+1)} = \displaystyle\sum_{k=0}^{\infty} \dfrac{(-1)^k \left(\dfrac{1}{2} \right)_k z^{2k+n}}{2^n (2k)! \Gamma(k+n+1)}.$$
Also
$$\begin{array}{ll}
\dfrac{\left( \dfrac{1}{2} \right)_k}{\Gamma(k+n+1)} &= \dfrac{\Gamma \left( k + \dfrac{1}{2} \right) \Gamma \left( n + \dfrac{1}{2} \right)}{\Gamma \left( \dfrac{1}{2} \right) \Gamma(k+n+1) \Gamma \left( n + \dfrac{1}{2} \right)} \\
&= \dfrac{B \left( k + \dfrac{1}{2}, n + \dfrac{1}{2} \right)}{\Gamma \left( \dfrac{1}{2} \right) \Gamma \left( n + \dfrac{1}{2} \right)} \\
&= \dfrac{2}{\Gamma \left( \dfrac{1}{2} \right) \Gamma \left( n + \dfrac{1}{2} \right)} \displaystyle\int_0^{\frac{\pi}{2}} \cos^{2k} \phi \sin^{2n} \phi \mathrm{d} \phi.
\end{array}$$
Therefore
$$\begin{array}{ll}
J_n(z) &= \dfrac{2 \left( \dfrac{z}{2} \right)^n}{\Gamma \left( \dfrac{1}{2} \right) \Gamma \left( n + \dfrac{1}{2} \right)} \displaystyle\int_0^{\frac{\pi}{2}} \sin^{2n} \phi \cos(z \cos \phi) \mathrm{d} \phi \\
&= \dfrac{2 \left( \dfrac{z}{2} \right)^n}{\Gamma \left( \dfrac{1}{2} \right) \Gamma \left( n + \dfrac{1}{2} \right)} \displaystyle\int_0^{\frac{\pi}{2}} \sin^{2n} \phi \cos(z \cos \phi) \mathrm{d} \phi.
\end{array}$$
\end{solution}
%%%%
%%
%%
%%%%
\begin{problem} \label{problem11chapter6}
Use the property
$$\dfrac{\mathrm{d}}{\mathrm{d}x} {}_0F_1(-;a;u) = \dfrac{1}{a} \dfrac{\mathrm{d}u}{\mathrm{d}x} {}_0F_1(-;a+1;u)$$
to obtain the differential recurrence relation (6) of Section 60.
\end{problem}
\begin{solution}
We know that $\dfrac{\mathrm{d}}{\mathrm{d}x} {}_0F_1(-;a;u) = \dfrac{1}{a} \dfrac{\mathrm{d}u}{\mathrm{d}x} {}_0F_1 (-;a+1;u).$
Since
$$(1) \hspace{30pt} J_n(z) = \dfrac{\left( \dfrac{z}{2} \right)^n}{\Gamma(1 + n)} {}_0F_1 \left(-;1+n;-\dfrac{z^2}{4} \right)$$
we obtain
$$\begin{array}{ll}
\dfrac{\mathrm{d}}{\mathrm{d}z} \left[ z^{-n} J_n(z) \right] &= \dfrac{1}{2^n \Gamma(1+n)} \dfrac{1}{1+n} \left( - \dfrac{z}{2} \right) {}_0F_1 \left(-;2+n; - \dfrac{z^2}{4} \right) \\
&= -z^{-n} \dfrac{\left( \dfrac{z}{2} \right)^{n+1}}{\Gamma(2+n)} {}_0F_1 \left(-;2+n; -\dfrac{z^2}{4} \right) \\
&= - z^{-n} J_{n+1}(z),
\end{array}$$
which yields $(6)$ of Section~60.
\end{solution}
%%%%
%%
%%
%%%%
\begin{problem} \label{problem12chapter6}
Expand
$${}_0F_1 \left[ \begin{array}{rlr}
-; & & \\
& & \dfrac{2xt - t^2}{4} \\
1+\alpha; & & 
\end{array} \right]$$
in a series of powers of $x$ and thus arrive at the result
$$\left( \dfrac{t-2x}{t} \right)^{-\frac{1}{2} \alpha} J_{\alpha}(\sqrt{t^2-2xt}) = \displaystyle\sum_{n=0}^{\infty} \dfrac{J_{\alpha+n}(t)x^n}{n!}.$$
\end{problem}
\begin{solution}
Consider ${}_0F_1 \left[-;1+\alpha; \dfrac{1}{4}(2xt-t^2) \right]$. We obtain
\begin{eqnarray*}
\lefteqn{{}_0F_1 \left[ \begin{array}{rlr}
- ; & & \\
& & \dfrac{2xt-t^2}{4} \\
1+\alpha; & &
\end{array} \right]} \\
& &= \displaystyle\sum_{n=0}^{\infty} \dfrac{t^n (2x-t)^n}{2^{2n} (1+\alpha)_n n!} \\
& &= \displaystyle\sum_{n=0}^{\infty} \displaystyle\sum_{k=0}^n \dfrac{(-1)^k (2k)^n t^{n+2k}}{w^{2n+2k}(1+\alpha)_{n+k} k! n!} \\
& &= \displaystyle\sum_{n=0}^{\infty} \displaystyle\sum_{k=0}^{\infty} \dfrac{(-1)^k t^{2k}}{2^{2k} k! (1+\alpha+n)_k} \dfrac{t^n (2x)^n}{2^{2n} n! (1+\alpha)_n} \\
& &= \displaystyle\sum_{n=0}^{\infty} {}_0F_1 \left(-;1+\alpha+n; - \dfrac{t^2}{4} \right) \dfrac{ \left( \dfrac{t}{2} \right)^n x^n \Gamma(1 + \alpha)}{n! \Gamma(1 + \alpha + n)}.
\end{eqnarray*}
Now
$$J_{\alpha+n}(t) = \dfrac{\left( \dfrac{t}{2} \right)^{\alpha + n}}{\Gamma(\alpha + n +1)} {}_0F_1 \left(-; 1 + \alpha + n; -\dfrac{t^2}{4} \right),$$
so we obtain
$${}_0F_1 \left[ \begin{array}{rlr}
-; & & \\
& & \dfrac{2xt-t^2}{4} \\
1 + \alpha ; & & 
\end{array} \right] = \Gamma(1 + \alpha) \left( \dfrac{t}{2} \right)^{-\alpha} \displaystyle\sum_{n=0}^{\infty} \dfrac{J_{n+\alpha}(t)x^n}{n!}$$
or
$$\left( \dfrac{\sqrt{t^2-2xt}}{2} \right)^{-\alpha} \Gamma(1 + \alpha) J_{\alpha} \left( \sqrt{t^2 - 2xt} \right) = \Gamma(1 + \alpha) \left( \dfrac{t}{2} \right)^{-\alpha} \displaystyle\sum_{n=0}^{\infty} \dfrac{J_{n + \alpha}(t)x^n}{n!},$$
or
$$\left( \dfrac{t-2x}{t} \right)^{\frac{\alpha}{2}} J_{\alpha} ( \sqrt{t^2-2xt} ) = \displaystyle\sum_{n=0}^{\infty} \dfrac{J_{n + \alpha}(t)x^n}{n!}.$$
\end{solution}
%%%%
%%
%%
%%%%
\begin{problem} \label{problem13chapter6}
Use the realtions (3) and (6) of Section 60 to prove that: For real $x$, between any two consecutive zeros of $x^{-n}F_n(x)$, there lies one and only one zero of $x^{-n}F_{n+1}(x).$
\end{problem}
\begin{solution}
We are given that
$$\dfrac{\mathrm{d}}{\mathrm{d}x} \left[ x^n J_n(x) \right] = x^n J_{n-1}(x),$$
$$\dfrac{\mathrm{d}}{\mathrm{d}x} \left[ x^{-n} J_n(x) \right] = -x^{-n} J_{n+1}(x).$$
We know that $J_n(x)$ has exactly $n$ zeros at $x=0$. Let the others (we have proved there are any) on the axis of reals be at $\alpha_{1,n}, \alpha_{2,n}, \ldots.$ 
The curve $y = x^{-n}J_n(x)$ has its real zeros only at the $\alpha$'s. By Rolle's theorem we see that the zeros $\beta_{1,n_1}, \beta_{2,n_2}, \ldots$ of 
$$y' = -x^{-n}J_{n+1}(x)$$
are such that an odd number of them lie between each two consecutive $\alpha$'s. The curve
$$y_2 = x^{n+1}J_{n+1}(x)$$
has its zeros at $x=0$ and at the $\beta$'s. But
$$y_2' = x^{n+1}J_n(x),$$
so the $\alpha$'s lie between consecutive $\beta$'s.
\end{solution}
%%%%
%%
%%
%%%%
\begin{problem} \label{problem14chapter6}
For the function $I_n(z)$ of Section 65 obtain the following properties by using the methods, but not the results, of this chapter:
$$(1) zI_n'(z) = zI_{n-1}(z) - nI_n(z),$$
$$(2) zI_n'(z) = zI_{n+1}(z) + nI_n(z),$$
$$(3) 2I_n'(z) = I_{n-1}(z) + I_{n+1}(z),$$
$$(4) 2nI_n(z) = z[I_{n-1}(z)-I_{n+1}(z)].$$
\end{problem}
\begin{solution}(Solution by Leon Hall)
$$\begin{array}{ll}
I_n(z) &= i^{-n}J_n(iz) \\
&= \dfrac{\left( \frac{z}{2} \right)^n}{\Gamma(1+n)} {}_0F_1 \left( -; 1+n; \dfrac{z^2}{4} \right) \\
&= \dfrac{\left( \frac{z}{2} \right)^n}{\Gamma(1+n)} \displaystyle\sum_{k=0}^{\infty} \dfrac{1}{(1+n)_k k!} \left( \dfrac{z^2}{4} \right)^k \\
&= \displaystyle\sum_{k=0}^{\infty} \dfrac{z^{2k+n}}{2^{2k+n}k! \Gamma(k+n+1)}.
\end{array}$$
$$(1+n)_k = (1+n)(2+n) \ldots (k+n),$$
so

$$\Gamma(1+n) (1+n)_k = \Gamma(k+n+1).$$
So, as in the method of Section~60,
$$\begin{array}{ll}
\dfrac{\mathrm{d}}{\mathrm{d}z} \left[ z^n I_n(z) \right] &= \displaystyle\sum_{k=0}^{\infty} \dfrac{z^{2k+n2n-1}}{2^{2k+n-1}k! \Gamma(n+k)} \\
&= z^n \displaystyle\sum_{k=0}^{\infty} \dfrac{z^{2k+n-1}}{2^{2k+n-1}k! \Gamma(n+k)} \\
&= z^n \displaystyle\sum_{k=0}^{\infty} \dfrac{z^{2k+n-1}}{2^{2k+n-1}k! \Gamma(k + (n-1)+1)} \\
&= z^n I_{n-1}(z),
\end{array}$$
or
$$z^nI_n'(z) + nz^{n-1}I_n(z) = z^n I_{n-1}(z),$$
which is equivalent to
$$zI_n(z) = zI_{n-1}(z) - nI_n(z),$$
which is $(1)$.

Similarly, 
$$\begin{array}{ll}
\dfrac{\mathrm{d}}{\mathrm{d}z} \left[ z^{-n} I_n(z) \right] &= \dfrac{\mathrm{d}}{\mathrm{d}z} \displaystyle\sum_{k=0}^{\infty} \dfrac{z^{2k}}{2^{2k+n} k! \Gamma(k+n+1)} \\
&= \displaystyle\sum_{k=1}^{\infty} \dfrac{z^{2k-1}}{2^{2k+n-1} (k-1)! \Gamma(k+n+1)} \\
&= \displaystyle\sum_{k=0}^{\infty} \dfrac{z^{2k+1}}{2^{2k+n+1} k! \Gamma(k+n+1+1)} \\
&= z^{-n} I_{n+1}(z),
\end{array}$$
and
$$z^{-n} I_n'(z) - nz^{-n-1} I_n(z) = z^{-n} I_{n+1}(z),$$
or
$$zI_n'(z) = nI_n(z) + z I_{n+1}(z),$$
which is $(2)$.

Adding $(1)$ and $(2)$:
$$2z I_n'(z) = z I_{n-1}(z) + zI_{n+1}(z)$$
or
$$2I_n'(z) = I_{n-1}(z) + I_{n+1}(z),$$
which is $(3)$.

Equating the right sides of $(1)$ and $(2)$:
$$zI_{n-1}(z) - nI_n(z) = zI_{n+1}(z) + nI_n(z),$$
or
$$2nI_n(z) = z[I_{n-1}(z) - I_{n+1}(z)],$$
which is $(4)$.
\end{solution}
%%%%
%%
%%
%%%%
\begin{problem} \label{problem15chapter6}
Show that $I_n(z)$ is one solution of the equation
$$z^2 w'' + zw' - (z^2+n^2)w = 0.$$
\end{problem}
\begin{solution}(Solution by Leon Hall)
Because $I_n$ is a ${}_0F_1$ function times $z^n$, we know from Section~46 that $u={}_0F_1(-;b;y)$ is a solution of
$$y \dfrac{\mathrm{d}^2y}{\mathrm{d}y^2} + b \dfrac{\mathrm{d}u}{\mathrm{d}y} - u = 0,$$
and so ${}_0F_1 \left( -; 1+n; \dfrac{z^2}{4} \right)$ is a solution of
$$z \dfrac{d^2u}{dz^2} + (2n+1) \dfrac{du}{dz} - zu=0.$$
Thus is $w = z^nu$, $I_n(z)$ will be a solution of 
$$z^{-n+1} w'' - 2nz^{-n} w' + n(n+1) z^{-n-1}w + (2n+1) [z^{-n}w'- nz^{-n-1}w] - z^{-n+1}w=0$$
or
$$z^{-n}[zw'' - (2n-2n-1)w' - [-n(n+1)z^{-1} + n(2n+1)z^{-1}+z]w] = 0$$
or
$$z^2 w'' + zw' - (n^2+z^2)w=0.$$
\end{solution}
%%%%
%%
%%
%%%%
\begin{problem} \label{problem16chapter6}
Show that, for $\mathrm{Re}(n) > -\dfrac{1}{2},$
$$I_n(z) = \dfrac{2 \left( \frac{1}{2} z \right)^n}{\Gamma \left(\frac{1}{2} \right) \Gamma \left( n + \frac{1}{2} \right)} \displaystyle\int_0^{\frac{\pi}{2}} \sin^{2n} \phi \cosh(z \cos(\phi)) \mathrm{d} \phi.$$
\end{problem}
\begin{solution}(Solution by Leon Hall)
For $n$ not a negative integer,
$$I_n(z) = i^{-n} J_n(iz),$$
and using the result of Problem~\ref{problem10chapter6},
$$I_n(z) = i^{-n} \dfrac{2(\frac{1}{2}iz)^n}{\Gamma(\frac{1}{2}) \Gamma(n+\frac{1}{2})} \displaystyle\int_0^{\frac{1}{2}\pi} \sin^{2n} \phi \cosh(z \cos \phi) \mathrm{d}\phi.$$
The powers of $i$ cancel, and because $\cos(iw) = \cosh w$ we get
$$I_n(z) = \dfrac{2(\frac{1}{2}z)^n}{\Gamma(\frac{1}{2}) \Gamma(n+\frac{1}{2})} \displaystyle\int_0^{\frac{1}{2}\pi} \sin^{2n} \phi \cosh(z \cos \phi) \mathrm{d} \phi$$
for $\mathrm{Re}(n) > -\dfrac{1}{2}$ as desired.
\end{solution}
%%%%
%%
%%
%%%%
\begin{problem} \label{problem17chapter6}
For negative integral $n$ define $I_n(z) = (-1)^n I_{-n}(z)$, thus completing the definition in Section 65. Show that $I_n(-z) = (-1)^N I_n(z)$ and that
$$\exp \left[ \dfrac{1}{2} z(t+t^{-1}) = \displaystyle\sum_{n=-\infty}^{\infty}I_n(z) t^n. \right]$$
\end{problem}
\begin{solution}(Solution by Leon Hall)
We have
$$\begin{array}{ll}
\displaystyle\sum_{n=-\infty}^{\infty} I_n(z) t^n &= \displaystyle\sum_{n=-\infty}^{-1} (-1)^n I_{-n}(z) t^{-n} + \displaystyle\sum_{n=0}^{\infty} I_n(z) t^n \\
&= \displaystyle\sum_{n=0}^{\infty} (-1)^{n+1} I_{n+1}(z) t^{-n-1} + \displaystyle\sum_{n=0}^{\infty} I_n(z) t^n.
\end{array}$$
Now proceed exactly as in the proof of Theorem~39, the only difference being that $I_n(z)$ involves ${}_0F_1 \left( -; 1+n; \dfrac{z^2}{4} \right)$ whereas $J_n(z)$ involves ${}_0F_1 \left( -;1+n; -\dfrac{z^2}{4} \right)$, to get
$$\displaystyle\sum_{n=-\infty}^{\infty} I_n(z) t^n = \exp \left[ \dfrac{1}{2} z (t+t^{-1}) \right].$$
\end{solution}
%%%%
%%
%%
%%%%
\begin{problem} \label{problem18chapter6}
Use the integral evaluated in Section 56 to show that
$$\displaystyle\int_0^t [\sqrt{x(t-x)}]^n J_n(\sqrt{x(t-x)}) \mathrm{d}x = 2^{-n} \sqrt{\pi} t^{n+\frac{1}{2}} J_{n + \frac{1}{2}} \left( \dfrac{t}{2} \right).$$
\end{problem}
\begin{solution}
\begin{eqnarray*}
\lefteqn{\displaystyle\int_0^t [\sqrt{x(t-x)} ]^n J_n(\sqrt{x(t-x)})\mathrm{d}x} \\
& & = \dfrac{1}{2^n \Gamma(1+n)} \displaystyle\int_0^t x^n(t-x)^n {}_0F_1 \left[ \begin{array}{rlr}
-; & & \\
& & -\dfrac{x(z-x)}{4} \\
1+n;
\end{array} \right] \mathrm{d}x.
\end{eqnarray*}
Now use Theorem~37 with $\alpha = n+1$, $\beta = n+1$, $p=0$, $q=1$, $b_1=1+n$, $c=-\dfrac{1}{4}$, $k=1$, $s=1.$ The result is
\begin{eqnarray*}
\lefteqn{\displaystyle\int_0^t [\sqrt{x(t-x)}]^n J_n(\sqrt{x(t-x)})\mathrm{d}x} \\
& &= \dfrac{1}{2^n \Gamma(1+n)} B(1+n,1+n)t^{2n+1} F \left[ \begin{array}{rlr} 
n+1,n+1; & & \\
& & -\dfrac{t^2}{4 \cdot 4} \\
1+n, \dfrac{2n+2}{2}, \dfrac{2n+3}{2}; & & 
\end{array} \right] \\
& &= \dfrac{\Gamma(1+n) \Gamma(1+n)}{2^n \Gamma(1+n) \Gamma(2+2n)} t^{2n+1} {}_0F_1 \left[ \begin{array}{rlr}
-; & & \\
& & -\dfrac{t^2}{16} \\
n+ \dfrac{3}{2}; & & 
\end{array} \right] \\
& &= \dfrac{\Gamma(1+n) \Gamma \left( n + \dfrac{3}{2} \right) 2^{n+1} \left( \dfrac{t}{r} \right)^{n + \frac{1}{2}} t^{n + \frac{1}{2}}}{\Gamma(2+2n) \Gamma \left( n + \dfrac{3}{2} \right)} {}_0 F_1 \left[ \begin{array}{rlr}
- ; & & \\
& & - \dfrac{\left( \dfrac{t}{2} \right)^2}{4} \\
n + \dfrac{3}{2}; & & 
\end{array} \right] \\
& &= 2^{n+1} t^{n + \frac{1}{2} } \dfrac{\Gamma(1+n) \Gamma \left( \dfrac{3}{2} + n \right)}{\Gamma(2+2n)} J_{n + \frac{1}{2}} \left( \dfrac{t}{2} \right).
\end{eqnarray*}
By Legendre's duplication formula, $\Gamma(2z) = \dfrac{2^{2z-1} \Gamma(z) \Gamma \left( z + \dfrac{1}{2} \right)}{\sqrt{\pi}},$ we get
$$\Gamma(2 + 2n) = \dfrac{2^{1+2n} \Gamma(1+n)\Gamma \left( \dfrac{3}{2} + n \right)}{\sqrt{\pi}}.$$
Hence
$$\begin{array}{ll}
\displaystyle\int_0^t [\sqrt{x(t-x)}]^n J_n(\sqrt{x(t-x)}) \mathrm{d}x &= \dfrac{2^{n+1} t^{n + \frac{1}{2}} \sqrt{\pi}}{2^{1+2n}} J_{n + \frac{1}{2}} \left( \dfrac{t}{2} \right) \\
&= 2^{-n} \sqrt{\pi} t^{n + \frac{1}{2}} J_{n + \frac{1}{2}} \left( \dfrac{t}{2} \right).
\end{array}$$
\end{solution}
%%%%
%%
%%
%%%%
\begin{problem} \label{problem19chapter6}
By the method of Exercise~\ref{problem18chapter6} show that
$$\displaystyle\int_0^1 \sqrt{1-x} \sin(\alpha \sqrt{x}) \mathrm{d}x = \pi \alpha^{-1}J_2(\alpha),$$
and, in general, that
$$\displaystyle\int_0^1 (1-x)^{c-1} x^{\frac{1}{2}n} J_n(\alpha \sqrt{x}) \mathrm{d}x = \Gamma(c) \left( \dfrac{2}{\alpha} \right)^c J_{n+c}(\alpha).$$
\end{problem}
\begin{solution}
Consider
$$\displaystyle\int_0^1 \sqrt{1-x} \sin(\alpha \sqrt{x}) \mathrm{d}x.$$
We know that
$$\sin z = z {}_0F_1 \left(-;\dfrac{3}{2}; -\dfrac{z^2}{4} \right).$$
Hence
$$\displaystyle\int_0^1 \sqrt{1-x} \sin(\alpha \sqrt{x}) \mathrm{d}x = \alpha \displaystyle\int_0^1 x^{\frac{1}{2}} (1-x)^{\frac{1}{2}} {}_0F_1 \left(-;\dfrac{3}{2}; -\dfrac{\alpha^2 x}{4} \right) \mathrm{d}x$$
We now use Theorem 3 with $t=1$, $\alpha = \dfrac{3}{2}$, $\beta = \dfrac{3}{2}$, $p=0$, $q=1$, $b_1 = \dfrac{3}{2}$, $c = -\dfrac{\alpha^2}{4}$, $k=1$, $s=0$. We get
$$\begin{array}{ll}
\displaystyle\int_0^1 \sqrt{1-x} \sin(\alpha \sqrt{x}) \mathrm{d}x &= \alpha B \left( \dfrac{3}{2}, \dfrac{3}{2} \right) \cdot 1^2 F \left[ \begin{array}{rlr}
\dfrac{3}{2}; & & \\
& & -\dfrac{\alpha^2}{4} \\
\dfrac{3}{2}, \dfrac{3}{1}; & & 
\end{array} \right] \\
&= \alpha \dfrac{\Gamma \left( \dfrac{3}{2} \right) \Gamma \left( \dfrac{3}{2} \right)}{\Gamma(3)} {}_0F_1 \left(-;3;-\dfrac{\alpha^2}{4} \right)
\end{array}$$
Now let us turn to
$$\displaystyle\int_0^1 \!(1-x)^{c-1} x^{\frac{1}{2}n} J_n(\alpha \sqrt{x}) \mathrm{d}x = \dfrac{(\frac{\alpha}{2})^n}{\Gamma(1+n)} \displaystyle\int_0^1 \!(1-x)^{c-1} x^n {}_0F_1 \left(-;1+n;-\dfrac{\alpha^2 x}{4} \right) \mathrm{d}x$$
and use Theorem 37 with $\alpha=n+1$, $\beta=c$, $p=0$, $q=1$, $b_1=1+n$, $t=1$, $c=-\dfrac{\alpha^2}{4}$, $k=1$, $s=0.$
The result is
\begin{eqnarray*}
\lefteqn{\displaystyle\int_0^1 (1-x)^{t-1} x^{tn} J_n(\alpha \sqrt{x}) \mathrm{d}x} \\
& &= \dfrac{(\frac{\alpha}{2})^n}{\Gamma(1+n)} B(n+1,c) F \left[ \begin{array}{rlr}
n+1; & & \\
& & -\dfrac{\alpha^2}{4} \cdot 1 \\
1+n, c+n+1;
\end{array} \right] \\
& &= \dfrac{\Gamma(n+1)\Gamma(c)}{\Gamma(1+n)} \dfrac{(\frac{\alpha}{2})^n}{\Gamma(n+c+1)} {}_0F_1 \left(-;c+n+1; - \dfrac{\alpha^2}{4} \right) \\
&&= \Gamma(c) \left( \dfrac{\alpha}{2} \right)^{-c} J_{n+c}(\alpha),
\end{eqnarray*}
as desired.
\end{solution}
%%%%
%%
%%
%%%%
\begin{problem} \label{problem20chapter6}
Show that
$$\displaystyle\int_0^t \exp[-2x(t-x)] I_0[2x(t-x)]\mathrm{d}x = \displaystyle\int_0^t \exp(-\beta^2) \mathrm{d} \beta.$$
\end{problem}
\begin{solution}
Consider $\displaystyle\int_0^t \exp [ -2x(t-x)]I_0[2x(t-x)] \mathrm{d}x.$
Now
$$\exp[-2x(t-x)]I_0[2x(t-x)] = \exp[-2x(t-x)]{}_0F_1[-;1;x^2(1-x)^2].$$
In Kummer's second formula we have
$${}_1F_1(a; 2a; 2z) = e^z {}_0F_1 \left(-; 1 + \dfrac{1}{2}; \dfrac{z^2}{4} \right).$$
Use $a = \dfrac{1}{2}$ and $z = -2x(t-x)$ to get
$$\exp[-2x(t-x)]I_0[2x(t-x)] = {}_1F_1 \left( \dfrac{1}{2}; 1 ; -4x(t-x) \right).$$
Then,
$$\displaystyle\int_0^t \exp[-2x(t-x)]I_0[2x(t-x)]dx = \displaystyle\int_0^t {}_1F_1 \left( \dfrac{1}{2}; 1 ; -4x(t-x) \right) \mathrm{d}x,$$
to which we may apply Theorem 37 with $\alpha=1$, $\beta=1$, $p=1$, $q=1$, $a_1 = \dfrac{1}{2}$, $b_1=1$, $c=-4$, $k=1$, $s=1.$ We thus get
$$\begin{array}{ll}
\displaystyle\int_0^t \exp[-2x(t-x)]I_0[2x(t-x)]\mathrm{d}x &= B(1,1)t F \left[ \begin{array}{rlr}
\dfrac{1}{2}, \dfrac{1}{1}, \dfrac{1}{1}; & & \\
& & -4 \dfrac{t^2}{4} \\
1, \dfrac{2}{2}, \dfrac{3}{2}; & & 
\end{array} \right] \\
&= t {}_1F_1 \left( \dfrac{1}{2}; \dfrac{3}{2}; -t^2 \right) \\
&= \displaystyle\sum_{n=0}^{\infty} \dfrac{(-1)^n \left( \dfrac{1}{2} \right)_n t^{2n+1}}{\left( \dfrac{3}{2} \right)_n n!} \\
&= \displaystyle\sum_{n=0}^{\infty} \dfrac{(-1)^n t^{n+1}}{n! (2n+1)} \\
&= \displaystyle\int_0^t \displaystyle\sum_{n=0}^{\infty} \dfrac{(-1)^n \beta^{2n}}{n!} \mathrm{d} \beta \\
&= \displaystyle\int_0^t \exp(-\beta^2) \mathrm{d} \beta.
\end{array}$$
\end{solution}
%%%%
%%
%%
%%%%
\begin{problem} \label{problem21chapter6}
Show that
$$\displaystyle\int_0^t [x(t-s)]^{-\frac{1}{2}} \exp[4x(t-x)]\mathrm{d}x = \pi \exp \left( \dfrac{1}{2}t^2 \right) I_0 \left( \dfrac{1}{2} t^2 \right).$$
\end{problem}
\begin{solution}
$$\displaystyle\int_0^t [x(t-x)]^{-\frac{1}{2}} \exp[4x(t-x)]\mathrm{d}x = \displaystyle\int_0^t x^{-\frac{1}{2}}(t-x)^{-\frac{1}{2}} {}_0F_0(-;-;4x(t-x))\mathrm{d}x.$$
We use Theorem 37 with $\alpha = \dfrac{1}{2}, \beta = \dfrac{1}{2}, p=q=0, k=1, s=1, c=4:$
$$\begin{array}{ll}
\displaystyle\int_0^t [x(t-x)]^{-\frac{1}{2}} \exp[4x(t-x)]\mathrm{d}x &= B \left( \dfrac{1}{2}, \dfrac{1}{2} \right) t^0 F \left[ \begin{array}{rlr}
\dfrac{1}{2}, \dfrac{1}{2}; & & \\
& & \dfrac{4t^2}{4} \\
\dfrac{1}{2}, \dfrac{2}{2}; & & 
\end{array} \right] \\
&= \dfrac{\Gamma(\frac{1}{2}) \Gamma(\frac{1}{2})}{\Gamma(1)} {}_1F_1 \left( \dfrac{1}{2}; 1 ; t^2 \right) \\
&= \pi \exp \left( \dfrac{1}{2}t^2 \right) {}_0F_1 \left(-;t; \dfrac{t^4}{16} \right) \\
&= \pi \exp \left( \dfrac{t^2}{2} \right) I_0 \left( \dfrac{t^2}{2} \right).
\end{array}$$
\end{solution}
%%%%
%%
%%
%%%%
\begin{problem} \label{problem22chapter6}(Solution by Leon Hall)
Obtain Neumann's expansion
$$\left( \dfrac{z}{2} \right)^n = \displaystyle\sum_{k=0}^{\infty} \dfrac{(n+2k)(n+k-1)!J_{n+2k}(z)}{k!}, n \geq 1.$$
\end{problem}
\begin{solution}
Let 
$$F_n(z) = \displaystyle\sum_{k=0}^{\infty} \dfrac{(n+2k)(n+k-1)!}{k!} \left( \dfrac{z}{2} \right)^{-n} J_{n+2k}(z).$$
Then
\begin{eqnarray*}
\lefteqn{\dfrac{\mathrm{d}}{\mathrm{d}z} \left\{ \left( \dfrac{z}{2} \right)^{-n} J_{n+2k}(z) \right\}} \\
& &= \left( \dfrac{z}{2} \right)^{-n} J_{n+2k}'(z) - \dfrac{n}{2} \left( \dfrac{z}{2} \right)^{-n} J_{n+2k}(z) \\
& &= \left( \dfrac{z}{2} \right)^{-n} \left[ J_{n+2k}'(z) - \dfrac{n}{z} J_{n+2k}(z) \right] \\
& &= \dfrac{(\frac{z}{2})^{-n}}{n+2k} \left[ nJ_{n+2k}'(z) + 2kJ_{n+2k}'(z)-\dfrac{n(n+2k)}{z} J_{n+2k}(z). \right]
\end{eqnarray*}
Using $(8),$ Section 60, and $(1)$, Section 61, and simplifying gives
$$\dfrac{\mathrm{d}}{\mathrm{d}z} \left\{ \left( \dfrac{z}{2} \right)^{-n} J_{n+2k}(z) \right\} = \dfrac{(\frac{z}{2})^{-n}}{n+2k} \left[ kJ_{n+2k-1}(z) - (n+k) J_{n+2k+1}(z) \right].$$
So 
$$F_n'(z) = \left( \dfrac{z}{2} \right)^{-n} \left[ \displaystyle\sum_{k=0}^{\infty} \dfrac{k(n+k-1)!}{k!} J_{n+2k-1}(z) - \displaystyle\sum_{k=0}^{\infty} \dfrac{(n+k)!}{k!} J_{n+2k+1}(z). \right]$$
Note that the $k=0$ term in the first series is zero, and so shifting the index in the first series yields the second series. 

Thus, $F_n'(z)=0$, making $F_n(z)=\mathrm{constant}$. From the structure of the Bessel functions, we see that
$$F_n(0)=\left( \dfrac{z}{2} \right)^{-n} \dfrac{n!}{n!} \left( \dfrac{z}{2} \right)^n = 1$$
and so $F_n(z)=1$, from which Neumann's expansion immediately follows.
\end{solution}
%%%%
%%
%%
%%%%
\begin{problem} \label{problem23chapter6}(Solution by Leon Hall)
Prove Theorem 39, page 113, by forming the product of the series for $\exp \left( \dfrac{1}{2} zt \right)$ and the series for $\exp \left( - \dfrac{1}{2}zt^{-1} \right).$
\end{problem}
\begin{solution}
Theorem~39 is: for $t\neq 0$ and for all finite $z$,
$$\exp \left[ \dfrac{z}{2}  \left(t - \frac{1}{t} \right) \right] = \displaystyle\sum_{n=-\infty}^{\infty} J_n(z) t^n.$$
We know 
$$\exp \left( \dfrac{z}{2} t \right) = \displaystyle\sum_{n=0}^{\infty} \dfrac{z^n}{2^n n!} t^n$$
and
$$\exp \left( - \dfrac{z}{2} t^{-1} \right) = \displaystyle\sum_{m=0}^{\infty} \dfrac{(-1)^m z^m}{2^m m!} t^{-m} = \displaystyle\sum_{n=-\infty}^0 \dfrac{(-1)^n z^{-n}}{2^{-n} (-n)!}t^n.$$
Now, in the product $\left( \displaystyle\sum_{n=0}^{\infty} a_n t^n \right)\left( \displaystyle\sum_{n=-\infty}^0 b_n t^n \right),$
the coefficient of $t^n$, $n \geq 0$, is given by
$$\displaystyle\sum_{k=0}^{\infty} a_{n+k}b_k$$
and the coefficient of $t^{-n}$, $n>0$ is given by
$$\displaystyle\sum_{k=0}^{\infty} a_k b_{n+k}.$$
Thus, in the product $\exp \left( \dfrac{z}{2} t \right) \exp \left( - \dfrac{z}{2} t^{-1} \right),$ or $\exp \left[ \dfrac{z}{2} \left(t - \frac{1}{t} \right) \right]$, the coefficient of $t^n$ for $n \geq 0$ is:
$$\begin{array}{ll}
\displaystyle\sum_{k=0}^{\infty} \left( \dfrac{z^{k+n}}{2^{k+n}(k+n)!} \right) \left( \dfrac{(-1)^k z^k}{2^k k!} \right) &= \displaystyle\sum_{k=0}^{\infty} \dfrac{(-1)^k z^{2k+n}}{2^{2k+n} k! (k+n)!} \\
&= J_n(z),
\end{array}$$
and the coefficient of $t^{-n}$, $n>0$ is:
$$\begin{array}{ll}
\displaystyle\sum_{k=0}^{\infty} \left( \dfrac{z^k}{2^k k!} \right) \left( \dfrac{(-1)^{k+n} z^{k+n}}{2^{k+n}(k+n)!} \right) &= (-1)^n \displaystyle\sum_{k=0}^{\infty} \dfrac{(-1)^k z^{2k+n}}{2^{2k+n}k! (k+n)!} \\
&= (-1)^n J_n(z) \\
&= J_{-n}(z).
\end{array}$$
\end{solution}